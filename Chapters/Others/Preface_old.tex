
\chapter*{Preface}
\markboth{Preface}{Preface}
\addcontentsline{toc}{chapter}{Preface} 



The Standard Model of particle physics is both incredibly successful and glaringly incomplete theory. 
It brings together all elementary particles that make up the known universe, which constitutes just about a 5\%, in a single theory describing three of the four
fundamental interactions.
Chapter~\ref{chap:Introduction} presents the elements and scope of the Standard Model.
Among these particles, the top quark and the Higgs boson are of special
interest because they can help to answer some of the open questions, such as the matter-antimatter asymmetry.  
The object of study of this thesis focusses in these two 
singular particles and its interplay, which is discussed in Chapter~\ref{chap:topHiggsPhysics}.

The discovery of a Higgs boson by the ATLAS and CMS experiments in 2012 opened 
a new field for exploration in the realm of particle physics. In order to better understand the
Standard Model, it is of prominent interest to determine the Yukawa coupling of 
the Higgs boson to the top quark, being the latter the most massive fundamental particle 
and, consequently, the one with the largest coupling to the Higgs boson.



The direct measurement of the Yukawa coupling is only possible at the LHC via two associated
Higgs-boson productions: with a top-quark--antiquark pair (\ttH) and with a single top quark.
While the former process just permits the determination of the magnitude of the coupling, the only way of simultaneously 
measuring its sign and magnitude is through the latter one. The possible observation of an excess 
of signal events with respect to the Standard Model prediction, would be an 
evidence of new physics in terms of \CP-violating coupling.

The studies presented at this dissertation are carried using an integrated luminosity of 140~fb$^{-1}$ 
of proton--proton collision data at centre-of-mass energy of 13~TeV collected by the ATLAS detector during the 
LHC Run 2. Located at the European Organisation for Nuclear Research,
the LHC is the most powerful particle accelerator in the world and ATLAS one of its largest detectors. The experimental 
setup in which this work is contextualised is described in Chapter~\ref{chap:ATLAS}. The data and generation
of Monte Carlo simulations within the ATLAS detector is described in Chapter~\ref{chap:DataAndMC}. The reconstruction 
and identification of physical objects is explained in Chapter~\ref{chap:ObjectReconstruction}.

In this work it is presented a search for the associated Higgs-boson production with 
a single top quark with an additional parton (\tHq) in a final state with two light-flavoured-charged 
leptons (electrons or muons) and one hadronically-decaying \Ptau-lepton (named \dileptau channel).
This search is exceptionally challenging due to the extremely small cross-section of the \tHq process, $\mathcal{O}(10^2$~fb) 
at centre-of-mass energy of $13$~TeV, 
and also because the \dileptau final-state channel, in particular, only accounts 
for a 3.5\% of the total \tHq production. 
% Cross section 70~fb \cite{LHCHiggsCrossSectionWorkingGroup:2016ypw}

Therefore, machine-learning techniques are used to distinguish the \tHq signal events from 
the background. Particularly, boosted-decision trees are employed to define signal-enriched regions as well
as control regions that constrain the most important background processes. 

Additionally, to help identifying signal events within the data, the reconstruction of the event plays an important role. 
Different tools are used to reconstruct the four momenta of the top quark and Higgs boson from the 
reconstructed objects. This information can be later used to build variables that help separating the
signal events from the processes that mimic the \dileptau signature. 
The reconstruction of the events is also enhanced by similar machine-learning methods
since in the scenario in which the light-flavour leptons have the same sign, a priori, 
it is not possible to determine which lepton is originated from the Higgs boson and 
which from the top quark. 

Significant suppression of the background events with jets wrongly selected as leptons 
or non-prompt leptons originating from heavy-flavour decays
is achieved by demanding electrons and muons to pass strict identification and isolation requirements. 
Simultaneously, hadronic $\Ptau$-leptons are demanded to pass the requirement of a
recurrent-neural-network-based discriminator to reduce misidentifications from jets.

The tools and methods developed to search for the associated \tHq production 
search are described in Chapter~\ref{chap:Analysis_tH}, where the results
are presented as well. The conclusion of the thesis is presented in Chapter~\ref{chap:Conclusion}.

% ATLAS style guide: Generalhttps://cds.cern.ch/record/1110290/files/gen-pub-2008-001.pdf
% Notation for taus: https://twiki.cern.ch/twiki/pub/AtlasProtected/EditorialBoardGuidelines/tau-notation.pdf

%\pablo{Conceptos triviales que también debería introducir en algún momento \url{http://opendata.atlas.cern/books/current/get-started/_book/GLOSSARY.html}}

