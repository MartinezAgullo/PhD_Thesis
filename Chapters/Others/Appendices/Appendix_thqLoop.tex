\chapter{The \texttt{tHqLoop} framework for post-processing}
\label{chap:Appendix:tHqLoop}


The simulation of the physical process that takes place within ATLAS detector is presented in detail
in Section \ref{sec:Chap3.1:MC}. All these steps take place within the ATLAS software infrastructure,
Athena. 
The simulated event data samples are produced using  the \texttt{SingleTopAnalysis} framework, 
which is constructed on top of \texttt{AnalysisTop}, which is part of Athena. 
The output of this production is referred as Single-top NTuples and its generation is executed centrally on 
the grid for multiple top-quark-focussed analyses. Afterwards, the Single-top NTuples are transferred to the local sites.
%~\cite{SingleTopAnalysis}
%~\cite{AnalysisTop}

After this stage, a post-processing software developed specifically for the \tHq search performs, locally,
several more channel-specific tasks. This software, known as \thqloop, loops over all the events
from the Single-top NTuples to perform channel-specific tasks.
Examples of these tasks are the application of tighter object selections, selecting events which are more signal-like,
the event reconstruction or the computation of additional variables that are needed in the subsequent analysis.
The \thqloop framework also applies the correct configurations like scale factors, systematic uncertainties and overlap removal.
Other channel-specific tasks include the assignment of the
light-lepton origin in the \dilepSStau channel (see Section~\ref{sec:ChaptH:Sig:LepAsign}) or the application of
scale factors that correct the \tauhad misidentification rate in the \dileptau and \lepditau channels.
It is precisely with the \thqloop framework that the selection of the final-state objects are selected 
and, hence, the different channels defined. 



