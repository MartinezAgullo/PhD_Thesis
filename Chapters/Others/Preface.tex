
\chapter*{Preface}
\markboth{Preface}{Preface}
\addcontentsline{toc}{chapter}{Preface} 


%%% STRUCTURE: IDEA 1
% Introduction: 	SM has many open questions
% 			Top quark and top Higgs are very singular particles that can help answer these questions. These
%			are the object of analysis of this work.
% Higgs: 	New field of research
% Top: 	Window for new phys, 
%
% Analsis: Contributions to two analyses are presented in this work.
%		- tHq studies: motivation to do so, challenging, channels, MVA
%		- Polarisation:  	In QCD tops are unpolarised but in EW and tWb vertex can be only studied in top decay
%					In EW top quarks are highly polarised due to the V-A nature and the tWb vertex can be
%					studied in both production and decay
% 					Lifetime ($10^{-25}\,$s) << depolarisation scale ($10^{-21}\,$s) , therefore, the decay
%					products preserve the spin information of the top quark.
%					This information is accesible via angular distributions in the top quark rest frame
%					


The Standard Model of particle physics is both incredibly successful and glaringly incomplete theory. It brings together all  
elementary particles that make up the known universe in a single theory. Among these, the top quark and the Higgs boson are of special
interest because they can help to answer some of the open questions. The object of study of this thesis focusses in these two 
singular particles and its interplay. 

%This is done trough two different analyses. On one hand the measurement of the top-quark polarisation
%and, on the other hand, the search of the associated production of a Higgs boson and a single-top quark. 

The studies presented at this dissertation have been carried using an integrated luminosity of 139$\,$fb$^{-1}$ 
of proton-proton collision data at center-of-mass energy of 13$\,$TeV collected by the ATLAS detector during the 
Large Hadron Collider (LHC) Run 2.  Located at the European Organization for Nuclear Research,
the LHC is the most powerful particle accelerator in the world and ATLAS on of its largest detectors. The experimental 
setup in which this work is contextualised is described in Chapter \ref{chap:ATLAS}. The data and generation
of Monte Carlo simulations within ATLAS is described in Chapter \ref{chap:DataAndMC}. The reconstruction 
and identification of physical objects is explained in Chapter \ref{chap:ObjectReconstuction}.

\begin{comment}
%Measurement of polarisation
\paragraph{Measurement of the top-quark polarisation}\mbox{}\\
The top quark is the heaviest elementary particle discovered so far, with a mass of cd
$\mtop = 172.76 \pm 0.30 \,$GeV \cite{pdgTop}. Due to its large mass, it presents some unique features. 
Being one of the most relevant the fact that it is the only quark that decays before hadronising. This allows to 
measure its properties through its decay products. At LHC the top quark can be produced either via the strong
interaction, resulting in quark-antiquark pairs (\ttbar), or by the electroweak interaction, which produces them singly. 
In the second scenario, the single-top quarks are created with their spin aligned in a particular direction.
My contribution to the measurement of the polarisation observables is presented in Chapter \ref{chap:Polarisation}.
The result of this work is presented in reference \cite{ATLAS:2021vhx}.




%Search of tHq production%
\paragraph{Search for associated production of a Higgs boson and a single top quark}\mbox{}\\
\end{comment}

The discovery of a Higgs boson by the ATLAS \cite{20121_ATLAS_HiggsDiscovery} and CMS \cite{201230_CMS_HiggsDiscovery} experiments in 2012 opened 
a new field for exploration in the realm of particle physics. In order to better understand 
Standard Model, it is of prominent interest to determine the Yukawa coupling of 
the Higgs boson to the top quark (\yt), being the latter the most massive fundamental particle 
and, consequently, the one with the largest coupling to the Higgs boson.

% Higgs boson prediction by Englert and Brout \cite{Englert:1964et}, and Higgs \cite{Higgs:1964pj}

The direct measurement of \yt is only possible at the LHC via two associated
Higgs productions: with a top-quark-antiquark pair (\ttH) and with a single-top quark with an additional parton (\tHq).
While the \ttH just permits the determination of the magnitude of \yt, the only way of simultaneously 
measuring its sign and magnitude is through the \tH production \cite{Demartin:2015uha}. The possible observation of an excess 
of signal events with respect to the Standard Model prediction, would be an 
evidence of new physics in terms of CP-violating \yt coupling.

In this work it is presented a search for the \tHq production in a final state with two light-flavoured-charged 
leptons (electrons or muons) and one hadronically-decaying \Ptau lepton (named \dileptau$\,$ channel).
This search is exceptionally challenging due to the extremely small cross-section of the \tHq process
(70$\,$fb \cite{LHCHiggsCrossSectionWorkingGroup:2016ypw}), and of the \dileptau final-state channel, in particular, which only accounts 
for a 3.5\% of the total \tHq production.

Therefore, to distinguish the \tHq signal events from background events, machine-learning techniques
are used. Particularly, boosted-decision trees are employed to define signal-enriched regions as well
as control regions that constrain the most important background precesses. The most relevant backgrounds
are those related to top-quark-antiquark-pair production without and with and additional boson
(\ttbar, $t\bar{t}H$, $t\bar{t}W$ and $t\bar{t}Z$) and $Z$ boson plus jets. 

Additionally, to help identifying signal events within the data,  the reconstruction of the event plays an important role. 
Different tools are used to retrieve the four momentum of the top quark and Higgs boson from the 
reconstructed objects. This information can be later used to build variables that help separating the
signal events from the processes that mimic the \dileptau signature. 
The reconstruction of the events is also enhanced by similar machine-learning methods
since in the scenario in which the light-flavour leptons have the same sign, a priori, 
it is not possible to determine which lepton is originated from the Higgs boson and 
which from the top quark. 

Significant suppression of the background events with jets wrongly selected as leptons  
or non-prompt leptons originating from heavy-flavour decays
is achieved by demanding electrons and muons to pass strict identification and isolation requirements. 
Simultaneously, hadronic-$\tau$ leptons are demanded to pass the requirement of a
recurrent-neural-network-based discriminator to reduce misidentifications from jets.

The tools and methods developed for the associated \tHq production search are described in Chapter \ref{chap:Analysis_tH}.







%- \ttH sensitive to the \yt module \\
%- \tHq sesntive both to the sign and magnitude of \yt


%The top quark is the heaviest elementary particle discovered so far, with a mass of 
%\ $\mtop = 172.76 \pm 0.30$ GeV \cite{pdgTop}, followed by the Higgs boson, with 
%$\mH = 125.25 \pm 0.17$ GeV \cite{pdgHiggs}.
%The main topic of this thesis is the study of the top quark, specially its interaction with the Higgs boson.
%This project has been developed in the context of ATLAS experiment, which is one of the four detectors 
%operating at the Large Hadron Collider (LHC) oft the European Organization for Nuclear Research (CERN). 
 

%This thesis describes the search for the associated production of a Higgs boson and a single top quark ($\tHq$) in
%the final state with three leptons, being two of them light leptons (\Pe / $\mu$) and the other one a 
%hadronically-decaying  tau quark (\tauhad). The examined channel is referred as \dileptau and is further divided in
%two sub-channels attending to the relative electric charge between the light leptons (\dilepSStau and \dilepOStau).
 %The research carried targets SM Higgs bosons decaying to \Htautau, \HWW and \HZZ.

%This analysis uses an integrated luminosity of 139 fb$^{-1}$ of proton-proton collision data at centre-of-mass energy
%of 13 TeV collected by ATLAS during LHC Run 2. The ATLAS detector is one of the four detectors at the Large
%Hadron Collider (LHC) of the European Organization for Nuclear Research (CERN). 


%This analysis presents a great challenge due to the difficulty of distinguish the \tHq production in the \dileptau 
%channels from other SM processes with much higher cross-sections. The \tHq process has an extremely small 
%cross-section ($70\,$fb at $13\,$TeV) and the \dileptau channel accounts only for the 3.5\% of the total production.



%Additional studies are carried  to assign origin the light-flavoured leptons, i.e. whether these came from the Higgs
%boson or the top quark.  To do so, a gradient boosted decision tree is used. 

%Moreover, due to the arduousness of separating the \tHq from the backgrounds, multivariate techniques are 
%used. With them are defined signal-enriched regions and control regions to constrain the most important 
%background processes, which are those related to top-quark-antiquark-pair production (\ttbar, \ttH, 
%\ttW and \ttZ) and \PZ boson plus jets.



%The possible observation of a an excess of signal events with respect to the Standard Model predictions, would be a 
%clear evidence of new physics in terms of \CP-violating Yukawa couplings between the top quark and the Higgs 
%boson.




%The \tHq search has required the development not only of the simulation and analysis tools (MC generated samples
%for signal and background, analysis software, etc) but also of the methodology (mva methods).


%Furthermore, other channels (defined by their final state) are being studied in the context of \tHq research


%\paragraph{Other considerations}\mbox{}\\
%During this thesis the ``god-given'' units are used. In this system (name natural system) the Planck constant and the 
%speed of light have the same magnitude and dimensions ($\hbar=c=1$), implying that: 
%\begin{center}
%[length] = [time] = [energy]$^{-1}$ = [mass]$^{-1}$.
%\end{center}
%In the international system (SI) units, the values are $\hbar = 1.055 \times10^{-34}$ Js and $c = 2.998 \times10^{8}$ m/s. 
%An exception is made for the Chapter \ref{chap:ATLAS}, where the SI is used for describing the design of the LHC
%and ATLAS.

%\pablo{Conceptos triviales que también debería introducir en algún momento \url{http://opendata.atlas.cern/books/current/get-started/_book/GLOSSARY.html}}

