
\chapter*{Preface}
\markboth{Preface}{Preface}
\addcontentsline{toc}{chapter}{Preface} 

This thesis documents a physics analysis of crucial importance for the ATLAS (A Toroidal LHC ApparatuS) experiment at the Large Hadron
Collider (LHC) at the CERN laboratory: the cross-section measurement of the production of the Higgs boson in association with a single top
quark. This process involves the two most special elementary particles: the Higgs boson, which gives mass to all fundamental particles in
the Standard Model (SM), and the top quark, which is the heaviest elementary particle and thus the one with the strongest coupling to the Higgs
boson.

The coupling between the Higgs boson and the top quark is one of the most intriguing parameters of the SM. While its magnitude has already been measured, its sign is still unknown. The only way to determine its relative sign is through the production of a single top quark in association with a Higgs boson. This thesis aims to measure the rate of the \tHq production targeting the final state with two light-flavoured charged-leptons 
and one hadronically-decaying $\Ptau$-lepton (named \dileptau channel).
The possible observation of an excess of signal events with respect to the SM prediction would be evidence of new physics in terms of CP-violating coupling and it would help to explain one of the current limitations of the SM, the matter--antimatter asymmetry. 

The study presented uses 140~fb$^{-1}$ of proton--proton collision data at a centre-of-mass energy 
of 13~TeV from the LHC Run 2 collected by the ATLAS detector. 
This search is exceptionally challenging due to the extremely small cross-section of the \tHq process, $\mathcal{O}(10^2$~fb), of
which only 3.5\% of events have a \dileptau final state. Additionally, the large presence of processes that can mimic the signal
signature in the detector (known as backgrounds) makes it even more difficult to select a pure dataset for the analysis. 
Therefore, machine-learning techniques are used to distinguish the \tHq 
signal events from the background as well as to distinguish the main background processes. 
Particularly, boosted-decision trees are employed to define the signal, control, and validation orthogonal regions of 
the phase space used in the analysis. 

The analysis is further split into two categories depending on whether the light-flavoured charged-leptons have the same charge (\dilepSStau) or opposite charge (\dilepOStau). This split is motivated by the fact that the background composition is very different for the two cases, requiring basically two completely different data analyses. For the \dilepSStau sophisticated multivariate analysis is used for the light-lepton-origin assignment, which is to match to its parent which can be either the top quark or the Higgs boson.

The measured signal strength in the two channels is found
to be compatible with the predictions from the Standard Model; upper limits on these cross-sections are also reported. 
This result significantly contributes to getting closer to a statistical combination of all channels of the process \tHq process being
currently measured within the ATLAS experiment.
This thesis constitutes an important step towards the first observation of this process, which will require more data (probably even more than what will collected during LHC Run 3) and a better understanding of the main systematic uncertainties


Chapter~\ref{chap:Introduction} introduces the fundamental concepts 
and scope of the SM, setting the stage for the in-depth discussion
that follows in the subsequent chapters.
The characteristics of the top quark and the Higgs boson are discussed in
Chapter~\ref{chap:topHiggsPhysics}. In this chapter, the interaction
between these two particles is described in detail, with 
special consideration to the scenario with \CP-violating interaction.
The experimental setup in which this work is contextualised is described in Chapter~\ref{chap:ATLAS}. 
 The chapter provides a comprehensive overview of the LHC, the world's most powerful 
 particle accelerator nowadays, and the ATLAS machine, its largest detector.
The phenomenology of proton--proton collisions and generation of Monte Carlo simulations within the ATLAS 
detector are covered in Chapter~\ref{chap:DataAndMC}. 
Chapter~\ref{chap:ObjectReconstruction} focusses on the methodologies for the reconstruction and identification of physical objects.
The strategies, tools and methods developed to set the best limits to the production cross-section of the \tHq process in 
both the \dilepOStau and the \dilepSStau channels ults of this thesis are also presented in this chapter.
Finally, the conclusion of the \tHq search is presented in Chapter~\ref{chap:Conclusion}.


\begin{comment}
At the heart of our universe lies a captivating puzzle, a tale of fundamental particles so profoundly mysterious. This thesis is an odyssey into the world of the smallest constituents of matter, guided by the beacon of the Standard Model of particle physics (SM). While this model stands as a titan in our understanding of the universe, it harbours a secret – an enigmatic incompleteness that invites the curious mind.

Enter the two protagonists of our story: the top quark, the most massive particle in the SM, and the Higgs boson.
The latter famously unveiled in 2012 by the ATLAS and CMS experiments, marking a new era in particle physics, and the boson of the field that gives mass to the fundamental particles. These particles are not just any characters in the cosmic drama; they may hold the keys to unlocking one of the most perplexing mysteries of our time – the matter--antimatter asymmetry. This thesis is dedicated to unraveling their intricate dance, a performance that is pivotal to our understanding  of the SM.

In the colossal arena of the Large Hadron Collider (LHC), the world's most powerful particle accelerator, we witness a spectacle like no other. The ATLAS detector, a marvel of human ingenuity, serves as our lens into this subatomic world. Our quest? To better understand the Yukawa coupling of the Higgs boson to the top quark, a venture that could illuminate new realms of physics and challenge the SM itself.
The focus here is to establish the most precise limits on the associated production of a single-top quark and a Higgs boson (\tHq production) in the final state characterised by two light leptons (\emu) and one hadronically decaying \Ptau lepton (\tauhad), a channel known as \dileptau.

This is not a journey for the faint-hearted. We delve into the rarest of events using an integrated luminosity of 140~fb$^{-1}$ 
of proton--proton collision data at centre-of-mass energy of 13~TeV.  Armed with cutting-edge machine learning techniques such as boosted-decision trees, we sift through the cosmic haystack in search of these elusive needles that are the \tHq evetns in the \dileptau final state.


Our expedition is charted across several chapters, each a stepping stone towards understanding the universe's deepest secrets.
Chapter~\ref{chap:Introduction} introduces the fundamental concepts 
components and scope of the SM, setting the stage for the in-depth discussion
that follows in the subsequent chapters.
The characteristics of the top quark and the Higgs boson are discussed in
Chapter~\ref{chap:topHiggsPhysics}. In this chapter, the interaction
between the these two particles is described in detail, with 
special consideration to the scenario with \CP-violating interaction.
The experimental setup in which this work is contextualised is described in Chapter~\ref{chap:ATLAS}. 
 The chapter provides a comprehensive overview of the Large Hadron Collider, the world's most powerful 
 particle accelerator, and the ATLAS detector, its largest experiment.
The phenomenology of proton--proton collisions and generation of Monte Carlo simulations within the ATLAS 
detector are covered in Chapter~\ref{chap:DataAndMC}. 
Chapter~\ref{chap:ObjectReconstruction} focuses on the methodologies for the reconstruction and identification of physical objects.
The strategies, tools, and methods developed for the \tHq production search are described
 in Chapter~\ref{chap:Analysis_tH}.
The results of this thesis are also presented in this chapter.
Finally, the conclusion of the \tHq search is presented in Chapter~\ref{chap:Conclusion}.

So, dear reader, brace yourself for a journey into the heart of matter, a tale of particles and their profound secrets, a quest driven by curiosity and the relentless pursuit of knowledge. Welcome to a story of cosmic proportions, where every discovery is a step closer to unraveling the mysteries of our universe.
\end{comment}



\begin{comment}
% Introduction to the topic
The Standard Model of particle physics is both incredibly successful yet glaringly incomplete theory. 
It brings together all elementary particles that make up the known universe in a single theory describing three of the four
fundamental interactions.
Among these particles, the top quark and the Higgs boson are of special
interest because they can help to answer one of the open questions, the matter-antimatter asymmetry.  
The object of study of this thesis focusses on these two singular particles and their interplay.  

The discovery of a Higgs boson by the ATLAS and CMS experiments in 2012 opened 
a new field for exploration in the realm of particle physics. In order to better understand the
SM, it is of prominent interest to determine the Yukawa coupling of 
the Higgs boson to the top quark, being the latter the most massive fundamental particle 
and, consequently, the one with the largest coupling to the Higgs boson.
%The motivation for conducting this research lies in addressing fundamental gaps in our understanding of the
%SM.
 
The direct measurement of the Yukawa coupling is only possible at the LHC via two associated
Higgs-boson productions: with a top-quark--antiquark pair (\ttH) and with a single top quark.
While the former process only permits determination of the magnitude of the coupling, 
the latter one uniquely allows for simultaneous measurement of both its sign and magnitude.
The possible observation of an excess 
of signal events with respect to the SM prediction, would be  
evidence of new physics in terms of \CP-violating coupling.
%\cite{Demartin:2015uha}

% Introduction to the analysis
The studies presented at this dissertation are carried using an integrated luminosity of 140~fb$^{-1}$ 
of proton--proton collision data at centre-of-mass energy of 13~TeV collected by the ATLAS detector during the 
LHC Run 2. Located at the European Organisation for Nuclear Research,
the LHC is the most powerful particle accelerator in the world and ATLAS one of its largest detectors.

In this work, a search for the associated Higgs-boson production with 
a single top quark with an additional parton (\tHq) in a final state with two light-flavoured-charged 
leptons (electrons or muons) and one hadronically-decaying \Ptau-lepton (named \dileptau channel) is presented.
This search is exceptionally challenging due to the extremely small cross-section of the \tHq process, $\mathcal{O}(10^2$~fb) 
at centre-of-mass energy of 13~TeV, 
and also because the \dileptau final-state channel, in particular, only accounts 
for a 3.5\% of the total \tHq production. 
% Cross section 70~fb \cite{LHCHiggsCrossSectionWorkingGroup:2016ypw}

Therefore, machine-learning techniques are used to distinguish the \tHq signal events from 
the background. Particularly, boosted-decision trees are employed to define the regions of 
the phase space used in the analysis. 
The reconstruction of the events is also enhanced by similar machine-learning methods. 


%Organisation of the thesis
Chapter~\ref{chap:Introduction} introduces the fundamental concepts 
components and scope of the SM, setting the stage for the in-depth discussion
that follows in the subsequent chapters.
The characteristics of the top quark and the Higgs boson are discussed in
Chapter~\ref{chap:topHiggsPhysics}. In this chapter, the interaction
between the these two particles is described in detail, with 
special consideration to the scenario with \CP-violating interaction.
The experimental setup in which this work is contextualised is described in Chapter~\ref{chap:ATLAS}. 
 The chapter provides a comprehensive overview of the Large Hadron Collider, the world's most powerful 
 particle accelerator, and the ATLAS detector, its largest experiment.
The phenomenology of proton--proton collisions and generation of Monte Carlo simulations within the ATLAS 
detector are covered in Chapter~\ref{chap:DataAndMC}. 
Chapter~\ref{chap:ObjectReconstruction} focusses on the methodologies for the reconstruction and identification of physical objects.
The strategies, tools and methods developed to search for the \tHq production 
search are described in Chapter~\ref{chap:Analysis_tH}.
The results of this thesis are also presented in this chapter.
Finally, the conclusion of the \tHq search is presented in Chapter~\ref{chap:Conclusion}.
\end{comment}

% ATLAS style guide (general): https://cds.cern.ch/record/1110290/files/gen-pub-2008-001.pdf
% Notation for taus: https://twiki.cern.ch/twiki/pub/AtlasProtected/EditorialBoardGuidelines/tau-notation.pdf
% http://opendata.atlas.cern/books/current/get-started/_book/GLOSSARY.html 

