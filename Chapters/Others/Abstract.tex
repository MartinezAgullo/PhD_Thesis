
\chapter*{Contraportada}
\markboth{Abastract}{Abastract}
%\addcontentsline{toc}{chapter}{Abastract} 

%TEXTO PARA CONTRAPORTADA

The discovery of a Higgs boson by the ATLAS and CMS experiments in 2012 
opened a new field for exploration in the realm of particle physics.
In order to better understand the Standard Model (SM) of particle physics, 
it is imperative to study the Yukawa coupling between this new particle and
the rest of the SM components. 
Among these, of prominent interest is the coupling of the Higgs boson to the top 
quark (\yt), which is the most massive fundamental particle and, consequently, the one 
with the strongest coupling to the Higgs

The direct measurement of \yt is only possible at LHC via two associated Higgs 
productions; with a pair of top quarks (\ttH) and with single-top quark (\tHq).
While the \ttH permits a model-independent determination of the magnitude of \yt, 
the only way of directly measuring its sign is through the \tHq production. This is due 
to the fact that the two leading-order Feynman diagrams for \tHq production interfere 
with each other depending on \yt sign. 
Current experimental constrains on \yt favour the SM predictions, even thought an 
opposite sign with respect to the SM expectations is not completely excluded yet.

In this work it is presented a search for the production of a Higgs boson in association 
with a single-top quark in a final state with two light-floavoured-charged leptons and one 
hadronically decaying tau lepton (named \dileptau channel). This analysis uses an integrated 
luminosity of 139 fb$^{-1}$ of proton--proton collision data at centre-of-mass energy of 
13$\,$TeV collected by ATLAS during LHC Run 2. 


This search is exceptionally challenging due to the extremely small cross-section of 
the \tHq process (70$\,$fb$^{-1}$) in general and, more particularly, the the \dileptau
final-state channel, which only accounts for a 3.5\% of the total \tHq production.

Because of this, the separation of the \tHq signal events from background events is 
done by means of machine-learning (ML) techniques using boosted-decision trees 
(BDT) to define both signal and control regions. The most most relevant background 
processes are those related to top-pairs production (such as \ttbar, \ttH, \ttZ and 
\ttW), \PZ-boson plus jets and single-top processes.


Significant suppression of the background events with jets wrongly selected as leptons 
is achieved by demanding electrons and muons to pass strict isolation requirements. 
Simultaneously, hadronic-tau leptons are demanded to pass the requirement of the
recurrent-neural-network-based discriminator to reduce misidentifications from jets.

The reconstruction of the events is also enhanced by similar ML methods
since in the scenario in which the light-flavour leptons have the same sign, in principle, it is not
possible to determine which lepton is originated from the Higgs boson and which from the
top quark.

The possible observation of an excess of signal events with respect to the SM predictions, 
would be a clear evidence of new physics in terms of \CP-violating \yt coupling.


Additionally, the contribution to the single-top-quark polarisation first measurement is presented as well. 
In this other analysis the components of the full polarisation vector of the top quark are measured
taking advantage of the peculiarities of the single-top-quark decay. Benefitting from the fact that
the top quark lifetime is smaller than the depolarisation timescale, the decay products preserve
the spin information of the top quark. Via angular distributions, it is measured in the top-quark rest
frame. 









