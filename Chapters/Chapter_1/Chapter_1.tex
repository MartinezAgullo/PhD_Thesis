%PREAMBLE
\begin{comment}
%\documentclass[11pt]{article}
%\newcommand{\pablo}[1]{\textcolor{blue}{{\bf  #1}}}
\newcommand{\carlos}[1]{\textcolor{red}{{\bf  #1}}}
\newcommand{\fabio}[1]{\textcolor{purple}{{\bf  #1}}}
\newcommand{\susana}[1]{\textcolor{violet}{{\bf  #1}}}

% HEP names :: https://ctan.javinator9889.com/macros/latex/contrib/hepnames/hepnames.pdf

\DeclarePairedDelimiter\bra{\langle}{\rvert}
\DeclarePairedDelimiter\ket{\lvert}{\rangle}
\DeclarePairedDelimiterX\braket[2]{\langle}{\rangle}{#1 \delimsize\vert #2}

\newcommand*{\yt}{\ensuremath{y_{t}}\xspace}
\newcommand*{\tchannel}{\ensuremath{t\text{-channel}}\xspace}
\newcommand*{\schannel}{\ensuremath{s\text{-channel}}\xspace}
%\newcommand*{\lepT}{\ensuremath{\Pl_{\Pt}}\xspace}
%\newcommand*{\lepH}{\ensuremath{\Pl_{\PHiggs}}\xspace}

%\newcommand*{\muR}{\ensuremath{\mu_{\text{R}}}\xspace}
%\newcommand*{\muF}{\ensuremath{\mu_{\text{F}}}\xspace}

% Add external packages
\usepackage[italic]{hepnicenames}

%%%%%%%%%%%%%%%%%%%%%%%
%  From NA-HIGG-2020-02-INT1-defs.sty    %
%%%%%%%%%%%%%%%%%%%%%%%
% Basic tHq-related macros
%\newcommand*{\tHq}{\ensuremath{\Pqt{}\PH{}\Pq}\xspace}
%\newcommand*{\tHq}{\ensuremath{\Ptop \PHiggs \Pq}\xspace}
%\newcommand*{\tHq}{\Pqt{}\PH{}\Pq}
\newcommand*{\tHq}{\ensuremath{tHq}\xspace}
\newcommand*{\tH}{\ensuremath{\Pqt{}\PH{}}\xspace}
\newcommand*{\tHqsec}{\texorpdfstring{\Pqt{}\PH{}\Pq}{tHq}}
\newcommand*{\tHqML}{\ensuremath{\Pqt{}\PH{}\Pq\,(\text{ML})}\xspace}
\newcommand*{\tHqbb}{\ensuremath{\Pqt{}\PH{}\Pq\,(\bbbar)}\xspace}
\newcommand*{\tbarHq}{\Paqt{}\PH{}Pq}
\newcommand*{\tHbb}{\ensuremath{\tHq (\PH \to \bbbar)}\xspace}
\newcommand*{\tHtautau}{\ensuremath{\tHq (\PH \to \Pgt{}\Pgt)}\xspace}
\newcommand*{\dR}{\ensuremath{\Delta R}\xspace}
\newcommand*{\trexfitter}{TRExFitter\xspace}
\newcommand*{\thqloop}{\texttt{tHqLoop}\xspace}


\newcommand*{\MpT}{\ensuremath{\vec{p}_{\text{T}}^{\text{miss}}}\xspace}
\newcommand*{\mtw}{\ensuremath{m_{\text{T}}(\Pl,\MET)}}
\newcommand*{\mlb}{\ensuremath{m_{\Pl\Pqb}}}
\newcommand*{\mOSSF}{\ensuremath{m_{\text{OSSF}}}\xspace}

% Background processes
\newcommand*{\ttX}{\ensuremath{\Pqt{}\Paqt{}X}\xspace}
\newcommand*{\tX}{\ensuremath{\Pqt{}X}\xspace}

%\newcommand*{\ttX}{\Pqt{}\Paqt{}+X}
\newcommand*{\ttH}{\Pqt{}\Paqt{}\PH}
%\newcommand*{\ttH}{\ensuremath{\Pqt{}\Paqt{}\PH}\xspace}
\newcommand*{\ttZ}{\Pqt{}\Paqt{}\PZ}
\newcommand*{\ttV}{\ensuremath{\Pqt{}\Paqt{}V}\xspace}
\newcommand*{\ttW}{\Pqt{}\Paqt{}\PW}
\newcommand*{\ttWj}{\ensuremath{t\bar{t}W+j}\xspace}
\newcommand*{\tZq}{\Pqt{}\PZ{}\Pq}
\newcommand*{\tWZ}{\Pqt{}\PW{}\PZ}
\newcommand*{\tWH}{\Pqt{}\PW{}\PH}
\newcommand*{\tHW}{\Pqt{}\PW{}\PH}
\newcommand*{\tW}{\Pqt{}\PW}
\newcommand*{\Wt}{\Pqt{}\PW}
\newcommand*{\diboson}{diboson\xspace}
\newcommand*{\Diboson}{Diboson\xspace}
\newcommand*{\triboson}{triboson\xspace}
\newcommand*{\Triboson}{Triboson\xspace}
\newcommand*{\Vjets}{\ensuremath{V\text{+\,jets}}\xspace}

\newcommand*{\ttt}{\ensuremath{ttt}\xspace}
\newcommand*{\tttt}{\ensuremath{t\bar{t}t\bar{t}}\xspace}
\newcommand*{\ggH}{\ensuremath{ggH}\xspace}
\newcommand*{\qqH}{\ensuremath{qqH}\xspace}
\newcommand*{\WH}{\ensuremath{WH}\xspace}
\newcommand*{\ZH}{\ensuremath{ZH}\xspace}

% Fake leptons
\newcommand*{\elHF}{\ensuremath{e_{\text{HF}}}\xspace}
\newcommand*{\muHF}{\ensuremath{\mu_{\text{HF}}}\xspace}
\newcommand*{\elCo}{\ensuremath{e_{\text{conv}}}\xspace} 
\newcommand*{\kelHF}{\ensuremath{\mu(e_{\text{HF}})}\xspace}
\newcommand*{\kmuHF}{\ensuremath{\mu(\mu_{\text{HF}})}\xspace}
\newcommand*{\kelCo}{\ensuremath{\mu(e_{\text{conv}})}\xspace} 

% Signal regions
\newcommand*{\dileptau}{\ensuremath{2\Pl+1\tauhad}\xspace}
\newcommand*{\dilepOStau}{\ensuremath{2\Pl\,\text{OS}+1\tauhad}\xspace}
\newcommand*{\dilepSStau}{\ensuremath{2\Pl\,\text{SS}+1\tauhad}\xspace}

\newcommand*{\onelep}{\ensuremath{1\Pl}\xspace}
\newcommand*{\dilep}{\ensuremath{2\Pl}\xspace}
\newcommand*{\dilepOS}{\ensuremath{2\Pl\,\text{OS}}\xspace}
\newcommand*{\dilepSS}{\ensuremath{2\Pl\,\text{SS}}\xspace}
%\newcommand*{\SS}{\ensuremath{\text{SS}}\xspace}
%\newcommand*{\OS}{\ensuremath{\text{OS}}\xspace}

\newcommand*{\trilep}{\ensuremath{3\Pl}\xspace}
\newcommand*{\lepditau}{\ensuremath{1\Pl+2\tauhad}\xspace}




\newcommand*{\lumi}{\ensuremath{\mathcal{L}}\xspace}
\newcommand*{\lumiunits}{$\,$ cm$^{2} \,$s$^{-1}$\xspace}


%Other
\newcommand*{\CM}{\ensuremath{\sqrt{s}}\xspace}
 
\newcommand*{\Wtb}{\ensuremath{tWb}\xspace}
\newcommand*{\tWb}{\Wtb}
%\newcommand*{\tWb}{\ensuremath{tWb}\xspace}
\newcommand*{\pfour}{\ensuremath{\boldsymbol{\textrm{p}}}\xspace} 

\newcommand*{\tchan}{\ensuremath{t}-channel}
\newcommand*{\mtop}{\ensuremath{m_{\Ptop}}\xspace}
%\newcommand*{\mH}{\ensuremath{m_H}\xspace}
%\newcommand*{\HT}{\ensuremath{H_{\text{T}}}\xspace}

%\newcommand*{\PWplus}{\ensuremath{\PW^{+}}\xspace}
%\newcommand*{\PWminus}{\ensuremath{\PW^{-}}\xspace}
%\newcommand*{\Pgamma}{\ensuremath{\gamma}\xspace}

 \newcommand*{\greekphys}{\ensuremath{\varphi\upsilon\sigma\iota\kappa \eta}\xspace}
 \newcommand*{\greekatom}{\ensuremath{\alpha \tau o \mu o \nu}\xspace}
 \newcommand{\emu}{\ensuremath{\Pe/\Pmu}\xspace}

\newcommand*{\momentum}{\ensuremath{\overrightarrow{p}}\xspace} 
%\newcommand*{\CP}{\ensuremath{\mathcal{CP}}\xspace}
\newcommand*{\CP}{CP\xspace}

% Decays (Please, check latex/atlasprocess.sty and latex/atlasparticle.sty for more definitions!)
\newcommand{\bb}{\ensuremath{\Pqb\Paqp}\xspace}
\newcommand{\WW}{\ensuremath{\PW\PW^{*}}\xspace}
\newcommand{\ZZ}{\ensuremath{\PZ\PZ^{*}}\xspace}
\newcommand{\Higgsdecays}{\ensuremath{\PH \rightarrow b\bar{b}, \WW, \ZZ, \tau\tau}\xspace}
\newcommand{\Higgsdecayslep}{\ensuremath{\PH \rightarrow \WW, \ZZ, \tau\tau}\xspace}
\newcommand{\HWW}{\ensuremath{\PH \rightarrow \WW}\xspace}
\newcommand{\HZZ}{\ensuremath{\PH \rightarrow \ZZ}\xspace}

% reconstruction definitions
\newcommand{\pnutop}{\ensuremath{\vec{p^{\Pnu, \text{top}}}}\xspace}
\newcommand{\pnutopx}{\ensuremath{p^{\Pnu, \text{top}}_x}\xspace}
\newcommand{\pnutopy}{\ensuremath{p^{\Pnu, \text{top}}_y}\xspace}
\newcommand{\pnutopz}{\ensuremath{p_{z}(\Pnu_{\text{top}})}\xspace}
\newcommand{\pnutopT}{\ensuremath{\pT(\Pnu_{\text{top}})}\xspace}
\newcommand{\phinutop}{\ensuremath{\phi(\Pnu_{\text{top}})}\xspace}
\newcommand{\pltop}{\ensuremath{\vec{p^{\Plepton, \text{top}}}}\xspace}
\newcommand{\pltopx}{\ensuremath{p^{\Plepton, \text{top}}_x}\xspace}
\newcommand{\pltopy}{\ensuremath{p^{\Plepton, \text{top}}_y}\xspace}
\newcommand{\pltopz}{\ensuremath{p^{\Plepton, \text{top}}_z}\xspace}
\newcommand{\pltopT}{\ensuremath{\pT(\Plepton_{\text{top}})}\xspace}
\newcommand{\philtop}{\ensuremath{\phi(\ell_{\text{top}})}\xspace}
\newcommand{\phibtop}{\ensuremath{\phi(b_{\text{top}})}\xspace}
\newcommand{\tauvis}{\ensuremath{\Ptau_{\text{vis}}}\xspace}

\newcommand{\leptop}{\ensuremath{\Plepton^{\text{top}}}\xspace}
\newcommand{\lepH}{\ensuremath{{\Plepton^{\PH}}}\xspace}
\newcommand{\Hvismass}{\ensuremath{m_{\PH}^{\text{vis}}}\xspace}
\newcommand{\toprecomass}{\ensuremath{\mtop^{\text{reco}}}\xspace}
% \newcommand{\toprecomass}{\ensuremath{\Pqt_{\text{reco}}^{\text{m}}\xspace}}
% \newcommand{\Hvismass}{\ensuremath{\PH_{\text{vis}}^{\text{m}}\xspace}}
\newcommand{\MMC}{\texttt{MissingMassCalculator}\xspace}

% luminosity (2015)
\newcommand{\lumiFifteenRelUnc}{1.13} % in [%]
\newcommand{\lumitagFifteen}{{\small\texttt{OfLumi-13TeV-008}}}
%\newcommand{\lumiFifteenInPbNoUnits}{3219.56}
%\newcommand{\lumiFifteenInFbNoUnits}{3.2}
\newcommand{\lumiFifteenInPbNoUnits}{3244.54} % final luminosity recommendation for Run 2 analyses (https://twiki.cern.ch/twiki/bin/viewauth/Atlas/LuminosityForPhysics#2015_2018_13_TeV_proton_proton_f)
\newcommand{\lumiFifteenInFbNoUnits}{3.2}
%\newcommand{\dataperiodsFifteen}{D--J}
\newcommand{\dataperiodsFifteen}{D--H,J}
\newcommand{\firstdatarunFifteen}{276262}
\newcommand{\lastdatarunFifteen}{284484}
\newcommand{\datarunsFifteen}{\firstdatarunFifteen--\lastdatarunFifteen}
\newcommand{\dataeventsFifteen}{220.58M}

% luminosity (2016)
\newcommand{\lumiSixteenRelUnc}{0.89} % in [%]
\newcommand{\lumitagSixteen}{{\small\texttt{OfLumi-13TeV-009}}}
%\newcommand{\lumiSixteenInPbNoUnits}{32988.1}
%\newcommand{\lumiSixteenInFbNoUnits}{33.0}
% final luminosity recommendation for Run 2 analyses (https://twiki.cern.ch/twiki/bin/viewauth/Atlas/LuminosityForPhysics#2015_2018_13_TeV_proton_proton_f)
\newcommand{\lumiSixteenInPbNoUnits}{33402.2}
\newcommand{\lumiSixteenInFbNoUnits}{33.4}
%\newcommand{\dataperiodsSixteen}{A--L}
\newcommand{\dataperiodsSixteen}{A--G,I,K,L}
\newcommand{\firstdatarunSixteen}{297730}
\newcommand{\lastdatarunSixteen}{311481}
\newcommand{\datarunsSixteen}{\firstdatarunSixteen--\lastdatarunSixteen}
\newcommand{\dataeventsSixteen}{1057.84M}

% luminosity (2017)
\newcommand{\lumiSeventeenRelUnc}{1.13} % in [%]
\newcommand{\lumitagSeventeen}{{\small\texttt{OfLumi-13TeV-010}}}
%\newcommand{\lumiSeventeenInPbNoUnits}{44307.4}
%\newcommand{\lumiSeventeenInFbNoUnits}{44.3}
% final luminosity recommendation for Run 2 analyses (https://twiki.cern.ch/twiki/bin/viewauth/Atlas/LuminosityForPhysics#2015_2018_13_TeV_proton_proton_f)
\newcommand{\lumiSeventeenInPbNoUnits}{44630.6}
\newcommand{\lumiSeventeenInFbNoUnits}{44.6} 
%\newcommand{\dataperiodsSeventeen}{B--K}
\newcommand{\dataperiodsSeventeen}{B--F,H,I,K}
\newcommand{\firstdatarunSeventeen}{325713}
\newcommand{\lastdatarunSeventeen}{340453}
\newcommand{\datarunsSeventeen}{\firstdatarunSeventeen--\lastdatarunSeventeen}
%%%\newcommand{\datarunsSeventeen}{324320--341649}
\newcommand{\dataeventsSeventeen}{1340.80M}

% luminosity (2018)
\newcommand{\lumiEightteenRelUnc}{1.10} % in [%]
\newcommand{\lumitagEightteen}{{\small\texttt{OfLumi-13TeV-010}}}
%\newcommand{\lumiEightteenInPbNoUnits}{58450.1}
%\newcommand{\lumiEightteenInFbNoUnits}{58.5}
% final luminosity recommendation for Run 2 analyses (https://twiki.cern.ch/twiki/bin/viewauth/Atlas/LuminosityForPhysics#2015_2018_13_TeV_proton_proton_f)
\newcommand{\lumiEightteenInPbNoUnits}{58791.6}
\newcommand{\lumiEightteenInFbNoUnits}{58.8}
\newcommand{\lumiEightteenInPb}{\SI{\lumiEightteenInPbNoUnits}{\per\pb}}
\newcommand{\lumiEightteenInFb}{\SI{\lumiEightteenInFbNoUnits}{\per\fb}}
%\newcommand{\dataperiodsEightteen}{B--Q}
\newcommand{\dataperiodsEightteen}{B--D,F,I,K,L,M,O,Q}
\newcommand{\firstdataruEightteen}{348885}
\newcommand{\lastdatarunEightteen}{364292}
\newcommand{\datarunsEightteen}{\firstdataruEightteen--\lastdatarunEightteen}
%%%\newcommand{\datarunsEightteen}{348197--364292}
\newcommand{\dataeventsEightteen}{1716.77M}

% luminosity (2015+2016+2017)
%\newcommand{\lumiInPb}{80515.06~\invpb}
% \newcommand{\partlumi}{\SI{80.52}{\per\fb}}
%\newcommand{\datafirstyear}{2015}
%\newcommand{\datalastyear}{2017}

% luminosity (2015+2016+2017+2018)
%
% https://twiki.cern.ch/twiki/bin/viewauth/Atlas/LuminosityForPhysics#2015_2018_13_TeV_proton_proton_f
% final luminosity recommendation for Run 2 analyses (central value + uncertainty)
\newcommand{\lumiRelUnc}{0.83} % in [%]
\newcommand{\lumiInPbNoUnits}{140068.94} % in pb-1
\newcommand{\lumiInFbNoUnits}{140} % in fb-1
\newcommand{\lumiWithUnc}{\ensuremath{140.1 \pm 1.2}\,\si{\per\fb}} % in fb-1
%
% old recommendation
%\newcommand{\lumiRelUnc}{1.7} % in [%]
%\newcommand{\lumiInPbNoUnits}{138965.16} % in pb-1
%\newcommand{\lumiInFbNoUnits}{139} % in fb-1
%\newcommand{\lumiWithUnc}{\ensuremath{\lumiInFbNoUnits \pm 2.4}\,\si{\per\fb}} % in fb-1
\newcommand{\dataeventsAll}{4335.99M}
%
\newcommand{\lumiInPb}{\SI{\lumiInPbNoUnits}{\per\pb}}
%\newcommand{\lumi}{\SI{\lumiInFbNoUnits}{\per\fb}}
\newcommand{\datafirstyear}{2015}
\newcommand{\datalastyear}{2018}

% % tunes and PDF sets
\def\cteq{CTEQ6L1\xspace}
\def\ctten{CT10\xspace}
\def\cttennlo{CT10\,NLO\xspace}
\def\cttennnlo{CT10\,NNLO\xspace}
\def\ctfourteennlo{CT14\,NLO\xspace}
\def\ctfourteennnlo{CT14\,NNLO\xspace}
\def\nnpdfnnlo{NNPDF3.0\,NNLO\xspace}
\def\nnpdfnlofourflav{NNPDF3.0\,NLO\,nf4\xspace}
\def\nnpdfnlo{NNPDF3.0\,NLO\xspace}
\def\nnpdftwonlo{NNPDF2.3\,NLO\xspace}
\def\nnpdftwo{NNPDF2.3\,LO\xspace}
\def\nnpdftwofiveflav{NNPDF2.3\,5f\,FFN\xspace}
\def\mstw{MSTW2008\,NLO\xspace}
\def\a14{A14\xspace}
\def\auet{AUET2\xspace}
\def\aznlo{AZNLO\xspace}
\def\mmhtnnlo{MMHT2014\,NNLO\xspace}
\def\mmhtnlo{MMHT2014\,NLO\xspace}
\def\mmhtlo{MMHT2014\,LO\xspace}
\def\mstwnlo{MSTW2008\,68\%\,CL\,NLO \xspace}
\def\mstwnnloninety{MSTW2008\,90\%\,CL\,NNLO \xspace}
\def\ueee{UE-EE-5\xspace}




 
\endinput


%\begin{document}
%ENDPREAMBLE
\end{comment}


\chapter{Theoretical Framework}
\label{chap:Introduction}
%{\LARGE \textbf{Theoretical framework}\\}

%\tableofcontents

%\vspace*{0.1 cm} 
%\hspace*{200pt} \\
%\hspace*{175pt} \textit{L’essentiel est invisible pour les yeux.} \\
%\hspace*{175pt} ---\textsc{Antoine de Saint-Exupéry,} \\% \textit{} \\
%\hspace*{200 pt}     \textsc{Le Petit Prince (1943)} \\% \textit{} \\
%\vspace*{2cm} 

\vspace*{0.1 cm} 
\hspace*{200pt} \\
\hspace*{120pt} \textit{Tot el que sentireu en aquest programa ha passat.} \\
\hspace*{120pt} \textit{[...] En algunes descripcions aquest programa} \\
\hspace*{120pt} \textit{podria ferir sensibilitats. [...] Començem!} \\
\hspace*{205pt} ---\textsc{Carles Porta,} \\% \textit{} \\
\hspace*{240 pt}     \textsc{Crims (2020)} \\% \textit{} \\
\vspace*{2cm} 


%%%%%%%%%%%%%%%%%%%%%%%
%  The Standard Model of particle physics   %
%%%%%%%%%%%%%%%%%%%%%%%
\section{The Standard Model of particle physics}
\label{sec:chap1:TheSM}
Since the very first moment of our history, the humankind has pursued the knowledge of nature,
has tried to understand and describe how the universe works at a fundamental level. 
In fact, the word physics comes from the Greek ``\greekphys'' which means 
``nature'' \cite{etymology_web}\cite{Greek_web}.
Most of the enquires regarding this, can be boiled down to two basic questions:
What are the ultimate building blocks of reality? and which are the rules that govern them?

In the 7$^{th}$ century BCE, the pre-Socratic philosopher Thales of Miletus already proclaimed that every 
event had a natural cause \cite{Singer_C}. Later, to understand how the basic components of the matter were
formed, the ancient Indian philosophers such as Kanada and Dignaga on the 6$^{th}$  century BCE
and Greeks Democritus and Leucippus on the 5$^{th}$  century BCE, developed the atomism,
which comes from ``\greekatom'' meaning uncuttable or indivisible
\cite{taylor2010atomists}\cite{leaman2002key}.




From then to our days, the search for the minute fragments that comprise the matter and its interactions has led us to the
Standard Model (SM) of particle physics, one of the most successful scientific theories cultivated so far. This understanding
of the universe can explain phenomena from behaviour of atoms to how stars burn. 


%\begin{align*}
%\mathcal{L} &= - \frac{1}{4} F_{\mu \nu} F^{\mu \nu} \\
%   &\phantom{{}=}+ i \bar{\psi} \cancel{D} \psi + h.c. \\
%    &\phantom{{}=}+ \bar{\psi}_i y_{ij} \psi_j \phi + h.c. \\
%    &\phantom{{}=}+ |D_\mu \phi|^2 - V(\phi)
%\end{align*}

\subsection{Introduction to the SM and its elementary particles}
\label{sec:chap1:SM_and_EParticles}
Based on Quantum Field Theory (QFT), the SM of particle physics provides the theoretical framework that constitutes what is 
currently accepted as the best description of particles physics. It aims to explain both all particles of matter and
 their interactions. The completion of the SM was a collaborative effort of several scientists during the second half of the
$20^{th}$, being the current formulation finalised in the decade of 1970. A representation of the fundamental particles, i.e. particles that
are not made of anything else, that compose the SM is presented in Figure \ref{fig:Chap1:SM}.
Most of these particles are unstable and decay to lighter particles within fractions of a second. 
% Maybe some comment about the fact that decaying to something doesn't mean to be made of that thing. 
 As the scheme in Figure \ref{fig:Chap1:SM} indicates, the 
12 fermions have their corresponding 12 anti-fermions and the quarks and gluons carry colour charge. 
\begin{figure}
    \centering
    \includegraphics[width = 1\textwidth]{Chapter1/SMofElementaryParticles_fancy_modified}
    \caption{Fundamental particles of the Standard Model (image modified from \cite{Purcell:1473657}). }
    % In each box the upper-left corner express the mass in eV, the upper-right the electric charge (green) and possible color charges (red), and the lower-right the spin.}
    \label{fig:Chap1:SM}
\end{figure}




The SM is a gauge theory based on the symmetry group $SU(3)_{C} \bigotimes SU(2)_{L} \bigotimes U(1)_{Y}$, which
describes all fundamental interactions except from the gravitational force\footnote{The gravitational interaction is described by Einstein's General Relativity (GR) \cite{Einstein:1916vd}.}.
This theory provides an explanation to strong, weak and electromagnetic interactions via the exchange of the corresponding
vector\footnote{``Vector bosons'' refer to all particles that have spin 1 in contrast to the ``scalar boson's' which have spin 0.} bosons (spin-1 gauge fields).
The mediation for the electromagnetic interaction (explicated in \ref{sec:chap1:QED}) 
is done by one massless photon (\Pgamma), this force is invariant under the $U(1)$ symmetry group.
While for the weak interactions, guided by $SU(2)$, three massive bosons,
\PWplus, \PWminus and \PZ, act as mediators ($m_{\PWpm}= 80.385\pm0.015\;$GeV \cite{ATLAS:2017rzl} and $m_{\PZ}= 91.1876\pm0.02\;$GeV \cite{ALEPH:2005ab}). 
Although the electromagnetic and weak interactions seem completely different at low energies, they are two aspects of the same force and
can be described simultaneously by the $SU(2)_{L} \bigotimes U(1)_{Y}$ 
symmetry group, which represents the so called Electro-Weak (EW) sector (detailed in Section \ref{sec:chap1:EW}).
The strong force, with its eight massless gluons (\Pgluon), is described by the $SU(3)_{C}$ colour group (see Section \ref{sec:chap1:QCD}). 
All these interactions differ in their magnitude, range and the physical phenomena that describe. These features
are summarised in Table \ref{tab:Chap1:FundamentalInteractions}, where not only the interactions of the SM are included but the
gravitation is as well.  

Apart from the vector bosons, there is one massive scalar boson, the Higgs boson ($m_{\PHiggs} = 125.25 \pm 0.17\;$GeV). 
Through the interaction with this particle, all massive particles of Figure \ref{fig:Chap1:SM} gain their mass via the EW spontaneous symmetry breaking.This mechanism was first described by Englert, Brout \cite{Englert:1964et} and Higgs \cite{Higgs:1964pj}, and its 
summarised in Section \ref{sec:chap1:ParticleMasses:HiggsMechanism}. 
% Gauge theory: A type of field theory in which the Lagrangian (and hence the dynamics of the system itself) does not 
% change (is invariant) under local transformations according to certain smooth families of operations 

\begin{table}[]
\centering
\begin{tabular}{l c c c c c}
\toprule
%Interaction & \begin{tabular}[c]{@{}l@{}}Mediator\\ boson\end{tabular} & Theory & \begin{tabular}[c]{@{}l@{}}Relative \\ stregth\end{tabular} & Range (m) \\ \midrule
Interaction     		& Theory  			& Mediator             	& Relative strength 	& Range (m) 	\\ \midrule
Strong          		& QCD	 		& $\Pgluon$              	& 1                		& $10^{-15}$    \\
Electromagnetic 	& QED/EW 		& $\Pgamma$          	&  1/137                	& $\infty$         	\\
Weak            		& EW   			& $\PWpm$, $\PZ$	&  $10^{-6}$              & $10^{18}$      \\
Gravitational     		& GR       			& -		 		&  $6\times10^{-39}$	& $\infty$  	\\ \bottomrule          
\end{tabular}
\caption{Typical strength of the fundamental interactions with respect to the strong interaction. Here the strength is understood as the coupling constant or gauge coupling parameter. 
%The description of the electromagnetism and weak interactions is unified by the EW interaction.
In GR the gravitational interaction is not a force but the effect of the four-dimensional spacetime curvature and, hence, it has no mediator in this formalism.}
\label{tab:Chap1:FundamentalInteractions}
\end{table}


%There are two important and distinct SU(3) symmetries that are relevant for the strong interactions: 
%SU(3) color symmetry of the quark and gluon dynamics and SU(3) flavour symmetry of light quarks. 
%Each of these symmetries refers to an underlying threefold symmetry in strong interaction physics.

%A representation of the fundamental particles that compose the SM is presented in Figure \ref{fig:Chap1:SM}.
%It is necessary to take into account that all quarks and leptons have their analogous antiparticles.% and 
% that the quarks and gluons carry the color charge\footnote{Antiquarks carry the anticolor charge}. Therefore, a more complete illustration
%of the complete set of fundamental is displayed in Figure \ref{fig:Chap1:SM_color}, where appear the colour variations and the antiparticles
%for all the particles in Figure \ref{fig:Chap1:SM}. The origin of the colour charge is discussed in Section \ref{sec:chap1:QCD}.

%\begin{figure}
%    \centeringring
%   \includegraphics[width = 0.75\textwidth]{Chapter1/SMofElementaryParticles}
%  \caption{Fundamental particles of the Standard Model}
%    \label{fig:Chap1:SM_basic}
%\end{figure}



Before describing the fundamental interactions of the SM in the QFT formalism, let's introduce the main two types of particles according 
to their spin, i.e. intrinsic angular momentum: fermions and bosons.

% FERMIONS
\paragraph{Fermions}\mbox{}\\
The fermions are the particles that follow the Fermi-Dirac statistics, i.e. obey the Pauli exclusion principle \cite{10230794692Dirac}, resulting 
in a distribution of particles over energy levels in which two elements with the same quantum numbers cannot occupy the same states.
The fermions include all particles with half-integer spin: quarks, leptons and baryons.
A baryon is a non-fundamental particle composed of an odd number of valence quarks
%\footnote{The hadrons (baryons are mesons) are understood as a sea of partons being the valence quarks those which are more probable to be found in the hadron according to the parton distribution functions.} (consequently having half-integer spin) and nearly all matter that
may be encountered or experienced in everyday life is baryonic matter. Some examples of baryons are\footnote{Between round brackets, the valence quarks are shown.} the proton (\Pup\Pup\Pdown), the 
neutron (\Pdown\Pdown\Pup), \PLambda (\Pup\Pdown\Pstrange), \PLambdac (\Pup\Pdown\Pcharm) and \PSigmaplus (\Pup\Pup\Pstrange).
%\PDelta (\Pup \Pup \Pdown) The difference between de \PDelta and the proton is the arrangement of the spin of the quarks, 
Apart from the 3-quark baryons, an exotic pentaquark state has been observed at LHCb experiment of the LHC \cite{LHCb:2019kea}. 


The fundamental fermionic matter (Table \ref{tab:Chap1:FundamentalFermions}) is organised in the three families of leptons and quarks:

\begin{center}
$\begin{bmatrix}
\Pnue & \Pup \\
\Pelectron & \Pdown 
\end{bmatrix}$
,
$\begin{bmatrix}
\Pnum & \Pcharm \\
\Pmuon & \Pstrange 
\end{bmatrix}$
,
$\begin{bmatrix}
\Pnut & \Ptop \\
\Ptauon & \Pbottom 
\end{bmatrix}$
\end{center}.

%where, according to QCD, each quark appears in three different colours:
These three generations, which are defined as the columns in Figure \ref{fig:Chap1:SM}, exhibit the same kind of 
gauge interactions and they only differ in their mass \cite{Pich:2007vu}.
According to the EW symmetry, each family can be classified as:
\begin{center}
$\begin{bmatrix}
\Pnulepton & \ensuremath{\Pquark_{u}} \\
\Pleptonminus & \ensuremath{\Pquark_{d}} 
\end{bmatrix}$
$\equiv$
$\begin{pmatrix}
\Pnulepton \\
\Pleptonminus
\end{pmatrix}_{L}$ ,
$\begin{pmatrix}
\ensuremath{\Pquark_{u}} \\
\ensuremath{\Pquark_{d}} 
\end{pmatrix}_{L}$,
$\Pleptonminus_R$, $\ensuremath{\Pquark_{uR}}$, $\ensuremath{\Pquark_{dR}}$
\end{center}
(plus the corresponding antiparticles) where the subindices \textit{L} and \textit{R} stand from left and right handed particles respectively. 
This structure responds to the fact that left-handed particles convert different than right-handed ones under $SU(2)$ transformations.
The left-handed fields are $SU(2)_L$ doublets and the right-handed ones $SU(2)_L$ singlets. This difference is explained with more
detail in Section \ref{sec:chap1:EW}. 

%As discussed in the following sections, the weak interaction only affects left-handed particles (and right-handed antiparticles) therefore,
%since the most basic representation of $SU(2)$ is a doublet, the $\begin{pmatrix}
%\Pnulepton \\
%\Pleptonminus
%\end{pmatrix}_{L}$ element appears. The $\Pleptonminus_{R}$ exists because QED affects the charged leptons but not the neutrinos and 
%, since the representation of $U(1)$ is a singlet and this force does not differentiate between eft and right-handed particle, the $\Pleptonminus_{R}$ 
 
%makes no difference between left and right-handed particles and,
%since the charged leptons but not the nautrinos are affected by QED, only the $\Pleptonminus_{R}$ but not the $\Pnulepton_R$ exits.
%The most basic representation of SU(2) is a doublet.
%\pablo{(Why there are right and left charged leptons but only left neutrinos? First of all, because  thy are not observed in Nature 
%and, secondly because the neutrinos are neutral and, hence, its quantum numbers under SM transformations are 1,1,1, i.e. they are singlets)}

The fundamental representation of $SU(3)$ is a triplet, this is why each quark can appear in three different colours, whereas each antiquark can exhibit one of the corresponding ``anticolours''. 

%->. L = Left-polarisation <- Left-handed particles transform different than right handed under SU2 transformations.
%La representación más básica de su2 es un doblete
%y las de su3 son triples (red, green blue)
%Igual que hay lepton right, en el SM no hay neutrino right : In the SM, neutrinos are left handed

%-> Should I introduce the conservation of the leptonic and baryonic numbers?

% Please add the following required packages to your document preamble:
% \usepackage{multirow}

The SM fermions properties are summarised in Table \ref{tab:Chap1:FundamentalFermions}. As can be seen in its last rows, the neutrino flavour states do no correspond to 
the mass states ($\nu_{1}$, $\nu_{2}$, $\nu_{3}$). What happens is that each flavour state is a quantum mechanical combination of neutrinos of different masses and viceversa.
More details about the neutrino masses can be found in a dedicated text in Section \ref{sec:chap1:SM_problems}

\begin{table}[]
\centering
\begin{tabular}{lccc}
\toprule
Family                   		& Name              			& Mass 								& Q    \\ \midrule
\multirow{6}{*}{Quarks} 	& Up 	 ($\Pup$)                	& $2.16^{+0.49}_{-0.26}\,$MeV     			& 2/3  \\
                         			& Down     ($\Pdown$)            	& $4.67^{+0.48}_{-0.17}\,$MeV	     			& -1/3  \\
                         			& Charm 	 ($\Pcharm$)            	& $1.27 \pm 0.02\,$GeV					& 2/3  \\
                         			& Strange  ($\Pstrange$)          	& $93^{+11}_{-5}\,$MeV     				& -1/3 \\
                         			& Top  	 ($\Ptop$)              	& $172.76 \pm 0.30\,$GeV     				& 2/3 \\
                         			& Bottom   ($\Pbottom$)          	& $4.18^{+0.03}_{-0.02}\,$GeV	     			& -1/3 \\ \midrule
\multirow{6}{*}{Leptons} 	& Electron  ($\Pelectron$)        	& $0.5109989461\pm0.0000000031\,$MeV    	& -1   \\
                         			& Muon      ($\Pmu$)		        	& $105.6583745\pm0.0000024\,$MeV     		& -1   \\
                         			& Tau          ($\Ptau$)     		& $776.86\pm0.12\,$MeV					& -1   \\
                         			& Electron neutrino ($\Pnue$) 	& \multicolumn{1}{c}{\multirow{3}{*}{\begin{tabular}[c]{@{}c@{}}
											$\Pnue$, $\Pnum$, $\Pnue$ \\
													$\neq$ \\
											$\nu_{1}$, $\nu_{2}$, $\nu_{3}$\end{tabular}}}  & 0 \\
                         			& Muon neutrino     ($\Pnum$)	&      & 0    \\
                         			& Tau neutrino        ($\Pnut$)  	&      & 0    \\ \bottomrule
\end{tabular}
\caption{Properties of the quarks and leptons. The electric charge, represented by Q, is presented in units 
of elementary charge ($1.602 \times10^{-19}\,$C) . The $\nu_{1}$, $\nu_{2}$, $\nu_{3}$ are the neutrino mass eigenstates.}
\label{tab:Chap1:FundamentalFermions}
\end{table}

The fundamental fermions are usually understood as the fundamental building blocks of matter. However, while the building blocks are important, there is a point that 
also has to be taken into account, the force. Without force these fermions would not interact which each other. The particles that mediate these interactions are the
gauge bosons. 




\paragraph{Bosons}\mbox{}\\
Bosons differ from fermions by obeying the Bose-Einstein statistics, thus, bosons are not limited to single occupancy for a determined state. In other words,
the Pauli exclusion principle is not applied. All particles with integer spin are bosons; from the particles shown on the right columns of Figure \ref{fig:Chap1:SM}
to the mesons. Mesons, along with baryons, are part of the hadron family, i.e. particles composed of quarks (see Section \ref{sec:chap1:QCD}). 
The particularity of mesons is that they are formed from an equal number of quarks and antiquarks (usually one of each) bound together by strong 
interactions. Some examples of mesons are \Ppiplus (\Pup \APdown), \Ppizero ($\frac{\Pup \APup - \Pdown \APdown}{\sqrt{2}}$), \PKplus (\Pup \APstrange) and \PJpsi(\Pcharm \APcharm).
% \Petac (\Pcharm \APcharm) and \PJpsi(\Pcharm \APcharm) have the same quark distribution composition but \Petac is a pseudoscalar and a vector.

The elementary vector bosons are the force carriers and presented in Table \ref{tab:Chap1:FundamentalInteractions} while the Higgs boson is a fundamental particle as well. 
%\begin{figure}
%    \centering
%    \includegraphics[width = 0.95\textwidth]{Chapter1/SM_Color}
%    \caption{Extended table of the particles composing the SM: Antiparticles and colour charge configurations are included.}
%    \label{fig:Chap1:SM_color}
%\end{figure}


%%%%%%%%%%%%%%%%%%%%%
%              Gauge Invariance                   %
%%%%%%%%%%%%%%%%%%%%%
\paragraph{Gauge Invariance}\mbox{}\\
Constituting one of the most successful theories of Physics, the SM is able to provide an elegant mathematical framework to
describe the experimental physics results with great precision.
Another key element to understand the SM is the concept of gauge invariance.
As it is illustrated during the rest of the Section \ref{sec:chap1:TheSM}, by demanding that 
the Lagrange density (also denoted as Lagrangian) invariant
under local gauge transformations, the existence of the SM force-carrier 
bosons (\Pgamma, \PWplus, \PWminus, \PZ and \Pg). %is predicted through
%the gauge invariant field strength tensors:
%\begin{align*}
%  A  : photon
%  B  : ew
%  W : ew
%  G	 : gluon
%\end{align*}


%%%%%%%%%%%%%%%
%              QED                    %
%%%%%%%%%%%%%%%
% Peskin: http://home.ustc.edu.cn/~gengb/200923/Peskin,%20An%20Introduction%20to%20Quantum%20Field%20Theory.pdf
% notes: http://www.freebookcentre.net/physics-books-download/Relativistic-Quantum-Field-Theory-Lecture-Notes-I.html
\subsection{Quantum electrodynamics}
\label{sec:chap1:QED}
The gauge invariance refers to the invariance of a theory under transformations which the theory is said to posses internal symmetry.
The transformations which are applied in all space-time locations simultaneously are known as ``global'' transformations while the ones that vary from 
one point to another are ``local''. Each local symmetry is the basis of a gauge theory and requires the introduction of its own gauge bosons as it is 
discussed in the following pages.

In QFT, particles are described as excitations of quantum fields that satisfy the corresponding mechanical field equations.
The Lagrangians in QFT are used analogous to those of classical mechanics, where the equation of motion can be derived from the Lagrangian density
function ($\mathcal{L}$) and the Euler-Lagrange equations for fields:
\begin{equation*}
\frac{\partial\mathcal{L}}{\partial \phi} - \partial_{\mu} \frac{\partial\mathcal{L}}{\partial (\partial_{\mu} \phi)} = 0 ,
\end{equation*}
where  $\partial_{\mu} = \frac{\partial}{\partial x^{\mu}}$ denotes the partial derivatives with respect to the four-vector $x^{\mu}$ and 
$\phi = \phi(\overrightarrow{x},t)$ is the quantum field of a fermion or boson. % A continuous quantity with a value at each point in space-time.
The Lagrangian is used to express the dynamics of the quantum field. In QFT, Noether's theorem \cite{Noether1918}
relates a symmetry in the $\mathcal{L}$ to a conserved current.

The Dirac equation, $(i \gamma^{\mu}\partial_{\mu} - m)\Psi(x)=0$, is one of the simplest relativistic field equations. Its Lagrangian
describes a free Dirac fermion:

\begin{equation}\label{eq:chap1:Dirac1}
\mathcal{L}_{0} = i \bar{\Psi}(x) \gamma^{\mu} \partial_{\mu} \Psi(x) - m \bar{\Psi}(x) \Psi(x) ,
\end{equation}
being $\Psi$ the wave function (spinor represented by four complex-valued components) of the particle, $\gamma^{\mu}$ 
are the Dirac or gamma matrices, $\{\gamma^{0},\gamma^{1},\gamma^{2}, \gamma^{3} \}$, $m$ the rest-mass of the fermion and $\bar{\Psi}= \Psi^{\dagger}\gamma^{0}$, the hermitic conjugate of the wave function.
% In the limit m->0 Dirac equation reduces to the Weyl equation
The gamma matrices build a set of orthogonal basis vectors for covariant vectors in a Minkowski space.
The first term of $\mathcal{L}_{0}$ is the kinetic term while the second is the mass term. 

This Lagrangian is invariant under $U(1)$ global transformations such as:
\begin{equation}\label{eq:chap1:DiracGlobalTransformation}
\Psi(x) \xrightarrow{\text{U(1)}} \Psi'(x) \equiv exp\{ i Q \theta \} \Psi(x),
\end{equation}
where $Q \theta$ is a real constant. The phase of $\Psi(x)$ is a pure convention-dependent quantity without a physical meaning since  the observables depend on $|\Psi(x)|^{2}$.


However, if $\theta$ was $x$ dependent, the transformation \ref{eq:chap1:DiracGlobalTransformation} would be:
\begin{equation}\label{eq:chap1:qedLocalTransf}
\Psi(x) \xrightarrow{\text{U(1)}} \Psi'(x) \equiv exp\{ i Q \theta (x) \} \Psi(x),
\end{equation}
which is not longer a global transformation but a local
transformation instead. The transformation in \ref{eq:chap1:qedLocalTransf} would not let the $\mathcal{L}_{0}$ in \ref{eq:chap1:Dirac1} 
invariant because the derivative in the kinetic term would go as:
\begin{equation}\label{eq:chap1:derivativeTransformation}
\partial_{\mu}\Psi(x) \xrightarrow{\text{U(1)}} exp\{ i Q \theta \} (\partial_{\mu} + iQ\partial_{\mu}\theta)\Psi(x).
\end{equation}

The gauge principle is the requirement that the $U(1)$ phase invariance should hold locally.
In order to do so, it is necessary to introduce an additional term to the Lagrangian so that when one applies $\Psi'(x) \equiv exp\{ i Q \theta (x) \} \Psi(x)$, 
the $\partial_{\mu}\theta$ term is canceled in \ref{eq:chap1:derivativeTransformation}.
To achieve this invariance, a term with the vector gauge field $A_{\mu}$ is inserted. This field transforms as
\begin{equation}\label{eq:chap1:AmuTransformation}
A_{\mu}(x) \xrightarrow{\text{U(1)}} A_{\mu}'(x) \equiv A_{\mu}(x)+\frac{1}{e}\partial_{\mu}\theta
\end{equation}
with a new $D_{\mu}$, which acts as follows:  %covariant derivative:
\begin{equation}\label{eq:chap1:NewQEDderivative}
D_{\mu} \Psi(x) \equiv [ \partial_{\mu} + ieQA_{\mu}(x)]\Psi(x)
\end{equation}
which transforms like the field:
\begin{equation*}\label{eq:chap1:QED_DerivativeTransformation}
D_{\mu} \Psi(x) \xrightarrow{\text{U(1)}} (D_{\mu} \Psi)'(x) \equiv  exp\{ i Q \theta \} D_{\mu}\Psi(x).
\end{equation*}

The Lagrangian density can be defined by replacing the partial derivatives in $\mathcal{L}_{0}$ (\ref{eq:chap1:Dirac1}) by the covariant derivative in \ref{eq:chap1:NewQEDderivative}:
\begin{equation}\label{eq:chap1:Dirac2}
\begin{split}
	\mathcal{L}_{QED} &\equiv i \bar{\Psi}(x) \gamma^{\mu} D{\mu} \Psi(x) - m \bar{\Psi}(x) \Psi(x) \\
				&=  i \bar{\Psi}(x) \gamma^{\mu} [ \partial_{\mu} + ieQA_{\mu}(x)] \Psi(x) - m \bar{\Psi}(x) \Psi(x) \\
				&=  i \bar{\Psi}(x) \gamma^{\mu} \partial_{\mu} \Psi(x) -\bar{\Psi}(x) \gamma^{\mu}eQA_{\mu}\Psi(x)   - m \bar{\Psi}(x) \Psi(x) \\
				&= \mathcal{L}_{0} - eQA_{\mu}\bar{\Psi}(x)\gamma^{\mu}\Psi(x).
\end{split}
\end{equation}

The resulting Lagrangian is invariant under $U(1)$ local transformation. When the conversions \ref{eq:chap1:qedLocalTransf} and 
\ref{eq:chap1:AmuTransformation} take place, the effects of the transformation are canceled out. 
Along with the original Lagrangian ($\mathcal{L}_{0}$), the $\mathcal{L}_{QED}$ has an additional term describing the interaction 
between the fermion $\Psi$ and the gauge field $A_{\mu}$ with a strength proportional to the charge $eQ$. 
This term, $eQA_{\mu}\bar{\Psi}\gamma^{\mu}\Psi$, that has been generated only by demanding the gauge invariance
under $U(1)$, is not other than the vertex of QED (Figure \ref{fig:Chap1:QED_Vertex}). 
\begin{figure}
    \centering
    \includegraphics[width = 0.35\textwidth]{Chapter1/QED_vertex}
    \caption{Three-point interaction vertex of QED.}
    \label{fig:Chap1:QED_Vertex}
\end{figure}

This new $A_{\mu}$ term is the electromagnetic field and its quanta is the photon.
A mass term containing $A^{\mu}A_{\mu}$ is forbidden because it would violate the $U(1)$ local invariance. 
In consequence, the mediator of the new $A_{\mu}$ field, the photon, is predicted to be a massless particle. 
To make $A_{\mu}$ a propagating field it is necessary to add the kinetic term of the field $A_{\mu}$:
\begin{equation}\label{eq:chap1:QED_Kin}
	\mathcal{L}_{kin} \equiv - \frac{1}{4}F_{\mu \nu}(x) F^{\mu \nu}(x),
\end{equation}
where $F^{\mu \nu} \equiv \partial_{\mu} A_{\nu} -\partial_{\nu} A_{\mu}$. 
The kinetic term $F_{\mu \nu} F_{\mu \nu} F^{\mu \nu}$ is already invariant under local $U(1)$ phase transformations.
From the QED Lagrangian in \ref{eq:chap1:Dirac2} and the kinetic term in \ref{eq:chap1:QED_Kin},
the Maxwell equations can be derived to describe electromagnetism, the infinite range\footnote{Since the photon is (predicted to be) massless, the electromagnetic interaction has an infinite range.} 
interaction that occurs between particles with electrical charge.
The $\mathcal{L}_{QED}$  with this kinetic term is written as:
\begin{equation}\label{eq:chap1:QED_Complete}
	\mathcal{L}_{QED} =  \bar{\Psi}(x) (i  \gamma^{\mu} \partial_{\mu} - m ) \Psi(x) - eQ \bar{\Psi}(x) \gamma^{\mu}A_{\mu} \Psi(x) - \frac{1}{4}F_{\mu \nu}(x) F^{\mu \nu}(x).
\end{equation}


%%%%%%%%%%%%%%%
%          ElectroWeak            %
%%%%%%%%%%%%%%%
\subsection{Electroweak interactions}
\label{sec:chap1:EW}

\subsubsection{Weak interactions and symmetries}
The weak interaction is mediated by the $\PWplus$, $\PWminus$ and $\PZ$ massive gauge bosons.
Due their large mass, the range of the interactions is within a scale of $\sim 10^{-18}$ m. %($m_{\PWpm} =  80.4$ GeV and $m_{\PZ} =  91.2$ GeV)
It is responsible for radioactive decays and flavour changing\footnote{The leptonic charges are conserved.} decays of fermions such as the decay of 
the muon ($\Pmuon \rightarrow \Pelectron \APneutrino_{e} \Pneutrino_{\mu}$).


Another particularity of this interaction is that it is the only interaction that violates several fundamental symmetries. 
There is a relation between symmetries and conservations laws which is known as Noether's theorem. Classical physics
examples of how the symmetries leads to conserved quantities are:
\begin{itemize}
	\item Invariance under change of time $\rightarrow$ Conservation of energy
	\item Invariance under translation in space $\rightarrow$ Conservation of momentum
	\item Invariance under rotation $\rightarrow$ Conservation of angular momentum
\end{itemize}

The three discrete symmetries that are fundamental for the SM formulation and are always hold for electromagnetic and strong interactions are:
\begin{itemize}
	\item \textbf{Charge conjugation ($\mathcal{C}$)}: Replace positive quantum charges by negative charges and vice versa. %This symmetry is behind
										%the conservation of lepton number, baryon number and strangeness.
										It does not affect mass, energy, momentum or spin. Essentially, it is a transformation that switches
										all particles with their corresponding antiparticles.
										\begin{align*}
											%\mathcal{C} \ket{ \Psi} = \ket{\bar{\Psi}}
											\mathcal{C} \Psi(\overrightarrow{r}, t) = \bar{\Psi}(\overrightarrow{r}, t)
										\end{align*}
										 
	\item \textbf{Parity ($\mathcal{P}$)}: Parity involves a transformation that changes the algebraic sign of the spatial coordinate system. 
								%It transforms a phenomenon by inverting its spatial coordinates through the origin. 
								It does not reverse time, mass, energy or other scalar quantities.
								\begin{align*}
									\mathcal{P}:\begin{pmatrix} x \\ y \\ z \end{pmatrix} \rightarrow \begin{pmatrix} -x \\ -y \\ -z \end{pmatrix} &&
									%\hat{\mathcal{P}} \Psi(r) = e^{i \frac{\theta}{2}} \Psi(-r)
									\mathcal{P} \Psi(\overrightarrow{r}, t) = \Psi(-\overrightarrow{r}, t)
								\end{align*}
	\item \textbf{Time reversal ($\mathcal{T}$)}: Consists in flipping the sign of the time
								\begin{align*}
									\mathcal{T}:t \rightarrow -t &&
									\mathcal{T} \Psi (\overrightarrow{r}, t) = \Psi (\overrightarrow{r}, -t)
									%\mathcal{T} \Psi(t, \overrightarrow{x})\mathcal{T}^{-1} = \Psi(-t, \overrightarrow{x})
 								\end{align*}
\end{itemize}

%\paragraph{\CP-symmetry}\mbox{}\\
The simultaneous combination of this three symmetries mentioned above results in the $\mathcal{CPT}$ symmetry, a profound symmetry of QFT which is
consistent through all experimental observations \cite{Moura:2022dev}. %If $\mathcal{CPT}$ symmetry is conserved, particles and their respective antiparticles 
%are predicted to have, for example, the same mass and lifetime. %The SM is structured on the pillars of CPT symmetry and Lorentz invariance.
Meanwhile, the $\mathcal{P}$-symmetry and the $\mathcal{C}$-symmetry can be combined to create the \CP-symmetry, the product of the two transformations.
The weak interaction violates $\mathcal{P}$ and $\mathcal{C}$ symmetries. 
It also violates the combined \CP-symmetry. Therefore, through the CPT theorem \cite{Bell:1955djs}, if the \CP is violated,
$\mathcal{T}$ is violated as well to preserve the $\mathcal{CPT}$ invariance \cite{Streater:1989vi}. %This has been verified experimentally

%\paragraph{Parity and Charge conjugation violation}\mbox{}\\
\paragraph{Parity violation}\mbox{}\\
Previously theorised by Lee and Yang \cite{Lee:1956qn}, the confirmation of the non-conservation of $\mathcal{P}$ in weak interactions 
arrived with the Wu experiment in 1957 \cite{Wu:1957my}. %A more detailed overview of \CP violation is developed in Section \ref{sec:chap1:CP_Violation}.
Studying the beta decay of the Cobalt-60, Wu and collaborators found that the neutrino 
and the antineutrino have the relative orientations of spin and linear momentum fixed.
The neutrino spin is always opposite to the linear momentum, this is called left-handed particles.
Meanwhile, for the antineutrinos, the momentum is always aligned in the same direction as the spin (right-handed particles).
This causes the weak interactions which emit neutrinos or antineutrinos to violate the conservation of parity.


Only left-handed particles and right-handed antiparticles are sensitive to the weak force. Dirac fermion fields, $\psi$, exhibit 
chiral symmetry and the right and left handed chiral states can be expressed as:
\begin{align}
	\psi_{L }(x) &= \frac{1}{2} (1-\gamma_{5})\psi (x) \equiv P_{L}  \psi (x) \\
	\psi_{R }(x) &= \frac{1}{2} (1+\gamma_{5}) \psi (x) \equiv P_{R}  \psi (x)
\end{align}
with
\begin{align*}
	\gamma^{5} &\equiv \gamma_{5} \equiv \gamma^{0}\gamma^{1}\gamma^{2}\gamma^{3} = \begin{pmatrix} 1 & 0 \\ 0 & 1 \end{pmatrix}
\end{align*}
where $P_{L}$ and $P_{L}$ are known as projection operators. The last equality is valid in the Dirac representation.
%This requieres a more complex structure to describe weak interactions.

%In the same year, the $\mathcal{C}$ violation was found too \cite{PhysRev.105.1415}. \pablo{Describe how the $\mathcal{C}$ violation was discovered}


%CP VIOLATION
\paragraph{\CP violation}\mbox{}\\
While $\mathcal{P}$ and $\mathcal{C}$ are violated in a maximal way by the weak interactions, 
the product of these two discrete transformations, \CP, is still a good symmetry (left-handed fermions 
$\leftrightarrow$ right-handed fermions).  Experiences such as the Wu experiment respect the \CP 
symmetry and, in fact, in the \CP is a symmetry of nearly all the observed phenomena. However, in 
1964 Cronin and Fitch discovered a slight (2\%) violation of the \CP symmetry in the decays of neutral 
kaons \cite{Christenson:1964fg}. The \CP violation plays a fundamental role to explain the dominance of
matter over antimatter in the present universe. More information about the matter-antimatter asymmetry
can be found in the dedicated text in Section \ref{sec:chap1:SM_problems}. 

Direct \CP violation is allowed in the SM if a complex phase is present in the CKM matrix (described below).
The ``direct'' \CP violation is a phenomenon where the same decay process has a different probability for a 
particle than for an antiparticle. An example of strong global \CP asymmetry observed corresponds to the decay
into two kaons and one pion. The probability of $\PBplus \rightarrow \Ppiplus \PKplus \PKminus$ is 20\% higher 
than for $\PBminus \rightarrow \Ppiminus \PKplus \PKminus$. 
%Source: https://moriond.in2p3.fr/2022/EW/slides/5/1/6_RCardinale-v1.pdf
%source: https://home.cern/news/news/physics/largest-matter-antimatter-asymmetry-observed
%So far, $\mathcal{CPT}$ is the only symmetry that stands unviolated and, given that \CP is not, the $\mathcal{T}$ must be violated as well. %Already commented

%Quizás discutir la relevancia que tiene CP en esta tesis

\paragraph{CKM matrix}\mbox{}\\
The eigenstates that interact through weak interactions, known as ``weak eigenstates'' 
(\Pdown', \Pstrange', \Pup'), are different from the physically observed mass eigenstates 
(\Pdown, \Pstrange , \Pup). This make possible the charged-flavour-changing-weak decays
trough the Cabibbo-Kobayashi-Maskawa (CKM) matrix.
The CKM matrix, $V_{CKM}$, describes the mixing between the three generations of quarks in the SM. 
The coupling of two quarks $i$ and $j$ to a \PW boson is proportional to the CKM matrix element $V_{ij}$.
\begin{equation}
	\begin{pmatrix} \Pdown' \\ \Pstrange ' \\ \Pup ' \end{pmatrix}  = \begin{pmatrix} 	V_{ud} & V_{us} & V_{ub} \\
																V_{cd} & V_{cs} & V_{cb} \\ 
																V_{td}  & V_{ts}  & V_{tb} \end{pmatrix}
												 \begin{pmatrix} \Pdown \\ \Pstrange  \\ \Pup  \end{pmatrix}
\end{equation}

It is a $3 \times 3$ unitary matrix described by four independent parameters: three angles ($\theta_{ij}$) and one phase ($\delta_{13}$). 
Different equivalent representations of the CKM matrix can be found in literature but the Particle Data Group recommends the standard CKM
parameterisation:

\begin{equation}
V_{CKM} = 	\begin{pmatrix}	
						c_{12}c_{13}				&                  -s_{12} c_{13}					& s_{13}e^{-i \delta_{13}} 	\\
			-s_{12}c_{23}-c_{12}s_{23}s_{13}e^{i\delta_{13}}	& -c_{12}c_{23}-s_{12}s_{23}s_{13}e^{i\delta_{13}}	& s_{23}c_{13} 			\\
			s_{12}s_{23}-c_{12}c_{23}s_{13}e^{e\delta_{13}}	& -c_{12}s_{23}-s_{12}c_{23}s_{13}e^{i\delta_{13}}	& c_{23}c_{13}
			\end{pmatrix}		
\end{equation}
where $c_{ij} \equiv \textrm{cos}\,\theta_{ij}$ and  $s_{ij} \equiv \textrm{sin}\,\theta_{ij}$, 
with $i$ and $j$ labelling the generations ($i,\,j \in \{1,2,3\}$).
The angles $\theta_{12}$,  $\theta_{23}$ and  $\theta_{13}$ are known as Euler angles.  
The complex phase $\delta_{13}$ allows the \CP violation \cite{Chau:1984fp}.
%There are other popular parameterisations such as the Kobayashi-Maskawa one
%or the Wolfenstein one. 

The different elements of the CKM matrix are determined experimentally and are
summarised in Table \ref{tab:Chap1:CKM}.
 As can be seen in this table, the largest values correspond to the diagonal elements of the CKM matrix.
This implies that the processes that do not change the flavour are strongly preferred over the 
family-changing charged currents. For instance, for the top quark, the decay to any of the three down-type 
quarks is allowed but only $|V_{td}|^{2}\times 100\% = 0.0064\%$ of times will decay to a down quark and
 $|V_{ts}|^{2}\times 100\% = 0.14\%$ to a strange quark.
%(using lowest value of $V_{tb} within its uncertainty) $|V_{tb}|^{2} \times 100 = 97 $

\begin{table}[]
\centering
%\begin{tabular}{c|l l}
\begin{tabular}{c|l }
\toprule
CKM element 	  & Value \\
\midrule
   $V_{ud}$         &   $0.9740 \pm 0.00011$	%& 	\cite{Hardy:2017G0} \cite{Pocanic:2003pf}  									
   	\\
   $V_{us}$         &   $0.22650 \pm 0.00048$    	%&	\cite{PhysRevLett.41.1692} \cite{Cabibbo:2003ea}  								
   	\\
   $V_{cd}$         &   $0.22636 \pm 0.0048$    		%&	\cite{FlavourLatticeAveragingGroup:2019iem} \cite{BaBar:2014xzf} \cite{CLEO:2009svp} 	
   	\\
   $V_{cs}$         &   $0.97340\pm0.011$		%&	\cite{CHORUS:2005nog}\cite{BaBar:2010ixw}  \cite{Belle:2006idb} 					
   	\\
   $V_{cb}$         &   $0.04053^{+0.00083}_{-0.00061}$	    	%&	\cite{HFLAV:2019otj}\cite{Belle:2017rcc}										
   	\\
   $V_{ub}$         &   $0.00361^{+0.00011}_{-0.00009}$   	%&	\cite{BaBar:2011xxm} \cite{Colquhoun:2015mfa}\cite{Bauer:2000xf}\\ %\cite{Neubert:1993um}
  	\\ 	
   $V_{td}$          &   $0.00854^{+0.00023}_{-0.00016}$    	%&	\cite{LHCb:2013lrq}\cite{Misiak:2015xwa}										
   	\\
   $V_{ts}$          &   $0.03978^{+0.00082}_{-0.00060}$  		%&	\cite{LHCb:2013lrq}\cite{Misiak:2015xwa}										
   	\\
   $V_{tb}$          &   $ 0.999172^{+0.000024}_{-0.000035}$ 		%&	\cite{CDF:2015gsg}\cite{CMS:2015nrd}\cite{D0:2011viq}	  						
   	\\
\bottomrule
\end{tabular}
\caption{Magnitude of the nine elements of the CKM matrix. The mean for the different measurements has been done by \cite{ParticleDataGroup:2020ssz}. Note how the elements that refer to quarks of the same generation are favoured over the flavour-changing currents.}
%\pablo{Igual me he venido arriba con las referencias. Son las que daba el Particle Data Group}
\label{tab:Chap1:CKM}
\end{table} %source: https://pdg.lbl.gov/2021/reviews/contents_sports.html


% is currently believed that CP-violation during the early universe can account for the "excess" matter, although the debate is not settled
%The \CP violation reflects the asymmetry between matter and antimatter. \pablo{ \CP violation was found with kaon oscillations <- Cite this discovery}

%La violacion cp se debe a que hay tres familias -> la fase compleja de la matriz CKM es la única fuente de violación cp


 %CPviolation depends on the CKM matrix elements associated with weak phase and strong phase:
% page 8: https://arxiv.org/pdf/2201.02385.pdf
% CP violation in ckm matrix: https://arxiv.org/pdf/hep-ph/0406184.pdf page 7

%Due to its ability to change the flavour of quarks and leptons, the theory describing the behaviour of the weak force is the quantum flavourdynamics but
%this term is rarely used because this interaction is better understood by the EW theory.
 
%\pablo{check: \cite{Hung:2021tmi}}


\subsubsection{Electroweak unification}
%After the discovery of $\mathcal{P}$ in Wu experiment, a search began to relate weak and electromagnetic interactions.
%Glashow-Salam-Weinberg model
At energies above the scale of the mass of the weak vector bosons ($E_{EW} \sim m_{Z} \sim m_{W} \sim 100\,\textrm{GeV} $), the electromagnetic 
and weak interactions are unified into the Electroweak (EW) force. In other words, electromagnetism and weak interactions are simultaneously described by the symmetry group $SU(2)_{L} \bigotimes U(1)_{Y}$. 
The subindex $L$ refers to left-handed fields and $Y$ to the weak hypercharge.  In contrast, at low energies, this interactions are treated as independent phenomena, 
the electromagnetism is described QED and the weak interaction by Fermi's theory.

In the EW model (Glashow-Salam-Weinberg model), two new quantum numbers  are assigned to the particles of the SM: the weak isospin ($\overrightarrow{T}$) and $Y$.
Here, the left-handed chiral states of fermions form isospin doublets ($\chi_{L}$) with $T_{3} = \pm 1/2$ and the right-handed form chiral states are composed of isospin singlets ($\chi_R$)  with $T_{3} = 0$.
For a particle, $T_{3}$ is the third component of the $\overrightarrow{T}$, which is related to the electric charge ($Q$) and the $U(1)$ hypercharge by Gell-Mann-Nishijima relation:
\begin{equation}\label{eq:chap1:EW:GMN}
	Q = T_{3} + \frac{1}{2} Y
\end{equation}
With this expression, the electromagnetic coupling and the electroweak couplings are connected.
Having $\chi_L$ with $T_{3} = \pm 1/2$ and $\chi_{R}$  with $T_{3} = 0$ implies that a $SU(2)$ weak interaction can rotate left-handed particles 
(i.e. convert a left-handed \Pelectron into a left-handed $\Pneutrino_{e}$ emitting a \PWminus) but cannot do the same with right-handed.

Using the gauge invariance principe it is possible to find the QED and QCD Lagrangians, as it is described in Sections \ref{sec:chap1:QED} and \ref{sec:chap1:QCD} respectively.

The free Lagrangian, as in the case of QED and QCD is:
\begin{equation}\label{eq:chap1:EW:LagrangianFree}
\begin{split}
	\mathcal{L} 	&= i \sum_{j=1}^{3} \bar{\Psi}(x) \gamma^{\mu} \partial_{\mu} \Psi(x) \\
				& = i \sum_{j=1}^{3} \bar{\chi_{L}}(x) \gamma^{\mu} \partial_{\mu} \chi_{L}(x)+  i \sum_{k=1}^{3} \bar{\chi_{R}}(x) \gamma^{\mu} \partial_{\mu} \chi_{R}(x)
\end{split}
\end{equation}
where the wave function $\Psi$ has been spited into the left isospin doublets $\chi_{L}$ and right isospin singlets $\chi_{R}$. The indices $j$ and $k$ run over 
the three generations of the SM. 
This Lagrangian should be invariant when a gauge transformation under the $SU(2)_{L} \times U(1)_{Y}$ symmetry group in the flavour space is applied:
\begin{align}
	\chi_{L}(x) 	&\xrightarrow{\text{$SU(2)_{L} \times U(1)_{Y}$}} \chi_{L}'(x) =  exp\{ i \alpha^{n} \tau_{n} \}\, exp\{ i \beta y \}\,\chi_{L}(x) \\
	\chi_{R}(x) 	&\xrightarrow{\text{$SU(2)_{L} \times U(1)_{Y}$}} \chi_{R}'(x) =  exp\{ i \beta y \} \,\chi_{R}(x)
\end{align}
with $\alpha, \, \beta \in \mathds{R}$ and $n \in \{1,2,3 \}$.
This transformation is given by the generators of $SU(2)_{L} \times U(1)_{Y}$, i.e. the Pauli matrices ($\tau_{n}$) and the weak hypercharge $y$. 
Note that $SU(2)_L$ transformation, $exp\{ i \alpha^{n} \tau_{nu} \}$, only acts on the doublet fields. This term containing the Pauli matrices is 
non-abelian like in QCD and, like in QCD, this leads to self-interacting terms.

To ensure invariance under  $SU(2)_{L} \times U(1)_{Y}$, four different gauge fields have to be added (three from $SU(2)$ and one from $U(1)$).
Four is also the correct number of gauge bosons needed to describe EW interactions: \PWplus, \PWminus, \PZ and \Pgamma.
While the three week isospin currents couple to the triplet of vector bosons $\PW^{n}_{\mu}$ with $n \in \{1,2,3 \}$, the weak hypercharge
couples to an isosinglet $B_\mu$.  The fields $\PW^{1}_{\mu}$ and $\PW^{2}_{\mu}$ are electrically charged whereas $\PW^{3}_{\mu}$ and $B_\mu$
are neutral fields. 
The EW covariant derivative is defined as:
\begin{align}
	D^{\mu} \chi_{L_{j}}(x)	& =  [\partial_{\mu} - i g \frac{\tau_{i}}{2}\PW^{i}_{\mu}(x)  - i g' \frac{y_j}{2}  B_{\mu}(x) ] \, \chi_{L_{j}} (x)  &&  i \in [1,2,3] 	\label{eq:chap1:EW:CovariantDerivatice1} \\
	D^{\mu} \chi_{R_{j}}(x)	& = [\partial_{\mu} - i g' \frac{y_j}{2}  B_{\mu}(x)] \,\chi_{R_{j}}(x)		,										\label{eq:chap1:EW:CovariantDerivatice2}
\end{align}
where $g$ and $g'$ are the interaction couplings to $\PW^{i}_\mu$ isotriplet and the $B_{\mu}$ isosinglet.

Using the derivatives in Equations \ref{eq:chap1:EW:CovariantDerivatice1} and \ref{eq:chap1:EW:CovariantDerivatice2}, the Lagrangian 
in \ref{eq:chap1:EW:LagrangianInvariant} is already invariant under local $SU(2)_{L} \times U(1)_{Y}$ transformations:
\begin{equation}\label{eq:chap1:EW:LagrangianInvariant}
	\mathcal{L} = i \sum_{j=1}^{3} \bar{\chi_{L}}^{j}(x) \gamma^{\mu} D_{\mu} \chi_{L}^{j}(x)+  i \sum_{k=1}^{3} \bar{\chi_{R}}^{k}(x) \gamma^{\mu} D_{\mu} \chi_{R}^{k}(x) 
\end{equation}

Finally, if kinetic terms for the gauge bosons are included in \ref{eq:chap1:EW:LagrangianInvariant}, the EW SM Lagrangian is obtained:
\begin{equation}
\begin{split}\label{eq:chap1:EW:FinalL}
	\mathcal{L}_{EW}  =	& i \sum_{j=1}^{3} \bar{\chi_{L}}^{j}(x) \gamma^{\mu} D_{\mu} \chi_{L}^{j}(x)+  i \sum_{k=1}^{3} \bar{\chi_{R}}^{k}(x) \gamma^{\mu} D_{\mu} \chi_{R}^{k}(x) \\
					& - \frac{1}{4} W^{n}_{\mu \nu}(x)W_{n}^{\mu \nu}(x) - \frac{1}{4} B_{\mu \nu}(x)B^{\mu \nu}(x)
\end{split}
\end{equation}
Where the addition of kinetic terms gives rise to cubic and quadratic self-interactions among the gauge fields.
Note that the mass terms of the fields are forbidden in order to ensure local gauge invariance and
since the observed $\PWplus$, $\PWminus$ and $\PZ$ bosons have masses different from zero, 
 for the moment let's assume that something breaks the symmetry generating the observed masses.

%This Lagrangian describes the interactions between gauge vector bosons and fermions below
%\begin{align*}
%	\chi_{L} 	&& =  \begin{pmatrix} \Pnue \\ \Pe  \end{pmatrix}_{L} & \begin{pmatrix} \Pnum \\ \Pmu  \end{pmatrix}_{L} & \begin{pmatrix} \Pnut \\ \Ptau  \end{pmatrix}_{L} & \begin{pmatrix} \Pup \\ \Pdown  \end{pmatrix}_{L} &\begin{pmatrix} \Pcharm \\ \Pstrange  \end{pmatrix}_{L} &\begin{pmatrix} \Ptop \\ \Pbottom \end{pmatrix}_{L}   \\
%	\chi_{R} 	&&=	 \Pe_{R} & \Pmu_{R} & \Ptau_{R} & \Pup_{R}  \Pdown_{R} & \Pcharm_{R}  \Pstrange_{R} & \Ptop_{R}   \Pbottom_{R}
%\end{align*}
%\pablo{Esto se ve muy feo: Editar $\chi_{L}$ y $\chi_{L}$.}

The in $\mathcal{L}_{EW}$ in \ref{eq:chap1:EW:FinalL} can be divided in two different parts according 
to the charge of the bosons: charged currents and neutral currents.
Relating the charged currents ($\PW^{1}_{\mu}$ and $\PW^{2}_{\mu}$) to the \PWplus and \PWminus 
bosons of the SM and the neutral ($\PW^{3}_{\mu}$ and $B_\mu$) ones with the $\PZ$ and $\Pgamma$, 
it is possible to build linear combinations fo the original gauge fields that define the SM bosons.

%\pablo{Reescribir a partir de aquí para que quede más bonito\\}

Therefore, from the charged-current interactions, the $\PWplus$ and $\PWminus$ bosons 
are:
\begin{equation}
	\PWpm \equiv \frac{1}{\sqrt{2}}(\PW^{1}_{\mu} \mp i\PW^{2}_{\mu})\, .
\end{equation}

While for the neutral-current these combinations can be defined as a rotation of the so called Weinberg 
(or weak mixing) angle $\theta_{W}$:
\begin{equation*}\label{eq:chap1:EW:ZmuAmu}
		\begin{pmatrix} \PW_{\mu}^3 \\ B_\mu \end{pmatrix} \equiv
		\begin{pmatrix} \textrm{cos} \,\theta_{W} & \textrm{sin} \,\theta_{W} \\  -\textrm{sin} \,\theta_{W} & \textrm{cos}\, \theta_{W} \end{pmatrix} \begin{pmatrix} \PZ_{\mu} \\ A_{\mu} \end{pmatrix} \, .
\end{equation*}
Rewriting this equation, the photon and $\PZ$-boson fields are 
\begin{align}
	A_{\mu}	&= B_{\mu} \textrm{cos} \,\theta_{W}+ W^{3}_{\mu}  \textrm{sin} \,\theta_{W}
	Z_{\mu}	&= -B_{\mu} \textrm{cos} \,\theta_{W}+ W^{3}_{\mu}  \textrm{sin} \,\theta_{W} \, .
\end{align}
In order to ensure that this $A_{\mu}$ is the one of QED, apart from the Gell-Mann-Nishijima relation (Equation
\ref{eq:chap1:EW:GMN}), it is requiered that the couplings of the \Pgamma, \PWpm and \PZ satisfy the relation:
\begin{equation}
	g \, \textrm{sin} \,\theta_{W} = g' \,\textrm{cos}\, \theta_{W} = e \, .
\end{equation}

Within the unified EW  model, once $ \theta_{W}$ is known, the mass of $\PZ$ is specified.
Current measurements of $ \theta_{W}$ give a value of $\textrm{sin}^{2} \theta_{W} = 0.2310 \pm 0.0005$ \cite{CMS:2018ktx}. 

%\pablo{Comentar algo el GIM mechanism y las FCNC }


There is no mass term for the bosons in the EW Lagrangian that has been obtained in \ref{eq:chap1:EW:FinalL} by demanding the $SU(2)_{L} \times U(1)_{Y}$ local invariance, 
which enters in contradiction with the experimental observations for the \PW and \PZ bosons ($m_{Z,W} \sim 80\,$GeV).  The introduction of such a mass term would break the symmetry, however, the 
it is possible to add the mass for the \PW and \PZ bosons without loosing the properties of the symmetry. The method to do so is known as Englert–Brout–Higgs–Guralnik–Hagen–Kibble mechanism
or, more commonly, just as Higgs mechanism. This mechanism is described in Section \ref{sec:chap1:ParticleMasses}.


%SU(2) con U(1) te da los bosons de W.Z y fotón pero sin masa 
%También  dan las corrientes que se observan experimentalmente

%Tomas $SU(2) \times U(1)$ y lo rompes a través del SSB, salen los W, Z y fotón con las masas que tocan  
%Romper la simetría pero conservando la local gauge invariance. 

%Al incluir el campo de Higgs, el Lagrangiano obtenido es invariante (es invariante en escalas energéticas de EW, es decir, $m_H$ hacia arriba. 
%y por debajo de esas energías, el lagrangiano ya no es invariante ) bajo $SU(2) \times U(1)$ pero los ``estados de la teoría'' no son invariantes 
% porque el estado vacío de la teoría no permanece invariante y, por lo tanto, el resto de estados tampoco. %Lo que se hace aquí es pasa al gauge unitario,



% From EW to Higgs https://cds.cern.ch/record/475776/files/9811456.pdf

%%%%%%%%%%%%%%%
%              QCD                    %
%%%%%%%%%%%%%%%
\subsection{Quantum chromodynamics}
\label{sec:chap1:QCD}

%%%%%%
% Quarks and Colour
%%%%%%%
\subsubsection{Quarks and colour}%\mbox{}\\
QCD is QFT-based theory for describing the strong interactions between quarks and gluons (partons). This type of interaction 
is the responsable   of the nuclear force, the one that acts between the protons and neutrons of atoms binding them together. Without the
strong force, the protons inside the nucleus would push each other apart due to the electromagnetic repulsion. It also holds the quarks within a hadron
together. 

QCD is based in the $SU(3)$ symmetry group and its name derives from the ``colour'' charge, an analogous to the electric charge of QED but for strong interactions.
The colour charge was introduced in 1964 \cite{Greenberg:1964pe} to explain how quarks could coexist within some hadrons apparently having the same 
quantum state without violating the Pauli exclusion principle. To satisfy the Fermi-Dirac statistics it is necessary to add an additional quantum number, the
colour, to the theory. Each specie of quark ($q$) may have three different colours ($q^{\alpha}$, $\alpha=$1, 2, 3): red, green, blue.
Baryons and mesons are described then by the colour singlet combinations:
\begin{align*}
B = \frac{1}{\sqrt{6}} \epsilon^{\alpha \beta \gamma} \ket{\Pquark_{\alpha}\Pquark_{\beta}\Pquark_{\gamma}} &&  M= \frac{1}{\sqrt{3}} \epsilon^{\alpha \beta} \ket{\Pquark_{\alpha}\APquark_{\beta}} \, .
\end{align*} 

Additionally, it is postulated that all hadrons must have a global neutral colour charge, i.e. the hadrons must be ``colourless''. This assumption is known as 
confinement hypothesis and it is made to avoid the existence of non-observed extra states with non-zero colour. It is called colour confinement becase it implies
that it is not possible to observe free quarks since they carry colour charge and, hence, they have to be confined within colour-singlet combinations.
 Figure \ref{fig:Chap1:ColourCharge}  depicts how different colours and anticolours combine to create the ``colourless'' state.

\begin{figure}
\centering
\begin{subfigure}{.5\textwidth}
  \centering
  \includegraphics[width=.4\linewidth]{Chapter1/QCD-Colours.png}
  \caption{Quark colours combine to be colourless.}
\end{subfigure}%
\begin{subfigure}{.5\textwidth}
  \centering
  \includegraphics[width=.4\linewidth]{Chapter1/QCD-AntiColours.png}
  \caption{Antiquark colours also combine to be colourless.}
\end{subfigure}
\caption{Colour charge combinations for quarks and antiquarks. Due to the confinement, the hadrons are colourless.}
\label{fig:Chap1:ColourCharge}
\end{figure}


%%%%%%
% % Non-Abelian gauge symmetry
%%%%%%%
% https://fse.studenttheses.ub.rug.nl/12978/1/ThesisCasperDijkstra.pdf
\subsubsection{Gauge invariance for \textit{SU(3)}}%\mbox{}\\
The dynamics of the quarks and gluons are controlled by the QCD Lagrangian. Using the power of the gauge invariance principle 
it is possible to deduce $\mathcal{L}_{QCD}$ similarly to the reasoning developed in Section \ref{sec:chap1:QED}.
Firstly, let's denote a quark field of colour $\alpha$ and flavour $f$ by $q_{f}^{\alpha}$. 
The vector $q^{T}_{f}\equiv (q_{f}^{1}, q_{f}^{2}, q_{f}^{3})$ is defined under the $SU(3)$ colour space,  
meaning that each dimension corresponds to a colour.
The Lagrangian
\begin{equation}\label{eq:chap1:QCD:Lagrangian_0}
\mathcal{L}_{0} = \sum_{f} \APquark_{f} (i \gamma^{\mu} \partial_{\mu} - m_{f})\Pquark_{f}
\end{equation}
is invariant under global $SU(3)$ transformation in the colour space,
\begin{align} \label{eq:chap1:QCD:q_trasnformation}
q_{f}^{\alpha} \rightarrow (q_{f}^{\alpha})' = U^{\alpha}_{\beta}q^{\beta}_{\alpha}, && UU^{\dagger} = U^{\dagger}U = 1, && \textrm{det }U = 1 .
\end{align}

In the $SU(N)$ algebra, $SU(N)$ is the group of $N \times N$ unitary matrices ($U$) which can be written in the form $U = exp \{ i (\lambda^{a}/2) \theta_{a} \}$ with $a = 1,2,...,N^{2}-1$.
Therefore, the SU(3) matrices can be written as
\begin{equation}\label{eq:chap1:QCD:U_matrix}
	U = exp \big\{ i \frac{\lambda^{a}}{2}\theta_{a} \big\}\, ,
\end{equation}
where the index $a$ goes from 1 to 8 for the arbitrary parameter $\theta_{a}$ and  $\frac{\lambda^{a}}{2}$, which denotes the fundamental representation of the $SU(3)$ algebra.
The Einstein notation for summation over repeated indices is implied. The matrices $\lambda^{a}$ are traceless and satisfy the commutation relations \cite{Pich:2007vu}:
%\pablo{Why does SU(3) has 8 gauge parameters? SU(N) is the group of $N \times N$ unitary matrices ($U$) }%For $SU(N)$  with $N=3$, the fundamental representation correspond to the eight Gell-Mann matrices.2 }
\begin{equation} \label{eq:chap1:QCD:ConmutationSU3}
	\big[ \frac{\lambda^{a}}{2}, \frac{\lambda^{b}}{2}\big] = i f^{abc} \frac{\lambda^{c}}{2} \, ,
\end{equation}
being $f^{abc}$ the $SU(3)_C$ structure constants, which are real and totally antisymmetric. 

To satisfy the gauge invariance requirement, the Lagrangian has to be invariant under $SU(3)$ local transformations, i.e, transformations in which 
the phase is dependent of the space-time location, $\theta_{a} = \theta_{a} (x)$. To fulfil the condition, the quark derivatives in the 
Lagrangian in \ref{eq:chap1:QCD:Lagrangian_0} have to  be substituted by covariant objets. Since there are eight independent gauge 
parameters, eight different gauge  bosons $G_{a}^{\mu}(x)$ are needed\footnote{The eightfold multiplicity of gluons is labeled by a combination of color and anticolor charge (e.g. red–antigreen)}. 
This bosons are the eight gluons and  the new covariant objects are:
\begin{equation*}
	D^{\mu}q_{f} \equiv \big[ \partial_{\mu}  + i g_{s} \frac{\lambda^{a}}{2} G_{a}^{\mu}(x) \big] q_{f} \equiv [ \partial_{\mu}  + i g_{s} G^{\mu}(x) ] q_{f} \, .
\end{equation*}
The compact matrix notation is used  $[G^{\mu}(x)]_{\alpha \beta} \equiv \big( \frac{\lambda^{a}}{2} \big)_{\alpha \beta} G_{a}^{\mu}(x)$.

To ensure that the covariant derivative ($D^{\mu}q_{f}$) transforms like the $q_f$, the transformation of the gauge fields are:
\begin{align}\label{eq:chap1:QCD:G_trasnformation}
	D^{\mu} \rightarrow (D^{\mu})'  = UD^{\mu}U^{\dagger} && G^{\mu} \rightarrow (G^{\mu})' = UG^{\mu}U^{\dagger}+\frac{i}{g_{s}} (\partial_{\mu} U )U^{\dagger} \, .
\end{align}
The quark and gluon fields transform under an infinitesimal local transformation, i.e. $\theta_{a} (x) =  \delta \theta_{a}(x) \approx 0$, the $SU(3)_C$ unitary matrices (eq. \ref{eq:chap1:QCD:U_matrix}) can be expressed as their first order expansion:
\begin{equation*}
	U = exp \big\{ i \frac{\lambda^{a}}{2}\theta_{a}(x) \big\} \approx 1 + i \big( \frac{\lambda^{a}}{2} \big) \delta \theta_{a}(x)
\end{equation*}
and, consequently, the transformations for que colour-vector field (eq. \ref{eq:chap1:QCD:q_trasnformation}) and 
gluon field (eq. \ref{eq:chap1:QCD:G_trasnformation}) become:
%\cite{Pich:2007vu},
\begin{align*}
	q^{\alpha}_{f}	& \rightarrow (q^{\alpha}_{f})' = q^{\alpha}_{f} + \big( \frac{\lambda^{a}}{2} \big)_{\alpha \beta} \delta \theta_{a} q_{f}^{\beta} \\
	G_{a}^{\mu}	& \rightarrow (G_{a}^{\mu})' =  G_{a}^{\mu} - i \frac{i}{g_{s}} \partial_{\mu} (\delta \theta_{a}) - f^{abc}\delta \theta_{b}G_{c}^{\mu}.
 \end{align*}
 

In contrast to the transformation for the photon field in QED (Equation \ref{eq:chap1:AmuTransformation}), 
the non-commutativity\footnote{Because the generators of $SU(3)$ do not conmute, QCD is known as 
non-Abelian gauge theory.} of the $SU(3)_C$ matrices give rise to an additional term involving the gluon 
fields themselves ($- f^{abc}\delta \theta_{b}G_{c}^{\mu}$), as the relation \ref{eq:chap1:QCD:ConmutationSU3} expresses.
%This transformation are more complicated than the ones of equation \ref{eq:chap1:AmuTransformation}.
%The adjoint representation of an $SU(N)$ group is given by the $(N^{2}-1)\times(N^{2}-1)$ matrices $(\lambda^{a} / 2)_{bc} \equiv -i f^{abc}$.
For constant $\delta \theta_{a}$, the transformation rule for the gauge fields is expressed in terms of the structure constants $f^{abc}$; 
thus, the gluon fields belong to the adjoint representation for the colour group. 
There is a unique coupling at $SU(3)_C$, $g_s$. All the colour-triplet flavours couple to the gluon fields with exactly the same interaction strength. 

It is necessary to introduce the corresponding fields strengths to build a gauge-invariant kinetic terms for the gluon fields.
\begin{align*}
	G^{\mu \nu}		& \equiv -i\frac{-i}{g_{s}} [D^{\mu}, D^{\nu}] =  \partial_{\mu}G^{\nu} - \partial_{\nu}G^{\mu} + i g_{s}[G^{\mu},G^{\nu}] \equiv \frac{\lambda^{a}}{2}G_{a}^{\mu \nu}(x) \\
	G_{a}^{\mu \nu}	& \equiv \partial_{\mu}G^{\nu}_{a} - \partial_{\nu}G^{\mu}_{a} - g_{s} f^{abc} G_{b}^{\mu}G_{c}^{\nu}\, .
\end{align*} 

Under a $SU(3)_C$ transformation,
\begin{equation}
	G^{\mu \nu} \rightarrow (G^{\mu \nu})' = UG^{\mu \nu}U^{\dagger}
\end{equation}
and the colour trace $\textrm{Tr}(G^{\mu \nu}G_{\mu \nu}) = \frac{1}{2}G^{\mu \nu}G_{\mu \nu}$ remains invariant.  Normalising the gluon kinetic term, 
the $SU(3)_C$ invariant QCD Lagrangian is obtained:
\begin{equation}\label{eq:chap1:QCD:Lagrangian_FinalCompact}
	\mathcal{L}_{QCD} \equiv   -\frac{1}{4}G^{\mu \nu}_{a}G_{\mu \nu}^{a} + \sum_{f} \APquark_{f} (i \gamma^{\mu} D_{\mu} - m_{f})\Pquark_{f} \, .
\end{equation}

Note how the gluon-gluon vertex is find by demanding the gauge invariance under local $SU(3)_C$ transformation.
A mass term is forbidden for the gluon fields by the $SU(3)_C$  gauge symmetry because a something of the form 
$\frac{1}{2}m^{2}_{G}G^{\mu}_{a}G_{\mu}^{a}$ would not be invariant under the transformation in \ref{eq:chap1:QCD:G_trasnformation}.
The gluons are, then, predicted by the theory to be spin-1 massless particles.

Thanks to the colour symmetry properties, this Lagrangian looks very simple and all its interactions depend on the strong coupling
constant, $g_s$. In contrast to the Lagrangian derived for QED (eq \ref{eq:chap1:Dirac2}), in $\mathcal{L}_{QCD}$ the boson field 
have a self-interacting term. This gluon self-interactions give rise to the triple and quadratic gluon vertex (center and right diagrams in Figure \ref{fig:Chap1:gluonvertex}).
This self-interactions among the gluon fields can explain features the asymptotic freedom and confinement, properties that were not
present in QED. The asymptotic freedom causes interactions between particles to become asymptotically weaker as the energy 
scale increases and the corresponding length scale decreases. The confinement implies that the strong forces increase with the distance, therefore, 
as two colour charges are separated, at some point it becomes energetically favorable for a new quark-antiquark pair to appear rather than keep
getting further. This new quarks bond with the previous two, preventing single quarks to be isolated. This mechanism, depicted in Figure \ref{fig:Chap1:colorConfinement},
explains why the strong interaction is responsable for keeping the quarks together forming hadrons.

\begin{figure}
    \centering
    \includegraphics[width = 0.75\textwidth]{Chapter1/QCD_vertex}
    \caption{The predicted QCD interaction vertices arising from the requirement of $SU(3)_C$ local gauge invariance. The presence of the triples and quadruple gluon vertices is possible to the Non-Abelian nature of $SU(3)_C$.}
    \label{fig:Chap1:gluonvertex}
\end{figure}

\begin{figure}
    \centering
    \includegraphics[width = 0.75\textwidth]{Chapter1/ConfinementQCD}
    \caption{The QCD colour confinement explains the inseparability of quarks inside a hadron in spite of investing ever more energy. In this example, the mechanism is shown for a meson.}
\label{fig:Chap1:colorConfinement}\end{figure}


%\pablo{Maybe, expand the QCD Lagrangian in \ref{eq:chap1:QCD:Lagrangian_FinalCompact} and explain its different elements.}


%%%%%%%%%%%%%%%%%%%%%
%                  Particle masses                   %
%%%%%%%%%%%%%%%%%%%%%
\subsection{Particle masses}
\label{sec:chap1:ParticleMasses}
For the QED Lagrangian, $\mathcal{L}_{QED}$  (eq. \ref{eq:chap1:QED_Complete}), it is clear how the mass of the photon must be zero in order to satisfy the $U(1)$ local gauge symmetry because,
if a mass term for the vector gauge field $A_{\mu}$ is included, the $\mathcal{L}_{QED}$ would be:
\begin{equation*}\label{eq:chap1:QED_massivephoton}
	\mathcal{L}_{QED} =  \bar{\Psi}(x) (i  \gamma^{\mu} \partial_{\mu} - m ) \Psi(x) - eQ \bar{\Psi}(x) \gamma^{\mu}A_{\mu} \Psi(x) - \frac{1}{4}F_{\mu \nu}(x) F^{\mu \nu}(x) + \frac{1}{2}m_{\Pgamma}^{2}A_{\mu}A^{\mu}
\end{equation*}
and, with the $U(1)$ transformation in Equation \ref{eq:chap1:AmuTransformation}, the new mass term becomes: 
\begin{equation*}
	\frac{1}{2}m_{\Pgamma}^{2}A_{\mu}A^{\mu} \rightarrow \frac{1}{2}m_{\Pgamma}^{2} (A_{\mu} + \frac{1}{e}\partial_{\mu}\theta ) (A^{\mu} +\frac{1}{e}\partial^{\mu}\theta) \neq \frac{1}{2}m_{\Pgamma}^{2}A_{\mu}A^{\mu}\, .
\end{equation*}
Confirming that the photon mass term is not invariant under local $U(1)$ and, consequently, that the photon must be massless to satisfy the gauge invariance.
Experimental efforts to measure the mass of the photon have set an upper limit of  $m_{\Pgamma} \leq 1 \times 10^{-18} $ eV \cite{Ryutov:2007zz}.
%Recent limits form cosmological measurements set the limit in $m_{\Pgamma} \leq 4 \times 10^{-15} $ \cite{Wei:2020wtf}.

With the Lagrangian of QCD in Equation \ref{eq:chap1:QCD:Lagrangian_FinalCompact} happens the same, the mass term for the gluon fields are forbidden 
by the $SU(3)_{C}$ gauge symmetry. Therefore, the mediating bosons for the strong interactions are massless as well (experimentally, a mass as large as upper limits
of a few MeV have been set, see \cite{Yndurain:1995uq}).


While the prohibition of mass terms for the bosons of QED and QCD is not a problem, this requirement also applies to the $SU(2)_L$.
This condition enters into open contradiction with the measurements of large masses for the
%\PW ($m_{\PW} = 80.370 \pm 0.007 \textrm{ (stat.)}\pm 0.017\textrm{ (syst.)}$ GeV\cite{ATLAS:2017rzl}) 
%and \PZ ($m_{\PZ} = 91.1852 \pm 0.0030$ GeV \cite{OPAL:2000ufp}) bosons of weak interactions. 
%\pablo{<- to do: Aquí no estoy utilizando las mismas medidas de la masa de $m_{\PW}$ y $m_{\PZ}$ que antes. He %de escoger una and stick to it.}
\PW and \PZ bosons of weak interactions.


For weak interactions, the problem of massless particles do not only affect the bosons.  Since under the $SU(2)_L$ transformations left-handed particles
transform as weak isospin doubles and right-handed particles as isospin singlets, the mass term of a spinor field $\Psi$ written as chiral states also
breaks the required gauge invariance: $-m \bar{\Psi}(x) \Psi(x) = -m \bar{\Psi}(x)(P_{R} + P_{L})\Psi(x) = -m (\bar{\Psi}_{R}(x) \Psi_{L}(x) + \bar{\Psi}_{L}(x) \Psi_{R}(x))$

The Higgs mechanism describes how both the \PW and \PZ bosons and the fermions acquire mass without breaking the local gauge symmetry of the SM. 

\subsubsection{The Higgs mechanism}\label{sec:chap1:ParticleMasses:HiggsMechanism}

\paragraph{Goldstone theorem and spontaneous symmetry breaking}\mbox{}\\
For a scalar field $\phi$ with a Lagrangian of the form:
\begin{equation}\label{eq:chap1:HiggsMechanism:Example}
	\mathcal{L} = \frac{1}{2}\partial_{\mu}\phi_{i}\partial^{\mu}\phi_{i} -V(\phi) \textrm{ where } V(\phi) = \frac{1}{2}\mu^{2}\phi_{i} \phi_{i}   + \frac{1}{4}\lambda(\phi_{i} \phi_{i})^{2} \, .
\end{equation}
This Lagrangian is invariant under $\phi_{i} \rightarrow \phi_{i}'= R_{ij}\phi_{j}$, where $R_{ij}$ are rotational matrices in 4-dimensions.
 The mass term is the one with $\phi_{i} \phi_{i}$ and 
the parameter $\lambda$ has to be positive for $\mathcal{L}$ to describe a physical system,  if $\lambda$ <0 the potential is unbounded from below.
Contrary, the parameter $\mu^{2}$ can be either positive or negative.  
As depicted in Figure \ref{fig:Chap1:SM:HiggsMechanism:PotentialExample:Positive}, if $\mu^{2}>0$, the vacuum expectation value 
(i.e. minimum of potential) is located at the origin $\phi_0$ and this $\mathcal{L}$  would describe a spin-0 particle of mass $\mu$. 
However, if $\mu^{2}<0$, the potential $V(\phi)$ has the form of Figure \ref{fig:Chap1:SM:HiggsMechanism:PotentialExample:Positive} and $\mathcal{L}$
would not represent anymore the Lagrangian of a particle of mass $\mu$. The vacuum expectation value is now multivalued:
\begin{equation*}
	\phi_{0} = \pm \sqrt{-\frac{\mu^{2}}{\lambda}} \equiv \pm v \, .
\end{equation*}

\begin{figure}
\centering
\begin{subfigure}{.5\textwidth}
  \centering
  \includegraphics[width=.74\linewidth]{Chapter1/Higgs_V_muplus}
  \caption{}
  \label{fig:Chap1:SM:HiggsMechanism:PotentialExample:Positive}
\end{subfigure}%
\begin{subfigure}{.5\textwidth}
  \centering
  \includegraphics[width=.74\linewidth]{Chapter1/Higgs_V_muminus}
  \caption{}
  \label{fig:Chap1:SM:HiggsMechanism:PotentialExample:Negative}
\end{subfigure}
\caption{The potential $V(\phi)$ of Lagrangian \ref{eq:chap1:HiggsMechanism:Example} for(a) $\mu^{2}$ positive and (b) negative.}  
\label{fig:Chap1:SM:HiggsMechanism:PotentialExample}
\end{figure}
Expanding the field around the minima at $\phi_{i} =(0,0,0,v)$, the $\mathcal{L}$ becomes:
\begin{equation}
\begin{split}
	\mathcal{L} 	&=  \frac{1}{2}\partial_{\mu}\sigma\partial^{\mu}\sigma  + \mu^{2}\sigma^{2} -\sqrt{\mu^{2}\lambda}\sigma^{3}   -\frac{1}{4}\lambda^{4} \\
				& + \frac{1}{2}\partial_{\mu}\pi_{i}\partial^{\mu}\pi_{i} -\frac{1}{4}\lambda (\pi_{i}\pi_{i})^{4} - \lambda v \pi_{i}\pi_{i}\sigma - \frac{1}{2}\pi_{i}\pi_{i}\sigma^{2} \, ,
\end{split}
\end{equation}
where $i$ runs from 1 to 3. Here $\sigma = \phi_{4}-v$ and $\pi_{i} = \phi_{i}$ are new boson fields, being the latter massless and the former with a mass of $m_{\sigma}^2 = -2\mu^{2}$.
The new terms break the original symmetry because the symmetry of the Lagrangian is not longer a symmetry of the vacuum, it has been spontaneously broken.
One massive $\sigma$ boson and three massless $\pi_{i}$ bosons with a residual $O(3)$ symmetry have appeared. This is a consequence of the Goldstone theorem which states that 
``for a continuous symmetry group $\mathcal{G}$ spontaneously broken down to a subgorup $\mathcal{H}$, the number of broken generators is equal to the number of massless scalars 
that appear in the theory'' \cite{Goldstone:1962es}. Therefore, since the $O(N)$ group has $N(N-1)/2$ generators, the $O(N-1)$ has $(N-1)(N-2)/2$ and, hence, $N-1$ Goldstone bosons
appear. The example shown is for $N=4$.

\paragraph{The Higgs mechanism in the SM - Bosons}\mbox{}\\
To apply this mechanism to the SM, it is necessary to generate mass for the \PWplus, \PWminus and \PZ bosons while keeping the photon massless. 
%The Higgs mechanism is based on the general idea of spontaneous symmetry breaking, meaning that for a system the ground state is not equivalent to a
%fully symmetric state. Implying that the systems evolves from a ground state which is not invariant under its full symmetry group.
In order to do so, the EW symmetry group $SU(2)_{L} \times U(1)_{Y}$ has to be broken into a $U(1)$ subgroup describing electromagnetism. 
A gauge-invariant interaction that gives masses to fermions without mixing chiral components
is introduced by defining a $SU(2)$ isospin doublet of complex scalar field with hypercharge $Y=1$:
\begin{equation*}
	\Phi = \begin{pmatrix}  \phi^+ \\ \phi_0  \end{pmatrix}\, .
\end{equation*}
Being $\phi^+$ positively charged and $\phi^0$ neutral.  The Lagrangian $\mathcal{L}_{Higgs}$ has to be added to the $\mathcal{L}_{EW}$ in \ref{eq:chap1:EW:FinalL}.
\begin{equation*}\label{eq:chap1:HiggsMechanism:HiggsLagrangianA}
	\mathcal{L}_{Higgs} = (D_{\mu} \Phi)^{\dagger}(D^{\mu} \Phi) - V(\Phi) \textrm{ where } V(\Phi) = \mu^{2}\Phi^{\dagger} \Phi + \lambda (\Phi^{\dagger} \Phi)^{2}\, ,
\end{equation*}
with $\lambda >0$ required for vacuum stability. When $\mu^{2}>0$, the minimum of the potential occurs when both fields ($\phi^+$ and $\phi^0$) are at zero. If $\mu^{2}<0$, the 
minimum of the potential has an infinite number of degenerate states that satisfy $\Phi^{\dagger} \Phi = \mu^{2}/2\lambda$ and the physical vacuum state will correspond 
to any particular point on the circle of Figure \ref{fig:Chap1:SM:HiggsMechanism:Potential}.
Having to chose a particular point breaks the global $U(1)$ symmetry of the Lagrangian. Without loss of generality, in this scenario, the ground state $\Phi_{0}$ can be chosen to be:
\begin{equation*}
	\Phi_{0} = \frac{1}{\sqrt{2}}\begin{pmatrix}  0 \\ v  \end{pmatrix} \textrm{ where } v = 2\sqrt{\frac{\mu^{2}}{\lambda}}\, .
\end{equation*}
being $v$ the vacuum expectation value. This defines the already mentioned circle in the minimum of $V(\Phi)$ in the $\mu^{2}<0$ scenario.

\begin{figure}
    \centering
    \includegraphics[width = 0.8\textwidth]{Chapter1/HiggsPotential}
    \caption{An illustration of the Higgs potential $V(\Phi)$ in the case of $\mu^{2}<0$. Choosing any particular point in the circle defined by $v$ spontaneously breaks the $U(1)$ rotational symmetry. This type of potential is
    frequently called ``Mexican hat''.}
    \label{fig:Chap1:SM:HiggsMechanism:Potential}
\end{figure}

The Lagrangian density must be formulated in terms of deviations from one of these
ground states. This can be done by introducing an excitation, $h(x)$, that can be understood
as a small deviation of the field from the ground state. 
Accordingly, the fields can be expanded around the minimum as:
\begin{equation*}
	\Phi = \frac{1}{\sqrt{2}} \begin{pmatrix}  0 \\ v+h(x)  \end{pmatrix} exp\{  i \chi(x)\}\, .
\end{equation*}
The new field $\chi(x)$ can be set to zero in the so called ``unitary gauge''.
\begin{equation}\label{eq:chap1:HiggsMechanism:SymmetryBreaking}
	\Phi = \frac{1}{\sqrt{2}} \begin{pmatrix}  0 \\ v+h(x)  \end{pmatrix} \, .
\end{equation}
Expanding the covariant derivative of the $\mathcal{L}_{Higgs}$:
\begin{equation*}\label{eq:chap1:HiggsMechanism:Covariant}
\begin{split}
	 (D_{\mu} \Phi)^{\dagger}(D^{\mu} \Phi) 	&=\left\lvert \left( \partial_{\mu} + ig  \frac{\tau^{k}}{2} \PW_{\mu}^{k} (x)+  ig'  \frac{y}{2} B_{\mu}  \right)\right\rvert^{2}  \\
									&= \frac{1}{2} \left\lvert \begin{pmatrix}   \partial_{\mu} +i\frac{1}{2}(gW^{3}_{\mu}+g'\frac{y}{2} B_{\mu}) && i\frac{g}{2}(W^{1}_{\mu}-iW^{2}_{\mu}) \\
									i\frac{g}{2}(W^{1}_{\mu}-iW^{2}_{\mu})  && \partial_{\mu} -i\frac{1}{2}(gW^{3}_{\mu}-g'\frac{y}{2} B_{\mu})
									 \end{pmatrix}   \begin{pmatrix}  0 \\ v+h  \end{pmatrix}\right\rvert ^{2} \\
									 &= \frac{1}{2}(\partial_{\mu} h)^{2} + \frac{1}{8}(v+h)^{2}|W^{1}_{\mu}-iW^{2}_{\mu}|^{2} \\
									 &+  \frac{1}{8}(v+h)^{2}|gW^{3}_{\mu}-g'B_{\mu}| +(\textrm{interaction terms}) \, ,
\end{split}
\end{equation*} %source: https://arxiv.org/pdf/1709.10508.pdf
where the $\tau_k$ with $k={1,2,3}$ are the Pauli Matrices. In this equation there are terms mixing the $W^{3}$ and the $B_{\mu}$ fields that, by using the physical fields defined in Equation \ref{eq:chap1:EW:ZmuAmu}, should disappear since the physical bosons do not mix. Applying the Relation \ref{eq:chap1:EW:ZmuAmu} into the covariant derivative, 
\begin{equation*}
	(D_{\mu} \Phi)^{\dagger}(D^{\mu} \Phi) = \frac{1}{2}+\frac{g^2v^2}{4}W^{+}_{\mu} W^{-\, \mu} + \frac{g^{2} v^{2}}{8 \textrm{cos}^{2}\theta_{W}}Z_{\mu}Z^{\mu} + 0 A_{\mu}A^{\mu}\, ,
\end{equation*}
the \PWplus, \PWminus and \PZ bosons have finally acquired mass. Through the Higgs mechanism, their masses within the SM are:
\begin{align*}
	M_{\PW} = \frac{1}{2}gv && M_{\PZ} = \frac{1}{2}\frac{gv}{\textrm{cos}\,\theta_{W}}
\end{align*}
Additionally, a new scalar field $h(x)$ has appeared with its correspondent mass term, the Higgs field. Note that the $h(x)$ was introduced as a perturbation from the ground
state of the Higgs potential $V(\Phi)$, so the Higgs boson can be understood as an excitation of the Higgs potential. Apart from couplings to the electroweak gauge
fields, the Higgs field has also self-interaction vertices. The mass of this boson is $m_{\PHiggs}=\sqrt{2}\mu$.

With this covariant term, the Higgs Lagrangian density of the system is obtained:
\begin{equation*}\label{eq:chap1:HiggsMechanism:HiggsLagrangianB}
\begin{split}
	\mathcal{L}_{Higgs} 	&= \frac{1}{2} (\partial_{\mu}h) (\partial^{\mu}h) + \frac{g}{4}(v+h)^{4}\PW_{\mu}\PW^{\mu} + \frac{g^{2}}{8 \textrm{cos}^{2}\theta_{W}}(v+h)^{2}\PZ_{\mu}\PZ^{\mu} \\
					&+\frac{\mu^{2}}{2}(v+h)^{2} - \frac{\lambda}{16}(v+h)^{4}
\end{split}
\end{equation*}
and expressing it in terms of the boson masses and coupling parameters, it can be written as:
\begin{equation}\label{eq:chap1:HiggsMechanism:HiggsLagrangianC}
\begin{split}
	\mathcal{L}_{Higgs} 	&= \frac{1}{2} (\partial_{\mu}h) (\partial^{\mu}h)   - \frac{1}{2}m^{2}_{\PHiggs}h^{2}+\frac{1}{2}m_{\PW}\PW_{\mu}\PW^{\mu}+\frac{1}{2}m_{\PZ}\PZ_{\mu}\PZ^{\mu} + g m_{\PW} h \PW_{\mu}\PW^{\mu} \\
					&+ \frac{g^{2}}{4}\PW_{\mu}\PW^{\mu} + g \frac{m_{\PZ}}{2 \textrm{cos}\,\theta_{W}} h \PZ_{\mu}\PZ^{\mu} - g^{2}\frac{1}{4 \textrm{cos}^{2}\theta_{W}}h^{2}\PZ_{\mu}\PZ^{\mu} 
					 -g \frac{m^{2}_{H}}{4 m_{\PW}}h^{3} \\
					& -g^{2}\frac{m^{2}_{H}}{32 m^{2}_{\PW}}h^{4}+\textrm{const.}
\end{split}
\end{equation}
As can be seen in the Lagrangian \ref{eq:chap1:HiggsMechanism:HiggsLagrangianC}, the coupling strengths of the \PW and \PZ fields to the Higgs are proportional to $m_{\PW}$ and $m_{\PZ}$ respectively.
%This why it is said that the more a fundamental particle interacts with the Higgs boson, the more massive it is.
% It is not that the particles that have more mass interact more with the Higgs fields but the 
% other way round: the particles that interact more with the Higgs fields have more mass, i.e. the interaction with the Higgs proceeds the mass of the particles.

%Another way to interpret this is by understanding the mass as the tendency to resist movement, therefore, if a particle interacts strongly with the Higgs field, it is more difficult for that particle to 
%move and hence this particle is more massive. The Higgs field acts as a viscosity. 

\paragraph{The Higgs mechanism in the SM - Fermions}\mbox{}\\ %Source: Book - Modern Particle Physics - Mark Thompson (me lo prestó Stef)
The Higgs mechanism for spontaneous symmetry braking of the $SU(2)_{L} \times U(1)_{Y}$ gauge group of the SM generates the masses of the \PWpm and \PZ bosons.
For originating the mass of the fermions without violating the EW gauge symmetry a similar procedure is carried but taking into account that the left-handed particles transform different
than the right-handed. To do so, additional terms including the Yukawa couplings are added into the Lagrangian. These terms are of the form:
\begin{align*}
-y_{f}(\bar{\chi}_{L}^f \Phi \chi_{R}^{f} + \bar{\chi}_{R}^f \Phi^{\dagger} \chi_{L}^{f} )\, ,
\end{align*}
where the $f$  superindex runs over all quarks and charged leptons. It is usual to express the second part of the summ just as ``plus hermitic conjugate'' (``+ h.c.''). Note that the hermitic conjugate part
is necessary to ensure that expression fulfils the requirement for a hermitian operator to be self-adjoint in a complex Hilbert space.
The different $y_f$ constants are known as Yukawa couplings of the particle $f$ to the Higgs field. The Higgs doublet is denoted by $\Phi$.
For the electron $SU(2)$ doublet, the element with this coupling can be written as:
\begin{equation}\label{eq:chap1:HiggsMechanism:ElectronDoubletLagrangian}
\mathcal{L}_{\Pe} = - y_{\Pe} \left[ (\bar{\Pnu}_{\Pe} \bar{\Pe})_{L} \begin{pmatrix}  \phi^{+} \\ \phi^{0}  \end{pmatrix} \Pe_{R} +  \bar{\Pe_{R}}  (\phi^{+*} \phi^{0*})\begin{pmatrix}  \Pnu_{\Pe} \\ \Pe  \end{pmatrix}_{L} \right] \, .
\end{equation}
Here, $y_{e}$ is the Yukawa coupling of the electron to the Higgs boson. 
After spontaneously breaking the symmetry as it is done 
in eq. \ref{eq:chap1:HiggsMechanism:SymmetryBreaking}, the Lagrangian in \ref{eq:chap1:HiggsMechanism:ElectronDoubletLagrangian} becomes:
\begin{equation}\label{eq:chap1:HiggsMechanism:ElectronBroken}
\mathcal{L}_{\Pe} = \frac{-y_{\Pe}}{\sqrt{2}} v (\bar{\Pe}_{L}\Pe_{R} + \bar{\Pe}_{R}\Pe_{L}) +  \frac{-y_{\Pe}}{\sqrt{2}} h (\bar{\Pe}_{L}\Pe_{R} + \bar{\Pe}_{R}\Pe_{L} )
\end{equation}
The $y_{\Pe}$ is not predicted by the Higgs mechanism, but can be chosen to be consistent with the observed electron mass ($m_{\Pe}$) so that $y_{\Pe}=\sqrt{2}\,m_{\Pe}/v$. Using this relation, the Lagrangian in \ref{eq:chap1:HiggsMechanism:ElectronBroken} becomes:
\begin{equation}\label{eq:chap1:HiggsMechanism:ElectronMassL}
\mathcal{L}_{\Pe} = -m_{\Pe} \bar{\Pe}\Pe - \frac{m_{\Pe}}{v}\bar{\Pe}\Pe h
\end{equation}
The first element of the Lagrangian in \ref{eq:chap1:HiggsMechanism:ElectronMassL} gives mass to the electron and gives rise to the coupling of the electron to the Higgs fields in its non-zero vacuum expectation. 
The second term represents the coupling of the electron and the Higgs boson itself.

The non-zero vacuum expectation value occurs only in the neutral part of the Higgs doublet (the lower in $\Phi = \begin{pmatrix}  \phi^+ \\ \phi_0  \end{pmatrix} $ ) due to the form 
in the ground state in \ref{eq:chap1:HiggsMechanism:SymmetryBreaking}. This implies that the combination $\bar{\chi}_{L}^f \Phi \chi_{R}^{f} + \bar{\chi}_{R}^f \Phi^{\dagger} \chi_{L}^{f}$ can only generate
masses  for the fermions in the lower component of an $SU(2)$ doublet, i.e. the charged leptons and the down type quarks. Putting aside the procedure to give mass to the up-type quarks,
this explains why the neutrinos do not get mass through the Higgs mechanism.

For the up-type quarks, a gauge invariant term can be constructed from $\bar{\chi}_{L}^f \Phi_{c} \chi_{R}^{f} + \bar{\chi}_{R}^f \Phi_{c}^{\dagger} \chi_{L}^{f}$:
\begin{equation*}%\label{eq:chap1:HiggsMechanism:UpType}
\mathcal{L}_{\Pup} = y_{\Pup} (\bar{\Pup} \bar{\Pdown})_{L}  \begin{pmatrix}  -\phi^{0*} \\ \phi^{-}  \end{pmatrix} \Pup_{R} + \textrm{h.c.}
\end{equation*}
Applying the symmetry breaking:
\begin{equation*}\label{eq:chap1:HiggsMechanism:UpTypeBroken}
\mathcal{L}_{\Pup} =  \frac{-y_{\Pup}}{\sqrt{2}} v (\bar{\Pup}_{L}\Pup_{R} + \bar{\Pup}_{R}\Pup_{L}) +  \frac{-y_{\Pup}}{\sqrt{2}} h (\bar{\Pup}_{L}\Pup_{R} + \bar{\Pup}_{R}\Pup_{L} )
\end{equation*}
with a Yukawa coupling between the up quark and the boson $y_{\Pup}=\sqrt{2}\,m_{\Pup}/v$, resulting in:
\begin{equation*}%\label{eq:chap1:HiggsMechanism:ElectronMassL}
\mathcal{L}_{\Pup} = -m_{\Pup} \bar{\Pup}\Pup - \frac{m_{\Pup}}{v}\bar{\Pup}\Pup h .
\end{equation*}

Therefore, for Dirac fermions, mass terms that let the Lagrangian invariant under local gauge transformations can be constructed from
\begin{align*}%\label{eq:chap1:HiggsMechanism:ForFermions}
\mathcal{L} = -y_{f}\left[\bar{\chi}_{L}^f \Phi \chi_{R}^{f} + (\bar{\chi}_{R}^f \Phi \chi_{L}^{f} )^{\dagger} \right] && \textrm{ or } && \mathcal{L} = y_{f}\left[\bar{\chi}_{L}^f \Phi_{c}\chi_{R}^{f} + (\bar{\chi}_{R}^f \Phi_{c} \chi_{L}^{f} )^{\dagger}\right] .
\end{align*}
The left Lagrangian is used for the leptons and down-type quarks, while the right one is used for the up-tupe quarks. These elements give rise not only to the mass of the fermions but also to the interaction
strengths between these fermions and the Higgs boson. The Yukawa coupling of the fermions to the Higgs field is given by:
\begin{equation}\label{eq:chap1:HiggsMechanism:YukawaCoupling}
	y_{f} = \sqrt{2} \, \frac{m_{f}}{v} \, ,
\end{equation}
where the Higgs vacuum expectation value is fixed by the Fermi coupling $G_{F}$ and is measured to be $v = \sqrt{2}\,G_{F} \approx 246.22\,$GeV. The $G_{F}$  
is measured from the $\APmuon$ lifetime \cite{MuLan:2010shf}. The $G_{F}$  is also used to determine the magnitude of the elements in the CKM matrix.

The value of fermionic masses is not predicted by the SM but obtained trough experimental observations.
Given the $\mtop = 172.76 \pm 0.30\,$GeV, it is of particular interest the Yukawa coupling of the top quark to the Higgs field, \yt, which is almost exactly equal to one.
 It is important to verify this because deviation of the measured \yt from the SM prediction would be a proof of new physics.
 %phenomena beyond the SM that could provide an 
%answer to several open questions concerning the fundamental interactions of elementary particles.
%For the top quark, the Yukawa coupling, $y_t$ almos exactly the unity. 


%https://www.youtube.com/watch?v=2TYFNOArnYo&t=2006s 




%%%%%%%%%%%%%%%%%%%%%
%                  Charge Parity                    %
%%%%%%%%%%%%%%%%%%%%%
%\subsection{Charge-Parity}
%\label{sec:chap1:CP_Violation}
% https://www.damtp.cam.ac.uk/user/tong/qft/four.pdf   <- p. 93-95 or PT and 96-96 C
% Parity and Charge conjugation: https://ocw.mit.edu/courses/physics/8-323-relativistic-quantum-field-theory-i-spring-2008/lecture-notes/ft1ls06p_08.pdf
% Intro of this paper: https://arxiv.org/pdf/2201.02385.pdf
% ATLAS: https://atlas.cern/updates/briefing/symmetry-breaking-higgs-boson <- Searching for new sources of matter–antimatter symmetry breaking in Higgs boson interaction with top quarks

%\pablo{Probably, this section is gonna be absorbed by \ref{sec:chap1:EW}}


%%%%%%%%%%%%%%%%%%%%%
%                        Wrap up                        %
%%%%%%%%%%%%%%%%%%%%%
\subsection{Wrap up}
\label{sec:chap1:Wrapup}
%Once the gauge symmetries and the fields with their gauge quantum numbers are specified, the Lagrangian of
%the SM is fixed by requiring it to be invariant under local gauge transformations, and renormalisable. 
%The SM Lagrangian can be divided into several pieces.

Perhaps the ultimate and definitive (if talking about definitive makes any sense) theory of particle physics 
is a simple equation with a small number of free parameters. Meanwhile, the SM is here, and while it is not 
the ultimate theory, it is unquestionably one of modern physics' greatest successes.
Despite its achievements, many questions remain unsolved.

\subsubsection{The parameters of the Standard Model}
\label{sec:chap1:Wrapup:ParamsOfSM}
The SM contains 25 free parameters that must be determined through observation and experimentation. 
These are the masses of the twelve fermions (assuming color variations and antiparticles are not viewed as 
separate fermions) or, more precisely, the twelve Yukawa couplings to the Higgs field ($m_{\nu_{1}}, m_{\nu_{2}}, m_{\nu_{3}}, m_{e}, m_{\mu}, m_{\tau},
	m_{u}, m_{d}, m_{c}, m_{s}, m_{t} \textrm{ and }  m_{b}$)
	
%\begin{equation*}
%	m_{\nu_{1}}, m_{\nu_{2}}, m_{\nu_{3}}, m_{e}, m_{\mu}, m_{\tau},
%	m_{u}, m_{d}, m_{c}, m_{s}, m_{t} \textrm{ and }  m_{b}
%\end{equation*} 
The three coupling constants of describing the strength of the gauge interactions ($g, g' \textrm{ and } g_{s}$)
%\begin{equation*}
%	g, g' \textrm{ and } g_{s}
%\end{equation*} 
 and the two parameters describing the Higgs potential ($\mu$ and $\lambda$) or, equivalently, its vacuum  
expectation value $v$ and the Higgs mass $m_{h}$.
%\begin{equation*}
%	v \textrm{ and }m_{h}
%\end{equation*} 
The three mixing angles and the complex phase of the CKM matrix and the four of the PMNS matrix ($\theta_{12}, \theta_{13}, \theta_{23}, \rho_{13}, \theta'_{a}, \theta'_{b}, \theta'_{c} \textrm{ and }\theta'_{d}$), which mixing of neutrino-mass eigenstates with neutrino-falvour eigenstates).
%\begin{equation*}
%	\theta_{12}, \theta_{13}, \theta_{23}, \rho_{13}, \theta'_{a}, \theta'_{b}, \theta'_{c} \textrm{ and }\theta'_{d}
%\end{equation*} 

From the 25 free parameters of the SM, 14 are associated to the Higgs field, eight with the flavour sector and
only three with the gauge interactions.


%%%%%%%%%%%%%%%%%%%%%%%%
%                  Problems with the SM                  %
%%%%%%%%%%%%%%%%%%%%%%%%
\subsubsection{Problems with the Standard Model}
\label{sec:chap1:SM_problems}
While the SM is very good theory that has passed rigorous testing, this is not the ending of the story, 
there are several limitations of the SM and a variety phenomena that it does not explain. The SM 
does not cover all questions in the universe and, hence, physicist continue looking for better theories 
to explain more.  There is a long list of  small and minor issues with the SM in the following pages only 
the most relevant ones are described.

\paragraph{Gravity}\mbox{}\\
Gravity is the first force that any person learns about and the one known by the humankind for the most time. 
The SM describes all the other fundamental interactions but this one. In the Table \ref{tab:Chap1:FundamentalInteractions},
the four forces are presented along with the theories to describe them. While QCD, QED and EW interactions are part of the
SM, the GR is not. GR is a geometric theory that currently describes the gravitation in modern physics.
Some of the suggested solutions to integrate gravitational interactions in the SM consist in
postulating a new force carrier particle, the ``graviton'', that mediates this interaction in a similar way to how the gauge bosons
were proposed. Other explanations state that the gravity can only be described by a deeper theory in which the 
time-space structure is not flat like it is in the SM but dynamic. 

\paragraph{Neutrino masses}\mbox{}\\
According to the SM the neutrinos are massless, nevertheless, many experiments confirm that this is not 
true \cite{KATRIN:2021uub}. This is due to a property of neutrinos that allows them to change their flavour 
while traveling through the space, this feature is known as ``neutrino oscillations''. Each of the three neutrino flavours
(\Pnue, \Pnum, \Pnut)
is a linear combination of three discrete neutrino-mass eigenstates ($\Pnu_{i}$ with $i\in\{1,2,3\}$) with mass eigenvalues 
($m_i$). While the neutrino oscillation experiments could probe the squared neutrino-mass eigenvalues 
($\Delta m^{2}_{ij}$), both the total scale of the masses and the sign of $\Delta m_{ij}$ remains as some the most 
relevant open questions in particle physics. Regarding to the sign of $\Delta m_{ij}$, it is known that the mass of 
$\Pnu_{2}$ is slightly higher than $\Pnu_{1}$ ($\Delta m^{2}_{21}\equiv m^{2}_{2} - m^{2}_{1} \sim 10^{-4}\,$eV) but 
for the third mass eigenstate it has not been measured yet whether it is greater (normal ordering) or lower (inverted 
ordering) than the other two, as it is depicted in Figure \ref{fig:Chap1:Neutrino_problem}. Nevertheless, 
the absolute square difference is known ($\Delta m^{2}_{31}\equiv |m^{2}_{3} - m^{2}_{1}| \sim 10^{-3}\,$eV). 


\begin{figure}
    \centering
    \includegraphics[width = 0.65\textwidth]{Chapter1/Neutrino_problem}
    \caption{Two potential mass orderings of neutrinos are the normal ordering (normal hierarchy) and the inverted ordering (inverted hierarchy).}
    \label{fig:Chap1:Neutrino_problem}
\end{figure}

Non-zero neutrino masses opened an interesting portal on beyond SM physics and, even though neutrinos are very elusive when it comes to detect them, some 
next-generation experiments such as Dune are very promising when it comes to set competitive and model independent limits on neutrino masses. 

Regarding to the nature of this mass, one could add mass terms to the SM as it is done in Section \ref{sec:chap1:ParticleMasses:HiggsMechanism} 
for the up-type quarks but the origin of the neutrino masses is still not known. 
% It is possible that this mass comes from the Higgs mechanism, however, this is not clear.
Also, if neutrinos gained mass through Yukawa interaction, it would imply the presence of right-handed neutrinos, 
which has not been observed.

% Sterile neutrinos: special kind of neutrino that has been proposed to explain some unexpected experimental results, but they have not been definitively discovered

% -  Neutrinos are the most elusive particles of the SM due to the difficulty of detect hem. However, some next generation experiments such 
% as Dune (Liquid argon detector) are very promising when it comes to set competitive and model independent limits on neutrino masses


\paragraph{Matter-antimmater asymmetry}\mbox{}\\
In principle, the Big Bang should have produced an equal amount of matter and antimatter 
which would all have then annihilated, leaving behind an empty Universe filled with EM radiation.
 However, everything we see now is essentially totally constituted of matter, from the tiniest life forms 
 on Earth to the greatest celestial objects. In comparison, there isn't a lot of antimatter around. 

By looking at the CMB radiation, which contain the residual photons of the Big Ban, researchers 
have determined that there was a symmetry between the matter and antimatter content in the early universe. 
For every $3 \times 10^{9}$ antimatter particles, there were $3 \times 10^{9}$ and 1 matter particles.
The matter and antimatter annihilate and produced the CMB and the remaining 1 part turned into all the 
stars and galaxies that are seen.  The field of cosmology that studies the processes that produced an 
asymmetry between leptons and antileptons in the very early universe is called leptogenesis.

Researches carried during the last few decades hace revealed that the las of nature do no equally apply to
matter and antimatter. So far, the only non-trivial difference between matter and antimatter found is the \CP asymmetry
(or \CP violation, which has been introduced in Section \ref{sec:chap1:EW}). 
Alas, the quantity of \CP asymmetry included in the SM is insufficient to explain the composition of the observable
universe and, hence, extensive searches of new sources of \CP violation are being carried.

In this context, the studies described in this thesis are part of the seek of new \CP-violation sources. As Section
\ref{sec:Chap1:tHq} details, the observation of a cross section\footnote{The definition of cross section can be found in Section \ref{sec:Chap2:LHC:lumi}.}
 greater than the one predicted by the SM would imply
a that Higgs-single-top-quark associated production does not conserve \CP.


%Where is the antimatter? One could think about antimatter galaxies, in the end, matter and antimatter emit the same kind of radiation. But, since the 
%antimatter galaxy would be surrounded by anti-hydrogen halo, when the halo of a matter galaxy and the halo of an antimatter galaxy enter in contact,
%the electrons and positrons of the halo would annihilate each other producing gamma rays. This gamma rays have not been observed and, hence,
%the existence of antimatter galaxies is desecrated.




%\paragraph{Strong \CP problem}\mbox{}\\ %https://cds.cern.ch/record/2652742/files/CERN-THESIS-2018-297.pdf
%https://inspirehep.net/literature/2089044
%\pablo{Write some lines here}

\paragraph{Dark energy}\mbox{}\\
According to cosmological observations, the matter described by the SM only makes up around 5\% of the universe.
It turns out that roughly 68\% of the universe is dark energy, which is not considered by the SM.
\begin{wrapfigure}{R}{0.5\textwidth}
\includegraphics[width = 0.4\textwidth]{Chapter1/Limitations-UniverseExpansion}
	\caption{The universe’s expansion over time. The dark-energy existence has been suggested to explain this expansion.} %Credit: NASA/WMAP Science Team/ Art by Dana Berry
	\label{fig:Chap1:UniverseExpansion}
\end{wrapfigure} 

Dark energy is an unknown type of energy postulated to explain the observed accelerated expansion of the universe as Figure \ref{fig:Chap1:UniverseExpansion}.
This expansion is dominated by a spatially smooth component with negative pressure called dark energy.  
Modern cosmological measurements are based in supernovae, cosmic microwave background fluctuations,
galaxy clustering and weak gravitational lensing, and methods agree with a spatially flat universe with about 30\% matter (visible
and dark) and 70\% dark energy \cite{Planck:2018vyg}.
%source:https://inspirehep.net/files/623a507c0d34d855e16642fb216668ab



\paragraph{Dark matter}\mbox{}\\
The rest of the energy content in the universe is the matter. 
Dark matter (DM) adds up for approximately 85\% of all matter and 27\% of all energy. 
This matter is called dark because it does not interact with the electromagnetic field, so maybe a name such us invisible matter would have been more appropriate
 since rather than being dark it just does not emit or reflect light.
The only way to interact with DM is via gravitational interaction, which is bout 25 orders of
magnitud weaker than the weak force (as Table \ref{tab:Chap1:FundamentalInteractions} shows). This is why DM is so difficult to detect. 
The SM does not provide a proper explanation but searches are being carrie and candidates such as 
weakly interacting massive particles (WIMPs) or axions\footnote{An axion is a hypothetical elementary particle postulated to resolve the strong CP problem \cite{Weinberg:1977ma} \cite{Wilczek:1977pj}.} have been proposed.

The existence of DM has been inferred through gravitational effects in astrophysical and cosmological observations. 
The rotational speed of the galaxies \cite{Rubin:1970zza}, the gravitational lensing \cite{Taylor:1998uk} and the CMB angular spectrum \cite{Planck:2015fie} are
some examples of phenomena that cannot be explained with general relativity unless there is more present matter what it is seen.

Although the vast majority of scientific community accepts dark matter existence, alternative explanations for the observed phenomena have suggested. 
Most of these model consists in modifications of GR.
The search of DM at particle colliders, which is focussed on large missing transverse energy signatures, have not result in any observation. Nevertheless,
the existence of a particle is never discarded, only its presence within the detector sensitivity limits.

%\paragraph{Naturalness}\mbox{}\\

%\paragraph{Hierarchy problem}\mbox{}\\

\begin{comment}
\paragraph{Unification of the strong interaction}\mbox{}\\
%\paragraph{Grand Unified Theories}\mbox{}\\
%\subsubsection{Grand Unified Theories}\mbox{}\\
There are unification attempts to treat all interactions as one, with the same coupling constant 
and the same symmetry group\footnote{The most popular symmetry group for unification is $SU(5)$.}
In the same way the EW unifies QED and Weak forces, the grand unified theories (GUT) unifies 
all three interactions (QED, Weak and QCD) of the SM at high energies, where the coupling constants 
approach each other. 
Note that gravity is still left out, because it is much weaker than the other interactions. 


 
%\subsection{Beyond the Standard Model}
\paragraph{Supersymmetry}\mbox{}\\
Originally motivated by the hierarchy problem, supersymmetry (SUSY) is an extension to the SM.
In SUSY the equations for force and the equations for matter are identical and each SM particle has its supersymmetric partner ``sparticle'' from which
differs by half spin unit. Therefore, for each SM fermion, the corresponding sfermion is a spin-0 scalar and hence a boson. Identically, 
there is a super-partner for each of the SM bosons. The gluons have the spin-half gauginos. The Higgs have a weak isospin doublet os spin-half Higgsinos ($\tilde{\PHiggs}_{1,2}^{0}$ and $\tilde{\PHiggs}^{\pm}$).
The new particles interact through the forces of the SM but would have different masses

SUSY is not a theory but a principle and any theory with that property is said to be supersymmetric. So, there is not one but dozens of supersymmetric theories.
A lot of focus has been put searching for supersymmetric particles but so far the supersymmetric partners have not been found, 
which is a good reason to be skeptical about SUSY. \pablo{maybe cite here some ATLAS SUSY searches}

\end{comment}

%\subsection{Beyond the Standard Model}
%\subsubsection{Extra dimensions)
%\subsubsection{String theory)
%\subsubsection{Composite Higgs modes)
%\subsection{Axions}
%\subsection{Seesaw mechanism for neutrinos}

\paragraph{Others}\mbox{}\\ \pablo{(quizás, esto the "others" sobra ya)}
The different problems mentioned hitherto are just some of the most relevant open questions that fundamental 
physics has not being able to answer jet. Nonetheless, there are many other issues whose discussion 
would need many pages and are outside the scope of this work. Even so, it won't harm to list a few of them:
\begin{itemize}
	\item Hierarchy problem: It is caused by the enormous distance between two fundamental physics scales: 
		the EW scale ($\sim10^2\,$GeV) and the Planck scale ($	\sim10^{19}\,$GeV). 
	\item Strong \CP problem: It refers to the fact that, while QCD does not explicitly prohibit \CP 
		violation in strong interactions, it has yet to be observed in experiments. 
	\item Naturalness:  It is the property that the dimensionless ratios between free parameters or physical constants 
		appearing in a physical theory should take values of order unity.  By looking at the parameters of the SM 
		described in Section \ref{sec:chap1:Wrapup:ParamsOfSM}, it can be seen that the naturalness pricinciple
		is not satisfied. For instance, the masses of the first generation of fermions are in the range of $1\,$MeV 
		while the top quark has a mass of 172-173$\,$GeV.
		Though this is not a flaw in the theory itself, it is frequently seen as a sign of undiscovered principles hidden 
		behind a more comprehensive theory.
	\item Composite Higgs models: 
	%\item Seesaw mechanism for neutrinos:
	\item Majorana neutrinos: It is not clear yet is neutrinos are Majorana particles, i.e. they are their own 
		antiparticles (\Pnu = \APnu = $\Pnu_{M}$).
		Current experiments trying to solve this question are focused on neutrinoless double-$\beta$ decay,
		which can occur only if neutrinos ara Majorana particles. 
	\begin{comment}
	\item String theory: It is theoretical framework in which fundamental point-like particles are understood as
		 as vibrational states of a more basic object, the so called ``string''. A string is a one-dimensional 
		 entity that can be either be open (forming a segment with to endpoints) or close (forming a loop)
		 and may have other special properties.  Despite being in development since the late 1970s, 
		 it has not been accepted nor discarded yet. 
	\end{comment}
\end{itemize}

%Closing words
Most of theoretical concepts of the SM where in place by the end of the 1960s. With the discovery of the \PW \cite{UA2:1983tsx_Wpm} and
\PZ \cite{UA1:1983mne_z0} bosons at CERN in the mid 1980s and the Higgs boson in 2012, the SM has established itself as one of the major
pillars of modern physics.  The understanding of the universe at the most fundamental level is based in this theory,
which has been tried to be summarised through the entire Section \ref{sec:chap1:TheSM}.

Despite its brilliance and success, the SM is not the ending of the story. As exposed above, there are far too many 
unanswered questions and loose ends. The HL-LHC \cite{ZurbanoFernandez:2020cco} and the 
next generation of experiments will look for evidence of physics outside the SM in the next years.

Among the open questions, unresolved concerns and measurements to be completed, this research is focused on the top 
quark\footnote{Here and in the following, the usage of the term top quark includes the top antiquark.}.
On one hand, contributions to the measurement of the polarisation of this quark are presented and, on the other hand,
the study of the associated production of a single-top quark with a Higgs boson is present as well. 
Now that the basics of the SM have been settled, in the sections to come, the context of these two topics
is being discussed. 




%%%%%%%%%%%%%%%%%%%%%%%%%%%%%%%%%%%%%%%%%%%%%%%%%%%%%%%
%%%%%%%%%%%%%%%%%%%%%%%%%%%%%%%%%%%%%%%%%%%%%%%%%%%%%%%
%%%%%%%%%%%%%%%%%%%%%%%%%%%%%%%%%%%%%%%%%%%%%%%%%%%%%%%
%%%%%%%%%%%%%%%%%%%%%%%%%%%%%%%%%%%%%%%%%%%%%%%%%%%%%%%
%%%%%%%%%%%%%%%%%%%%%%%%%%%%%%%%%%%%%%%%%%%%%%%%%%%%%%%
%%%%%%%%%%%%%%%%%%%%%%%%%%%%%%%%%%%%%%%%%%%%%%%%%%%%%%%
%%%%%%%%%%%%%%%%%%%%%%%%%%%%%%%%%%%%%%%%%%%%%%%%%%%%%%%
%%%%%%%%%%%%%%%%%%%%%%%%%%%%%%%%%%%%%%%%%%%%%%%%%%%%%%%
%%%%%%%%%%%%%%%%%%%%%%%%%%%%%%%%%%%%%%%%%%%%%%%%%%%%%%%
%%%%%%%%%%%%%%%%%%%%%%%%%%%%%%%%%%%%%%%%%%%%%%%%%%%%%%%
%%%%%%%%%%%%%%%%%%%%%%%%%%%%%%%%%%%%%%%%%%%%%%%%%%%%%%%
%%%%%%%%%%%%%%%%%%%%%%%%%%%%%%%%%%%%%%%%%%%%%%%%%%%%%%%
%%%%%%%%%%%%%%%%%%%%%%%%%%%%%%%%%%%%%%%%%%%%%%%%%%%%%%%
%%%%%%%%%%%%%%%%%%%%%%%%%%%%%%%%%%%%%%%%%%%%%%%%%%%%%%%
%%%%%%%%%%%%%%%%%%%%%%%%%%%%%%%%%%%%%%%%%%%%%%%%%%%%%%%
%%%%%%%%%%%%%%%%%%%%%%%%%%%%%%%%%%%%%%%%%%%%%%%%%%%%%%%







\chapter{Physics of the top quark and the Higgs boson}
\label{chap:topHiggsPhysics}

\vspace*{0.1 cm} 
\hspace*{200pt} \\
\hspace*{120pt} \textit{Magisches Theater} \\
\hspace*{120pt} \textit{Eintritt nicht für jedermann} \\
\hspace*{120pt} \textit{Nur für Verrückte!} \\
\hspace*{140pt} ---\textsc{Herman Hesse } \\% \textit{} \\
\hspace*{165pt}     \textsc{Der Steppenwolf (1927)} \\% \textit{} \\
\vspace*{2cm} 


%  Top quark physics at hadron colliders: https://citeseerx.ist.psu.edu/viewdoc/download?doi=10.1.1.205.6486&rep=rep1&type=pdf
%%%%%%%%%%%%%%%%%%%%%
%                 Top quark physics               % Why should we care about the top quark Yukawa coupling?
%%%%%%%%%%%%%%%%%%%%% https://link.springer.com/article/10.1134/S1063776115030152
\section{Top quark}
\label{sec:Chap1:Top}
The top
quark (\Ptop) or, for simplicity, just top is the up-type quark of the third generation of fermions.
Its most distinctive feature is its huge mass, which is the largest among all %Sometimes called truth quark, 
fundamental particle particles. 
The left-handed top is the $Q=2/3$ and $T_{3} = +1/2$ member of the weak isospin
doublet that also contains the bottom quark. The right-handed top quark is the $SU(2)_{L}$
weak isospin singlet ($Q=2/3$ and $T_{3} = 0$). %$T_{3}$ is the third component of the weak isospin.
Its phenomenology is driven by its large mass. 
The top quark is often regarded  as a window for new physics
since it provides a unique laboratory where to test the understanding of the SM. 

Due to being so massive, its life time is very short ($\tau_{\Ptop} = 5 \times 10^{-25}\,$s \cite{Taylor:1998uk}). 
Actually, it is shorter than the hadronisation timescale ($1/\Lambda_{QCD} \sim 10^{-24}\,$s).
This represents a exceptional opportunity to study quarks in free state, something that is quite exceptional
due to colour confinement, as explained  in Section \ref{sec:chap1:QCD}. 
In fact, the top quark is the only quark that can be investigated unbonded.
Its lifetime is also smaller than the spin decorrelation timescale ($\mtop /\Lambda^{2}_{QCD} \sim 10^{-21}\,$s$\,$\cite{Mahlon:2010gw}),
implying that the top-quark states conserve their spin state from its production to its decay.
Thanks to this, the top-quark properties, such as the spin information, can be accessed through its decay 
products and, consequently, be measured. %This is the base to study its polarisation, a work that is
%contextualised in Section \ref{sec:Chap1:Top:Polarisation} and described in Chapter \ref{chap:Polarisation}).


Another consequence of its large mass is that  the top quark is the only quark with a Yukawa 
coupling to the Higgs boson (\yt) of the order of one; hence, a thorough understanding of its 
properties (mass, couplings, decay branching ratios, production cross-section, etc.) can reveal
crucial information on basic interactions at the electroweak symmetry-breaking 
scale and beyond. The main objective of this thesis is, precisely, the study of the top quark and Higgs
boson interplay to, ultimately, help to determine if the \yt is that predicted by the SM or there is some \CP-violating
phase that would affect the sign of the Higgs-top Yukawa coupling. The theoretical base for the understanding
the associated production of a top quark and a Higgs boson given in Section \ref{sec:Chap1:top-Higgs} and 
the analysis investigating this matter is presented through the rest of the thesis.%in Chapter \ref{chap:Analysis_tH}.


% Interesting resources:
%.   - https://twiki.cern.ch/twiki/bin/view/AtlasPublic/TopPublicResults
%.   - https://pdg.lbl.gov/2020/reviews/rpp2020-rev-top-quark.pdf
%.   - https://iopscience.iop.org/article/10.1088/1361-6471/44/6/063001/pdf  %Top-quark physics at the Large Hadron Collider
%.   - http://cds.cern.ch/record/2799670/files/ATL-PHYS-PROC-2022-006.pdf % Highlights of top easurements with ATLAS (2022)


%%%%%%%%%%%%%%%%%%%%%
%                Top-quark discovery             %
%%%%%%%%%%%%%%%%%%%%%
\subsection{Top-quark discovery}
In 1973, Kobayashi and Maskawa postulated the possibility of a third generation of quarks to explain \CP
violations in kaon decays \cite{Kobayashi:1973fv}. To match the names of the up and down quarks, the new
generation's quarks were given the names top and bottom.
The GIM\footnote{Standing for Glashow–Iliopoulos–Maiani, it is the mechanism to describe the flavour-changing neutral currents. }
 mechanism\cite{Glashow:1970gm}, which predicted the existence of the yet-to-be-discovered charm quark, was used to 
make this prediction.
When the charm was observed \cite{SLAC-SP-017:1974ind}, the GIM was integrated into the SM
and the postulation of the third family, and thus the top quark, gained acceptance. 
Shortly after the charm, the bottom quark was discovered in the E288 experiment 
at Fermilab \cite{Herb:1977ek}, reinforcing the idea of the existence of the top quark.
However, due to its large mass, it took 18 years to confirm the existence of the top.


The top quark was observed for the first time at Tevatron with the CDF \cite{CDF:1995wbb} and D$\emptyset$ \cite{D0:1995jca} detectors via flavour-conservating strong interaction in 1995. 
%in 2009 via single-top quatk production in 2009 through charged current EW proccess
Back then and until the start of LHC Run$\,$1, Tevatron was the only accelerator powerful enough to produce top quarks. % \CM Tevatron 0.98 TeV


%%%%%%%%%%%%%%%%%%%%%
%                 Top quark mass                   %
%%%%%%%%%%%%%%%%%%%%%
\paragraph{Top quark mass}\mbox{}
%-> Latest results (june 2022) https://cds.cern.ch/record/2811385 \cite{JimenezPena:2811385}
%-> Top quark mass defines the stability of the EW vacuum. It is a key factor to test the 
%internal consistency of the SM.

As discussed in Section \ref{sec:chap1:Wrapup:ParamsOfSM}, \mtop is a free parameter 
of the SM. The theory does not predict its value, hence it must be determined experimentally. 
Note that what experiments measure is not the SM \mtop but the either the pole mass $\mtop^{\text{pole}}$ or
the MC top-quark mass $\mtop^{MC}$. It is expected that the difference between the $\mtop^{MC}$ definition 
and the $\mtop^{pole}$ is of order $1\,$GeV. % https://arxiv.org/pdf/1710.06019.pdf
To derive the \mtop from hadron collision data, two approaches are explored:
\begin{itemize}
	\item Direct measurements (also known as template methods)$\,$\cite{Amoroso:2746800}:
	 The $\mtop^{MC}$ is determined by reconstructing
	 (fully or partially) the decay products of one or more top quarks in a \ttbar or single-top event\footnote{In 
	 particle physics, an event is the result of a collision.}. A comparison of
	  the detector-level\footnote{ At detector level, the event information is presented as it is 
	  registered by the detector systems after the calibration and reconstruction processes.}
	  distributions with templates created with a MC generator is used to determine $\mtop^{MC}$.
	  Analysing \ttbar events with lepton-plus-jets and dilepton topologies provides the most precise results.
	  %$\rightarrow$ $m_{t}^{MC}$ with $O(480\,$MeV$)$ precision.
	  Figure \ref{fig:Chap1:top:mtop_MC} summarises the measurements of ATLAS 
	  and CMS for $\mtop^{MC}$ from direct-top-quark decay.
	  
	\begin{figure}
    	\centering
    	\includegraphics[width = 0.8\textwidth]{Chapter1/LHC_topMCmass_Nov22}
   	 \caption{Summary of the ATLAS and CMS $\mtop^{MC}$ measurements from top-quark decay. 
	 Results compared to LHC \mtop combination \cite{ATLAS:2022lsz}. 
	 The most precisely studied property of the top quark is its mass.}
    	\label{fig:Chap1:top:mtop_MC}
	\end{figure}

	
       \item Indirect measurements$\,$\cite{Amoroso:2746800}:
	The $\mtop^{\text{pole}}$ is measured from measurements of the cross section. These methods
	rely on the dependence on the value of the $\mtop$ for the total or differential production 
	cross-sections for processes involving top quarks. Figure \ref{fig:Chap1:top:mtop_Pole}
	presents $\mtop^{\text{pole}}$ indirect measurements.
	%$\rightarrow$ $m_{t}^{\text{pole}}$ with $O(1\,$GeV$)$ precision
	\begin{figure}
    	\centering
    	\includegraphics[width = 0.8\textwidth]{Chapter1/LHC_topPOLEmass_Sept19}
    	\caption{Summary of the measurements of the $\mtop^{\text{pole}}$ from \ttbar cross-section measurements.
    	A caparison to the measurements from top-quark decay is provided \cite{ATLAS:2018fwq}.}
    	\label{fig:Chap1:top:mtop_Pole}
	\end{figure}
	\end{itemize} 	
	% ref:  	Strategy for ATLAS top quark mass measurements: 2021-2023
	%         	https://cds.cern.ch/record/2746800?


\pablo{No he explicado qué es  $\mtop^{MC}$ y  $\mtop^{\text{pole}}$. No hace falta, no?}

Among the top quark's properties, its mass is the one that has received the most attention so far.
The most recent studies for the top quark mass measurements result in $\mtop = 172.76 \pm 0.30\,$GeV \cite{pdgTop}. 
This number is an average of the measurements at LHC with ATLAS ($172.69 \pm 66\,$GeV \cite{ATLAS:2018fwq})
 and CMS ($172.6 \pm 3.5\,$GeV at CMS \cite{CMS:2019fak}) 
 and at Tevatron with CDF and D$\emptyset$ (combined result: $174.30 \pm 0.89\,$GeV \cite{CDF:2016vzt}).
These values are measured from the kinematics of  \ttbar events.%\footnote{This \mtop results are sensitive to the 
%top quark mass used in the MC generator that is usually interpreted as the pole mass.}.

 
%Figure \ref{fig:Chap1:top:mtop_tt}  presents the results for \mtop measurements from \ttbar observables.



% Definition of mass: 
% Quark masses are fundamental parameters of QCD.  In QFT, the propagator of a massive fermion has a pole
% at $m_0$ called the pole mass, which also corresponds to the invariant mass calculated from
% its decay products. However, the quarks present a particular situation due to its confinement.
% Most measurements are based on the invariant mass of the top quark decay products.
% The measured top quark mass has been interpreted as the pole mass by the Particle Data Group.
%  For all other quarks, in contrast, the renormalised masses are used.

%The high mass of the top quark indicate its Yukawa coupling to the Higgs field of $\mathcal{O}(1)$.

%\begin{figure}
%    \centering
%    \includegraphics[width = 0.8\textwidth]{Chapter1/LHC_topmassfromXS_dec21}
%    \caption{Summary of the ATLAS and CMS measurements from \ttbar observables. 
%		Results compared to measurements from direct-top-quark decay.}
%    \label{fig:Chap1:top:mtop_tt}
%\end{figure}



% ATLAS		$\mtop = 172.69 \pm 0.25$(stat)$\pm 0.41$(syst) GeV
% CMS		$\mtop = 172.6 \pm 0.4$(stat)$\pm 1.6$(exp)$\pm 1.5$(model)  GeV
% Tevatron	$\mtop = 174.30 \pm 0.35$(stat)$\pm 0.54$(syst) GeV 




%%%%%%%%%%%%%%%%%%%%%%%%%%
%                Top quark production at LHC               %
%%%%%%%%%%%%%%%%%%%%%%%%%%
\subsection{Top quark production at LHC}
\label{sec:Chap1:Top:Production}
The LHC is sometimes referred as a top quark factory due to its ability to produce such particles. 
In this collider, at $\Pproton\Pproton$ collisions, the top quark is mainly produced via
two mechanisms: through QCD in top and anti-top pairs (\ttbar), and by means of the \Wtb 
vertex of EW in single-top quarks associated with other particles.  
Apart from the \ttbar (Section \ref{sec:Chap1:Top:Production:TopPairs}) and single-top
(Section \ref{sec:Chap1:Top:Production:SingleTop}) productions, the associated $\ttbar+X$
and four-top-quark productions (Sections \ref{sec:Chap1:Top:Production:ttbar_plus_X} and 
\ref{sec:Chap1:Top:Production:4tops} respectively) are presented as well.

Since the top quarks often constitute a main background in other physics analysis, 
a better understanding of this particle's properties will directly translate into improvements 
in those searches.

%%%%%%%%%%%%%%%%
%                Top pairs               %
%%%%%%%%%%%%%%%%        %https://twiki.cern.ch/twiki/bin/view/LHCPhysics/TtbarNNLO
\subsubsection{Top pairs}
\label{sec:Chap1:Top:Production:TopPairs}
The production top and anti-top pair of quarks is the largest source of production of top quarks in hadron collisions. This
process is one of the most important at LHC because it allows to precisely study the properties of the top quark. 
Additionally, due to the dominance of this production mode, the top-quark-pair production is also a major background 
in many searches for rare processes. Including the one carried in this thesis, where \ttbar is the main background in
the both of the analysed decay channels (see Section \ref{sec:ChaptH:Bkg}).

For the $\Pproton \APproton$ collisions at Tevatron or $\Pproton \Pproton$ at LHC, the \ttbar production is described by
perturbative QCD.  In this approach, a hard scattering process between the two hadrons is the result 
of an interaction between the quarks and gluons that constitute these hadrons. This model is described with detail in 
Section \ref{sec:Chap2:PhenoOfPP}.


At LHC, the gluon fusion (Figures \ref{fig:Chap1:top:topPairs:FeynmanB1}, 
\ref{fig:Chap1:top:topPairs:FeynmanB2}, \ref{fig:Chap1:top:topPairs:FeynmanB3}) dominates with a 90\% of 
the \ttbar production. It is followed by the quark and antiquark annihilation (Figure 
\ref{fig:Chap1:top:topPairs:FeynmanA}), which accounts for a 10\% of the total top-quark-pair production.
%Due to its primordial importance for the physics programme of LHC
The theoretical calculations for the \ttbar 
production are done to an accuracy of  next-to-next-to-leading order (NNLO)
in QCD and complemented with next-to-next-to-leading logarithmic resummation in reference \cite{Czakon_2020}:
$\sigma^{pred}_{\ttbar} = 832^{+55}_{-64}\,$pb.
\pablo{(En el paper de Czakon te pone la $\sigma^{pred}_{\ttbar}$ a 14 GeV y luego un plot de $\sigma^{pred}_{\ttbar}$ vs $\CM$ pero no da numerito a 23 TeV. Tampoco hay numerito en \cite{ATLAS:2022uqj}. El número lo saco de \cite{ATLAS:2022qak} quien lo saca de )}
ATLAS and CMS have measured the cross section trough different final state channels and its most
recent result are, respectively, $\sigma_{\ttbar} = 836 \pm 29\,$pb \cite{ATLAS:2022qak} and $\sigma_{\ttbar} = 791 \pm 36\,$pb \cite{CMS:2021vhb}.
%Figure \ref{fig:Chap1:top:topPairs:CrossSection} 
%shows the measurements for the \ttbar production cross-section ($\sigma_{\ttbar}$) at
%\CM =$13\,$TeV.
The measurements and the theory calculations are quoted at $\mtop=172.5\,$GeV.
% At Tevatron, gluon fusion was not so important, as they had proton-antiproton collisions . 

%\begin{comment}
%\begin{figure}
%\begin{minipage}[c]{0.74\linewidth}
%\subfloat[\label{fig:Chap1:top:topPairs:FeynmanB1}]
%  {\includegraphics[width=.3\linewidth]{Chapter1/TopQuarkPairsFeynman-B1}}\hfill
%\subfloat[\label{fig:Chap1:top:topPairs:FeynmanB2}]
%  {\includegraphics[width=.3\linewidth]{Chapter1/TopQuarkPairsFeynman-B2}}\hfill
%\subfloat[\label{fig:Chap1:top:topPairs:FeynmanB3}]
%  {\includegraphics[width=.3\linewidth]{Chapter1/TopQuarkPairsFeynman-B3}}\hfill
%\caption{Representative Feynman diagrams of the LO processes contributing to the \ttbar 
%	production through gluon fusion at LHC.}
%\label{fig:Chap1:top:topPairs:FeynmanB}
%\end{minipage}
%\hfill
%\begin{minipage}[c]{0.25\linewidth}
%\includegraphics[width = 0.99\textwidth]{Chapter1/TopQuarkPairsFeynman-A}
%    \caption{LO Feynman diagram for \ttbar production 
%    		via quark and anti-quark annihilation.}
%    \label{fig:Chap1:top:topPairs:FeynmanA}
%\end{minipage}%
%\end{figure}
%\end{comment}

\begin{figure}
\begin{subfigure}[h]{0.23\linewidth}
	\includegraphics[width=\linewidth]{Chapter1/TopQuarkPairsFeynman-B1}
	\caption{}
	\label{fig:Chap1:top:topPairs:FeynmanB1}
\end{subfigure}
\hfill
\begin{subfigure}[h]{0.23\linewidth}
	\includegraphics[width=\linewidth]{Chapter1/TopQuarkPairsFeynman-B2}
	\caption{}
	\label{fig:Chap1:top:topPairs:FeynmanB2}
\end{subfigure}
\hfill
\begin{subfigure}[h]{0.23\linewidth}
	\includegraphics[width=\linewidth]{Chapter1/TopQuarkPairsFeynman-B3}
	\caption{}
	\label{fig:Chap1:top:topPairs:FeynmanB3}
\end{subfigure}
\hfill
\begin{subfigure}[h]{0.25\linewidth}
	\includegraphics[width=\linewidth]{Chapter1/TopQuarkPairsFeynman-A}
	\caption{}
	\label{fig:Chap1:top:topPairs:FeynmanA}
\end{subfigure}%
\caption{Representative Feynman diagrams of the LO processes contributing to the \ttbar 
production. Subfigures (a), (b) and (c) correspond to the production through gluon fusion 
and Subfigure (d) to the production via quark and antiquark annihilation.}
\label{fig:Chap1:top:topPairs:Feynman}
\end{figure}




%\begin{figure}
%\subfloat[\label{fig:Chap1:top:topPairs:FeynmanB1}]
%  {\includegraphics[width=.3\linewidth]{Chapter1/TopQuarkPairsFeynman-B1}}\hfill
%\subfloat[\label{fig:Chap1:top:topPairs:FeynmanB2}]
%  {\includegraphics[width=.3\linewidth]{Chapter1/TopQuarkPairsFeynman-B2}}\hfill
%\subfloat[\label{fig:Chap1:top:topPairs:FeynmanB3}]
%  {\includegraphics[width=.3\linewidth]{Chapter1/TopQuarkPairsFeynman-B3}}\hfill
%\caption{Representative Feynman diagrams of the LO processes contributing to the \ttbar production via gluon fusion at LHC.}
%\label{fig:Chap1:top:topPairs:FeynmanB}
%\end{figure}

%\begin{figure}
%    \centering
%    \includegraphics[width = 0.4\textwidth]{Chapter1/TopQuarkPairsFeynman-A}
%    \caption{Representative Feynman diagrams of the LO processes contributing to 
%    		the \ttbar production at LHC through quark and anti-quark annihilation.}
%    \label{fig:Chap1:top:topPairs:FeynmanA}
%\end{figure}

%\begin{figure}
%    \centering
%    \includegraphics[width = 0.75\textwidth]{Chapter1/tt_xsec_13TeV_nov22}
%    \caption{Summary of measurements $\sigma_{\ttbar}$ at $\CM=13\,$TeV compared to 
%    		the exact NNLO QCD calculation complemented with next-to-next-to-leading logarithmic resummation %\cite{ATLAS:2022uqj}. \pablo{Susana sugiere eliminar estas tablas}}
		%Update from https://twiki.cern.ch/twiki/bin/view/LHCPhysics/LHCTopWGSummaryPlots#Pair_production_cross_section
%    \label{fig:Chap1:top:topPairs:CrossSection}
%\end{figure}

%\begin{figure}
%    \centering
%    \includegraphics[width = 0.75\textwidth]{Chapter1/tt_curve_toplhcwg_sep21}
%    \caption{Summary of LHC and Tevatron measurements of the top-pair production cross-section as a function of the centre-of-mass energy}
    % The measurements and the theory calculation are quoted at mtop=172.5 GeV.
%    \label{fig:Chap1:top:topPairs:CrossSection_vs_CM}
%\end{figure}


%%%%%%%%%%%%%%%%%%%%
%                Single-top quark               %
%%%%%%%%%%%%%%%%%%%%
\subsubsection{Single top}
\label{sec:Chap1:Top:Production:SingleTop}
In addition to the top-quark-antiquark-pair production, the single-top-quark processes are of great 
importance to the study of the top quark properties at the LHC.  
This mechanism has a cross section three times smaller than that of \ttbar and  
it's almost exclusively produced through the EW \Wtb vertex.
This is precisely the reason why single-top-quark production is essential to gather 
information about the \Wtb interaction and to directly measure $|V_{tb}|$ at hadron colliders.
The reason why the single-top quark is produced and decays via a $\Pbottom$-quark
%The reason to both decay to a $\Pbottom$-quark and be produced from a $\Pbottom$-quark 
and not from strange 
or down quarks is because the CKM elements $V_{ts}$ and $V_{td}$ 
are smaller than $V_{tb}$  by several orders of magnitude 
as Table \ref{tab:Chap1:CKM} shows. %The $V_{tb}$ element not only
%determines the top production mechanism but also its decay rate.

%Cross sections for single top production: \url{https://inspirehep.net/files/9be064abc15c44ccc329d82887e6e014}
At LO, there are three production modes for single top, being the \tchannel the dominant mechanism at the LHC 
with, approximately 70\% of  the single top quark cross-section at a $\CM=13\,$TeV.  % ($\sigma_{Single-\Ptop}$)
The other processes are the \schannel and the associated production $\Ptop\PW$ production. Only \tchannel and
$\Ptop\PW$ productions are relevant to the EW single-top production at LHC.%, since the \schannel has not being observed.


%%%     Single-top quark   :::  t-channel
\paragraph{\tchannel}\mbox{}\\
This production mode involves the scattering of a light quark and a gluon from the 
proton sea as shown in Figure \ref{fig:Chap1:top:singletop:tchannel}.
Note that additional diagrams to those in Figure \ref{fig:Chap1:top:singletop:tchannel} 
are obtained by either replacing the \Pup and \Pdown by a \Pcharm and \Pstrange
quarks or by switching the light quarks in the fermion line. The diagrams for antitop 
production are the charge conjugate of the ones presented. 

\begin{figure}
     \centering
     \begin{subfigure}[b]{0.3\textwidth}
         \centering
         \includegraphics[width=\textwidth]{Chapter1/Single-top-tchannel-A}
         \caption{}
         \label{fig:Chap1:top:singletop:tchannel_A}
     \end{subfigure}
     \hfill
     \begin{subfigure}[b]{0.3\textwidth}
         \centering
         \includegraphics[width=\textwidth]{Chapter1/Single-top-tchannel-B}
         \caption{}
         \label{fig:Chap1:top:singletop:tchannel_B}
     \end{subfigure}
     \hfill
     \begin{subfigure}[b]{0.3\textwidth}
         \centering
         \includegraphics[width=\textwidth]{Chapter1/Single-top-tchannel-C}
         \caption{}
         \label{fig:Chap1:top:singletop:tchannel_C}
     \end{subfigure}
        \caption{Representative Feynman diagrams for the single-top-quark production in the \tchannel process. Observe that the \Pup and 
        \Pdown quarks could be substituted by \Pcharm and \Pstrange quarks.}
        \label{fig:Chap1:top:singletop:tchannel}
\end{figure}

The measurements cross-sections at $13\,$TeV for single-top ($\sigma_{\tchannel, \Ptop}$) 
and single-anti-top ($\sigma_{\tchannel, \APtop}$) quarks in the \tchannel production are 
shown in Figure \ref{fig:Chap1:top:singletop:tchannel_CrossSection}.
The theoretical calculation at next-to-leading order
(NLO) at $13\,$TeV is $\sigma^{pred}_{\tchannel, \Ptop + \APtop }= 217.0^{+13.1}_{-11.1} \,\textrm{pb}$
\cite{CMS:2018lgn}. Due to the difference of valence quarks in the proton, the ratio 
between $\sigma^{pred}_{\tchannel, \Ptop}$ and $\sigma^{pred}_{\tchannel, \APtop}$ is 1.56. These numbers 
have been obtained using \HATHOR[2.1] \cite{Kant:2014oha}\cite{Aliev:2010zk} and a \mtop of $172.5\,$GeV.
%\begin{align*}
	%\sigma_{\tchannel, \Ptop} &= 136^{+4.1}_{-2.9} (\textrm{scale}) \pm 3.5(\textrm{PDF}+\alpha_{s})\,\textrm{pb,} \\
	%\sigma_{\tchannel, \APtop} &= 81.0^{+2.5}_{-1.7} (\textrm{scale}) \pm 3.2(\textrm{PDF}+\alpha_{s})\,\textrm{pb,} \\
	%\sigma_{\tchannel, \Ptop + \APtop} &= 217^{+6.6}_{-4.6} (\textrm{scale}) \pm 6.5(\textrm{PDF}+\alpha_{s})\,\textrm{pb.}
	%\sigma_{\tchannel, \Ptop} &= 136^{+7.6}_{-6.4}\,\textrm{pb,} \\
	%\sigma_{\tchannel, \APtop} &= 81.0^{+5.7}_{-4.9} \,\textrm{pb,} \\
	%\sigma_{\tchannel, \Ptop + \APtop} &= 217.0^{+13.1}_{-11.1} \,\textrm{pb.} \\
%\end{align*} % CMS Experimental results for these cross sections in: https://arxiv.org/pdf/1901.05247.pdf


\begin{figure}
    \centering
    \includegraphics[width = 0.75\textwidth]{Chapter1/singletop_tchan_xsec_13_lhc_nov20}
    \caption{Summary of the ATLAS and CMS measurements of the single top production cross-sections in the \tchannel at $13\,$TeV. The measurements are compared to NLO calculations \cite{ATLAS:2022uqj}.}
    \label{fig:Chap1:top:singletop:tchannel_CrossSection}
\end{figure}


The dominant process in the SM is the one in diagram \ref{fig:Chap1:top:singletop:tchannel_A}, while 
the one in \ref{fig:Chap1:top:singletop:tchannel_B} is included in order to form a gauge invariant set 
but its contribution is not very significative since for the gluon is easier to decay to $\Pbottom\APbottom$ 
pair than to a $\Ptop\APtop$ pair. These two $2 \rightarrow 3$ production modes are known 
as 4 Flavour Scheme (FS) because the proton is considered to be composed
by four quark flavours (\Pup, \Pdown, \Pcharm and \Pstrange). It is characterised by having a \Pbottom 
quark in the final state. This final-state \Pbottom-quark is sometimes referred as second\footnote{The 
first would be the one from the top quark decay.}.% \Pbottom and it 
%has a transverse momentum (\pT) distribution peaking around 2 or 3 GeV as can be seen 
%in Figure \ref{fig:Chap1:top:singletop:tchannel:ptVSeta}. This is the reason why the final \Pbottom quark 
%from the gluon splitting frequently goes undetected, because it does not pass the \pT threshold of the detector. 
%This is why, at detector level, whenever only jet is identified as originated from a \Pbottom quark, it is assumed to be
%the \Pbottom from the top-quark decay. This particularity becomes more important in  Chapter \ref{chap:Analysis_tH},
%where the number of detected \btagged jets is a relevant variable for the definition of the preselection region.


The $2 \rightarrow 2$ process in \ref{fig:Chap1:top:singletop:tchannel_C} is known as 
5FS because the proton has 
four flavours of quarks and since the process has a \Pbottom quark in the initial state, there are the five flavours.
The simulations for the 4FS and 5FS diagrams are produced separately and
merged afterwards. When adding the two contributions, some double-counting may appear due to the overlap in the phase space so one has to be careful. The naming 4FS and 5FS is later used again for the associated \tH production.


%%%     Single-top quark   :::  s-channel
\paragraph{\schannel}\mbox{}\\
The \schannel process for single-top is the one with less impact among single-top production channels.  
It is depicted in Figure \ref{fig:Chap1:top:singletop:schannel}. 
This production mode is also referred as the quark-antiquark annihilation or $W^{*}$ process  and it is very similar
to the Drell-Yann. 

According to the LHC cross-section group, at $\CM=13\,$TeV, the combined 
cross-section for the single top and single anti-top production in the \schannel 
is $\sigma^{pred}_{\schannel, \Ptop + \APtop} = 10.32^{+0.56}_{-0.61} \,\textrm{pb}$ 
\cite{Kant:2014oha}.
%\begin{align*}
%	\sigma_{\schannel, \Ptop} &= 6.35^{+0.18}_{-0.15} (\textrm{scale}) \pm 0.9(\textrm{PDF}+\alpha_{s})\,\textrm{pb,} \\
%	\sigma_{\schannel, \APtop} &= 3.97^{+0.11}_{-0.09} (\textrm{scale}) \pm 0.15(\textrm{PDF}+\alpha_{s})\,\textrm{pb,} \\
%	\sigma_{\schannel, \Ptop + \APtop} &= 10.32^{+0.29}_{-0.34} (\textrm{scale}) \pm 0.27(\textrm{PDF}+\alpha_{s})\,\textrm{pb.}
%	\sigma_{\schannel, \Ptop + \APtop} &= 10.32^{+0.56}_{-0.61} \,\textrm{pb.}
%\end{align*} % Source: https://twiki.cern.ch/twiki/bin/view/LHCPhysics/SingleTopRefXsec#Single_top_s_channel_cross_secti

%\begin{figure}
%    \centering
%    \includegraphics[width = 0.34\textwidth]{Chapter1/Single-top-schannel}
%    \caption{Representative Feynman diagram for the single-top-quark production in the \schannel.}
%    \label{fig:Chap1:top:singletop:schannel}
%\end{figure}

\begin{figure}
     \centering
     \begin{subfigure}[b]{0.3\textwidth}
         \centering
         \includegraphics[width=\textwidth]{Chapter1/Single-top-schannel}
         \caption{}
         \label{fig:Chap1:top:singletop:schannel}
     \end{subfigure}
     \begin{subfigure}[b]{0.3\textwidth}
         \centering
         \includegraphics[width=\textwidth]{Chapter1/Single-top-Wt-A}
         \caption{}
         \label{fig:Chap1:top:singletop:tW_A}
     \end{subfigure}
     \begin{subfigure}[b]{0.3\textwidth}
         \centering
         \includegraphics[width=\textwidth]{Chapter1/Single-top-Wt-B}
         \caption{}
         \label{fig:Chap1:top:singletop:tW_B}
     \end{subfigure}
        \caption{Representative Feynman diagrams for the single-top-quark production in (a) the \schannel
        and with (b, c) an associtared \PW boson. While the first one is not observed, the $\Ptop\PW$ is one the backgrounds
        in the \tHq analysis.}
        \label{fig:Chap1:top:singletop:SchannelAndAssocited}
\end{figure}


Although while at LHC only an evidence\footnote{The threshold for "evidence" corresponds to 
p-value=0.003 (three standard-deviations) while the standard for "discovery" is 
p-value=0.0000003 (five standard-deviations). \pablo{Review this p-values}}
of the \schannel production has been found \cite{ATLAS:2022wfk}, for Tevatron it was a
a significant part of the total single-top cross-section \cite{Cremonesi:2014dma}. 
% Table 4 of https://citeseerx.ist.psu.edu/viewdoc/download?doi=10.1.1.205.6486&rep=rep1&type=pdf


%%%     Single-top quark   :::  associated Wt-channel
\paragraph{Associated $\Ptop\PW$}\mbox{}\\
%\begin{minipage}[t]{0.5\linewidth}
Finally, the associated production of a single top quark with a \PW boson 
(sometimes referred as $\Ptop\PW$-channel) is represented by two the Feynman 
diagrams in Figures \ref{fig:Chap1:top:singletop:tW_A} and \ref{fig:Chap1:top:singletop:tW_B}.
To these two diagrams, the charge conjugate processes could be added to 
complete the $\Ptop\PW$ mechanisms. 
The predicted cross-section for the associated $\Ptop\PW$ is
$\sigma^{pred}_{tW, \Ptop + \APtop} = 71.7 \pm 5.2 \,\textrm{pb}$. This and 
all $\sigma$ in this section are calculated for a top mass of $\mtop = 172.5\,$GeV.  

Cross-section measurements for the associated $\Ptop\PW$ production  
performed by ATLAS and CMS at $13\,$TeV have found
 $\sigma_{tW} = 94^{+38}_{-32} \,\textrm{pb}$ \cite{ATLAS:2016ofl}
and $\sigma_{tW} = 79.2\pm8.9\,\textrm{pb}$ \cite{CMS:2022xey} respectively. 
Both results are compatible with the NLO + next-to-next-to-leading logarithmic (NNLL) prediction.
%\begin{equation*}
	%\sigma_{tW, \Ptop + \APtop} &= 71.7 \pm 1.80 (\textrm{scale}) \pm 3.40(\textrm{PDF}+\alpha_{s})\,\textrm{pb.}
%	\sigma_{tW, \Ptop + \APtop} = 71.7 \pm 5.2 \,\textrm{pb.}
%\end{equation*}

%and NLO in QCD with \HATHOR v.2.1. The PDF and $\alpha_{s}$ uncertainties are calculated
%using the PDF4LHC prescription \cite{Butterworth:2015oua} with the 
%MSTW2008 68\% CL NLO \cite{Martin:2009iq}\cite{Martin:2009bu}, CT10 NLO \cite{Lai:2010vv} 
%and NNPDF2.3 \cite{Ball:2013hta} PDF sets, added in quadrature to the scale uncertainty.
%The measurements for the associated $\Ptop\PW$ are shown in Figure \ref{fig:Chap1:top:singletop:tW_CrossSection}.
%The associated $\Ptop\PW$ production is an important background in the studies of the Higgs boson.
%\end{minipage} \hfill
%\begin{minipage}[t]{0.4\textwidth}
%\setbox0=\hbox{T} % tallest letter of the first line of other minipage
%\vskip-\ht0
%\begin{figure}
	%\begin{wrapfigure}{r}{0.45\textwidth}
%	    \includegraphics[width = 0.95\linewidth]{Chapter1/singletop_tW13_jun22}
% 	   \captionof{figure}{Cross-section measurements for the associated $\Ptop\PW$ production boson 
%	   performed by ATLAS and CMS at $13\,$TeV, and combined result compared 
%	   with the NLO + next-to-next-to-leading logarithmic (NNLL) prediction.
%	   \pablo{Quitar tabla y poner resultados en el párrafo.}}
%   	 \label{fig:Chap1:top:singletop:tW_CrossSection}
	 %\end{wrapfigure}
%\end{figure}
%\end{minipage}


%\begin{figure}
%     \centering
%     \begin{subfigure}[b]{0.3\textwidth}
%         \centering
%         \includegraphics[width=\textwidth]{Chapter1/Single-top-Wt-A}
%         \caption{}
%         \label{fig:Chap1:top:singletop:tW_A}
%     \end{subfigure}
%     \begin{subfigure}[b]{0.3\textwidth}
%         \centering
%         \includegraphics[width=\textwidth]{Chapter1/Single-top-Wt-B}
%         \caption{}
%         \label{fig:Chap1:top:singletop:tW_B}
%     \end{subfigure}
%        \caption{Representative Feynman diagrams for the single-top-quark production in association with a \PW boson.}
%        \label{fig:Chap1:top:singletop:tW}
%\end{figure}





%%%%%%%%%%%%%%%%
%                Four tops               %
%%%%%%%%%%%%%%%%
\subsubsection{Four tops}
\label{sec:Chap1:Top:Production:4tops}
%source: 
%		https://inspirehep.net/files/216bcad412f9e5d56fda0820c1b726c9
%		https://atlas.web.cern.ch/Atlas/GROUPS/PHYSICS/CONFNOTES/ATLAS-CONF-2020-013/
The production of four top quarks ($t\bar{t}t\bar{t}$) is a rare SM process that takes place at LHC 
with a predicted cross section of $\sigma^{pred}_{t\bar{t}t\bar{t}} = 12.0^{+2.2}_{-2.5}\,$fb for \Pproton\Pproton
collisions at $\CM=13\,$TeV (calculations at NLO for QCD and EW \cite{Frederix:2017wme}). 
ATLAS and CMS have measured the $t\bar{t}t\bar{t}$ production cross-section and obtained,
respectively, $\sigma_{t\bar{t}t\bar{t}} = 24^{+16}_{-14}\,$fb \cite{ATLAS:2021kqb} and 
$\sigma_{t\bar{t}t\bar{t}} = 12.6^{+5.8}_{-5.3}\,$fb \cite{CMS:2019rvj}.
While the first measurement does not agree with the theoretical prediction, 
it yields a larger statistical significance. 

% Remove summary plots and Feynman diagrams for the tttt
\begin{comment}
Figure \ref{fig:Chap1:top:4top_xsec} presents the $\sigma_{t\bar{t}t\bar{t}}$ measurements 
by ATLAS and CMS compared to the quoted theoretical calculation. 

The representative Feynman diagrams are presented  in Figure \ref{fig:Chap1:top:4Top:Feyn}, 
where of particular interest is the production by the exchange of a Higgs boson. 
This indicates a strong dependence of this type of production
with the top-quark-Yukawa coupling.

\begin{minipage}[t]{0.55\textwidth}
\begin{flushleft}
\noindent
\vskip0pt
	%\begin{figure}
    	%\centering
    	\includegraphics[width =\textwidth]{Chapter1/fourtop_summary_mar22}
    	\captionof{figure}{Summary of the ATLAS and CMS measurements of 
	the $t\bar{t}t\bar{t}$ production cross-sections at $13\,$TeV in various channels.
	\pablo{Quitar figura, mencionar el resultado más preciso combinado de ATLAS y su significancia.}}
        \label{fig:Chap1:top:4top_xsec}
        \strut
	%\end{figure}
\end{flushleft}	
\end{minipage}\hfill
\begin{minipage}[t]{0.4\textwidth} 
%\setbox0=\hbox{T} % tallest letter of the first line of other minipage
\vskip0pt
	%\begin{multicols}{2}
\includegraphics[width=0.85\linewidth]{Chapter1/Feynman_FourTop_A}\par 
    	%\end{multicols}
	%\begin{multicols}{2}
\includegraphics[width=0.85\linewidth]{Chapter1/Feynman_FourTop_B}\par
	%\end{multicols}
	\captionof{figure}{Representative Feynman diagrams for the $\Pgluon\Pgluon \rightarrow t\bar{t}t\bar{t}$ production.}
        \label{fig:Chap1:top:4Top:Feyn}
\end{minipage}
% Avoid the vertical space in the minipage envirorment:
% https://tex.stackexchange.com/questions/331617/too-much-vertical-space-before-a-minipage-environment
\end{comment}


%%%%%%%%%%%%%
%                tt-X              %
%%%%%%%%%%%%%
\subsubsection{Associated \ttbar\textit{+X} production}
\label{sec:Chap1:Top:Production:ttbar_plus_X}
% Measurements for top+X: https://atlas.web.cern.ch/Atlas/GROUPS/PHYSICS/PUBNOTES/ATL-PHYS-PUB-2022-049/
The associated top productions are important processes to measure
the coupling of the top to the other particles of the SM. 
When a pair of tops is produced along other particle 
it is referred as $\ttbar X$. The most relevant $\ttbar X$ productions
are those in which the pair is created with \PW, \PZ or \Pgamma boson.
From these, \ttW and \ttZ play a role in this thesis's analysis.
This two processes are backgrounds in
the \dilepOStau channel of the \tHq production, being the second and third
most important after \ttbar. 
%The \ttH process is not included in the \ttbar\textit{+X}  section but described in Section \ref{sec:Chap1:top-Higgs}. 
%The different production diagrams for \ttW are shown
%in Figure \ref{fig:Chap1:top:ttX:ttW_Feynman}. 
\begin{comment}
\begin{figure}
        \begin{subfigure}[b]{0.25\textwidth}
                \includegraphics[width=\linewidth]{Chapter1/ttW_qq_A}
                \caption{}
                %\label{}
        \end{subfigure}%
        \begin{subfigure}[b]{0.25\textwidth}
                \includegraphics[width=\linewidth]{Chapter1/ttW_qq_B}
                \caption{}
                %\label{}
        \end{subfigure}%
        \begin{subfigure}[b]{0.25\textwidth}
                \includegraphics[width=\linewidth]{Chapter1/ttW_gq_A}
                \caption{}
                %\label{}
        \end{subfigure}%
        \begin{subfigure}[b]{0.25\textwidth}
                \includegraphics[width=\linewidth]{Chapter1/ttW_gq_B}
                \caption{}
                %\label{}
        \end{subfigure}
        \caption{Representative Feynman diagrams for \ttW production.
        Left diagrams show the $\bar{q}q'\rightarrow \ttW$ processes 
        and right ones the $\bar{q}g\rightarrow \ttW q'$ production. \pablo{igual tampoco
        	hace falta enseñar estos diagramas}}\label{fig:Chap1:top:ttX:ttW_Feynman}
\end{figure}
\end{comment}

The cross sections for the \ttW, $\ttbar \gamma$ and $\ttbar \PZ$ productions 
have been measured for ATLAS and CMS with a large degree of agreement between the 
two experiments. %The ATLAS measurements and NLO calculations for \ttW are 
%$\sigma_{\ttW} = 0.87 \pm 0.27\,\text{pb}$ \cite{ATLAS:2019fwo}, 
%$\sigma^{pred}_{\ttW} = 0.72^{+0.08}_{-0.09}\,\text{pb}$ \cite{ATLAS:2018fwq}. For the $\ttbar \gamma$
%process $\sigma_{\ttbar \gamma} = 0.798 \pm 0.055\,\text{pb}$ \cite{ATLAS:2020yrp}, 
%$\sigma^{pred}_{\ttbar \gamma} = 0.77\pm0.14\,\text{pb}$. The $\ttbar \PZ$ production yields
%a cross section of $\sigma_{\ttbar \PZ} = 0.99 \pm 0.13\,\text{pb}$ \cite{ATLAS:2021fzm}, 
%$\sigma^{pred}_{\ttbar \PZ} = 0.86^{+0.09}_{-0.11}\,\text{pb}$ \cite{Kulesza:2020nfh}. 
These are presented in 
Figure \ref{fig:Chap1:top:ttX:Cross-Sec}. For both \ttbar\textit{+X} and $t\bar{t}t\bar{t}$, the cross sections
are small but with the complete Run$\,$2 sample it is possible to explore these production channels, which
are sensible to new physics \cite{ATLAS:2020hpj}.  The associated production of a \ttbar pair with a Higgs boson
is described in Section \ref{sec:Chap1:ttH}.

\begin{figure}
    \centering
    \includegraphics[width = 0.9\textwidth]{Chapter1/ttX_summary_Nov22}
    \caption{Summary of the ATLAS and CMS measurements of the $\ttbar X$ production 
    cross-sections at $13\,$TeV. Here $X=\PW$, $\PZ$ and $\Pgamma$.} %\pablo{Esta figura se queda porque ahorra tenxto.}} %Update from https://twiki.cern.ch/twiki/bin/view/LHCPhysics/LHCTopWGSummaryPlots#Pair_production_cross_section
    \label{fig:Chap1:top:ttX:Cross-Sec}
\end{figure}

%%%%%%%%%%%%%
%                tX              %
%%%%%%%%%%%%%
\subsubsection{Associated \Ptop\textit{+X} production}
\label{sec:Chap1:Top:Production:top_plus_X}
Not only the top pairs but also the single-top quark can be produced
in association with other particles ($\Ptop X$). This type of productions
play an important role in the  \tHq searches since the \tZq process
is one of the backgrounds in the more difficult to separate in the
\dilepOStau channel. 
The other associated $\Ptop X$ production of the EW type is the $\Ptop q \Pgamma$, in which the
top quark is produced along with a photon.  Both \tZq and $\Ptop q \Pgamma$
are sensitive to beyond the SM (BSM) physics like flavour-changing neutral currents  or vector-like quarks.
ATLAS and CMS have measured both processes and while for \tZq a good SM agreement
is found, this is not the case for the $\Ptop q \Pgamma$ as can be seen in 
Figure \ref{fig:Chap1:top:tX:Cross-Sec}

In this work, the most relevant $\Ptop X$ mode is the 
single-top-quark production in association with a Higgs boson
This is the main process to search for in the thesis and its features
are discussed with more detail in Section \ref{sec:Chap1:tHq}.

\begin{figure}
    \centering
    \includegraphics[width = 0.9\textwidth]{Chapter1/tX_summary_jun22.pdf}
    \caption{Summary of the ATLAS and CMS measurements of the $\Ptop X$ production 
    cross-sections at $13\,$TeV. Here $X=\PZ$ and $\Pgamma$. The \tHq process 
    in not included in this plot but the work developed in this document aims to help
    to provide better limits on ints cross-section.} %Update from https://twiki.cern.ch/twiki/bin/view/LHCPhysics/LHCTopWGSummaryPlots#Pair_production_cross_section
    \label{fig:Chap1:top:tX:Cross-Sec}
\end{figure}







%%%%%%%%%%%%%%%%%
%                Top decay               %
%%%%%%%%%%%%%%%%%
\subsection{Top-quark decay}
As advanced in the Section \ref{sec:Chap1:Top:Production:SingleTop}, due to the large $V_{tb}$ element of the CKM matrix, the 
top quark decays almost entirely ($\sim 99.8 \%$) thorough the medium of the \Wtb vertex to a \Pbottom quark and a \PW boson.
This vertex and the decay chain of the top quark is 
represented in Figure \ref{fig:Chap1:top:decay:TopQuarkDecay}.
The final state decay is classified according to the subsequent decay of the \PW boson. Since the \PW are massive vector bosons,
its lifetime is very short ($\tau_{\PW} \approx 3\times10^{-25}\,$s). For the \PWplus, the BRs for the different decay modes are \cite{Workman:2022ynf}:
\begin{align*}
	\PWplus &\rightarrow \Ppositron \Pneutrino_{e} 			&& (10.71 \pm 0.16)\% \\
	\PWplus &\rightarrow \APmuon \Pneutrino_{\mu} 			&& (10.63 \pm 0.15)\% \\
	\PWplus &\rightarrow \APtauon \Pneutrino_{\tau} 			&& (11.38 \pm 0.21)\% \\
	\PWplus &\rightarrow \Pquark \APquark \textrm{ (hadrons)}	&& (67.41 \pm 0.27)\% \\
	\PWplus &\rightarrow \textrm{invisible}					&& (1.4 \pm 2.9)\% 
\end{align*} 
For the conjugate proceses involving the \PWminus, the BR are the same. Therefore, the \PW decay 
and consequently the top-quark decay can be classified either as leptonic or hadronic.  
Due to its large mass, the \PW can decay to any quark except the top quark. 
For a certain decay mode, the BR is defined as the fraction times that the particle decays in that particular mode with 
respect to total possible decays.

\begin{figure}
    \centering
    \includegraphics[width = 0.45\textwidth]{Chapter1/TopQuarkDecay}
    \caption{Decay of a top quark to a \Pbottom quark and a \PW boson. The \PW boson can 
    		decay either leptonically to a neutrino and a lepton
    		or hadronically to a pair of light-flavour quarks. In the hadronic \PW decay, 
		a jet triplet is formed along with the \Pbottom quark.}
    \label{fig:Chap1:top:decay:TopQuarkDecay}
\end{figure}



%%%%%%%%%%%%%%%%%%%%
%                Top polarisation               %. Not including polarisation in the thesis
%%%%%%%%%%%%%%%%%%%%
\subsection{Top quark polarisation}
\label{sec:Chap1:Top:Polarisation}
% Polarisation paper: https://inspirehep.net/files/7e59402081847561840253401cf21138
% Figures from: https://atlas.web.cern.ch/Atlas/GROUPS/PHYSICS/PAPERS/TOPQ-2018-10/
As mentioned previously, the lifetime of the top quark is shorter than the depolarisation scale
and, hence, the top-quark spin information can be transferred into its decay products. This
allows to measure the top-quark polarisation from its child particles. The polarisation refers to 
the alignment between the momentum and the spin of the top quark and antiquarks.
The polarisation of \Ptop and \APtop are important quantities because they are sensitive
to many BSM effects and can also provide useful input for the MC generators which are
described in Section \ref{sec:Chap3.1:MC}.

At LHC, the single-top-quark production is the only source of highly polarised top quarks.
Specifically, in the \tchannel (see Section \ref{sec:Chap1:Top:Production:SingleTop}) the top
quark is created with a high degree of polarisation in the direction of the spectator quark momentum \cite{Komm:2014fca}.
As a consequence of the vector and axial-vector
form of the coupling of the top quark to the \PW boson
and bottom quark in the \tchannel (\tWb vertex), specific
values of the polarisation vectors $\{P_{x'},\, P_{y'},\, P_{z'}\}$ of top
quarks/antiquarks are expected in the SM.

Even though it is not described with detail in this manuscript, during the development 
of my thesis I have also been involved in the first measurement of the top-(anti)quark-polarisation vectors. 
My contribution is an extension of the work done in reference \cite{Martinez-Agullo:2017lty} and the results of these measurements
are published in reference \cite{ATLAS:2022vym}. In this work, the three components of the polarisation 
vector for the top quark and antiquark have been measured in the single-top \tchannel production. Using the 
entire Run$\,$2 dataset recorded by ATLAS and demanding events with exactly one light lepton,
I defined a set of stringent selection requirements to discriminate the \tchannel signal 
from the background contributions.  This signal-region\footnote{The signal region is a 
region of the phase space enriched with events of the signal process.}
definition used specific cuts\footnote{To ''cut´´ on a variable is to apply a threshold on this variable and
keep only events satisfying this condition. A cut-based analysis is
applying such thresholds on several variables to select events.}
 in several variables such as the lepton \pt or the invariant masses of several particles.
I'have also developed the so called trapezoidal cut, which is described in the published paper. 

The polarisation vectors are later obtained from the distributions 
of the direction cosines of the charged-lepton momentum in the top-quark rest frame:
$\text{cos}(\theta_{lx'})$, $\text{cos}(\theta_{ly'})$ and $\text{cos}(\theta_{lz'})$. 
Figure \ref{fig:Chap1:Polarisation:Observables} shows the distributions for this angular 
variables.

\begin{figure}[htp]
\centering
	\includegraphics[width=.3\textwidth]{Chapter1/PolarisationObservable_X}\hfill
	\includegraphics[width=.3\textwidth]{Chapter1/PolarisationObservable_Y}\hfill
	\includegraphics[width=.3\textwidth]{Chapter1/PolarisationObservable_Z}
	\caption{Normalised differential cross-sections as a function of 
	$\text{cos}(\theta_{lx'})$, $\text{cos}(\theta_{ly'})$ and $\text{cos}(\theta_{lz'})$. 
	The data is shown as black points with statistical uncertainties compared to the 
	predictions of the MC generators, which are shown as lines. The ratio between the
	predictions and data is shown on the lower panel. 
	These plots are inclusive for top quark and top antiquark.} 
	\label{fig:Chap1:Polarisation:Observables}
\end{figure}

Limits on the two of the components of the polarisation vector of the
top quark and antiquark have been set and Figure \ref{fig:Chap1:Polarisation:Result}
presents the observed best-fit polarisation measurements for $P_{x'}$ and $P_{z'}$ in the
two dimensional parameter space.
The components of polarisation are measured to be:

\begin{minipage}[t]{0.45\textwidth}
For top quarks
	\begin{itemize}
		\item $P_{x'}^{t} =  0.01 \pm 0.18$
		\item $P_{y'}^{t} = -0.029 \pm 0.027$
		\item $P_{z'}^{t} =   0.91 \pm 0.10$
	\end{itemize}
\end{minipage}
\begin{minipage}[t]{0.45\textwidth} 
For top antiquarks
	\begin{itemize}
		\item $P_{x'}^{\bar{t}} =  -0.02 \pm 0.20$
		\item $P_{y'}^{\bar{t}} = -0.007 \pm 0.051$
		\item $P_{z'}^{\bar{t}} =   -0.79 \pm 0.16$
	\end{itemize}
\end{minipage}

Data measurements of the polarisation-vector components and 
differential cross-sections show good agreement with SM predictions.
 The results are consistent with NNLO QCD predictions and expectation of
$P_{y'}^{t} = P_{y'}^{\bar{t}} = 0$ from the hypothesis of \CP symmetry 
in the top-quark and top-antiquark decay. 
The significance of this analysis lies in the fact that within a relatively short 
time period following the discovery of the top quark, it became feasible to 
perform a differential measurement of its polarisation for the first time.

\begin{figure}[h]
    \centering
    \includegraphics[width=0.75\textwidth]{Chapter1/PolarisationFancy}
    \caption{Observed best-fit limit on two-dimensional top quark polarisation
    parameter space $\{P_{z'},\, P_{x'}\}$. The statistical-only (green)
    and the statistical+systematic uncertainty contours have a 68\% CL.
    The physically allowed values for $P_{x'}$ and $P_{z'}$ are lay inside the black circle. 
    The red point indicates the parton-level prediction at NNLO.}
    \label{fig:Chap1:Polarisation:Result}
\end{figure}


% The unfolding is performed with the TUnfold software package, developed at
% Deutsches Elektronen-Synchrotron (DESY). The process of unfolding can be thought
% of as correcting for blurring detector effects in order to unveil the spectrum of a given
% observable that is as close to the truth as possible.

%Angular measurements in ATLAS:
%\begin{itemize}
%	\item Top polarisation: how the top is produced
%	\item Helicity fractions: how the top decays
%	\item Spin correlation: information between produced top-quarks. 
%		Provides information about quantum entanglement. %source: https://indico.cern.ch/event/1139204/contributions/4850084/attachments/2437000/4174393/valencia_top.pdf
		
%\end{itemize} 


\begin{comment}
For polarisation:
Top polarised in direction of spectator quark
Define 3 Polarisation directions: $\{ P_{x´}, P_{xy}, P_{z´}, \}$
Lepton direction used as spin analyser ($\alpha_{l}$) 

In the single top \tchannel production, the top quark is polarised due to left-handed \PW-coupling.
Top-quark spin points in the direction of the spectator-quark(\Pq’) 


%%%%%%%%%%%%%%%%%%%%
%                   Top Physics                   %
%%%%%%%%%%%%%%%%%%%%
\subsection{Top quark physics}
\label{sec:Chap1:Top:Physics}
\pablo{Probably this section is not necessary since the physics are mostly explained above.}
The top quark couples directly to all SM vector (\Pphoton, \PWplus, \PWminus, \PZ, \Pgluon) 
and scalar (\PHiggs) bosons. For both photons and gluons, the interaction is 
described by a vectorial fermion-gauge coupling $\bar{\Psi}\Psi A_{\mu}$.
From boson-fermion interacting term of the $\mathcal{L}_{QED}$ in \ref{eq:chap1:QED_Complete}, the coupling 
of the top quark to photons (Figure \ref{fig:Chap1:TopPhys:Couplings:A}) has a strength of 
$eQ\gamma^{\mu} =\frac{2}{3}e\gamma^{\mu}$. Meanwhile, for the top-gluon coupling (Figure \ref{fig:Chap1:TopPhys:Couplings:g}),
the expanded form of the gluon-fermion term in the $\mathcal{L}_{QCD}$ of \ref{eq:chap1:QCD:Lagrangian_FinalCompact} gives the $g_{s}\frac{\lambda_{a}}{2}\gamma^{\mu}$.

\begin{figure}
\centering
\begin{subfigure}{.3\textwidth}
  \centering
  \includegraphics[width=.94\linewidth]{Chapter1/top_Coupling_topPhoton}
  \caption{Top-photon coupling.}
  \label{fig:Chap1:TopPhys:Couplings:A}
\end{subfigure}%
\begin{subfigure}{.3\textwidth}
  \centering
  \includegraphics[width=.94\linewidth]{Chapter1/top_Coupling_topGluon}
  \caption{Top-gluon coupling.}
  \label{fig:Chap1:TopPhys:Couplings:g}
\end{subfigure}
\begin{subfigure}{.3\textwidth}
  \centering
  \includegraphics[width=.94\linewidth]{Chapter1/top_Coupling_topWb}
  \caption{Top-\PW coupling.}
  \label{fig:Chap1:TopPhys:Couplings:W}
\end{subfigure}%
\begin{subfigure}{.3\textwidth}
  \centering
  \includegraphics[width=.94\linewidth]{Chapter1/top_Coupling_topZ}
  \caption{Top-\PZ cuopling.}
  \label{fig:Chap1:TopPhys:Couplings:Z}
\end{subfigure}
\begin{subfigure}{.3\textwidth}
  \centering
  \includegraphics[width=.94\linewidth]{Chapter1/top_Coupling_toph}
  \caption{Top-Higgs coupling.}
  \label{fig:Chap1:TopPhys:Couplings:H}
\end{subfigure}
\caption{Top quark coupling to SM bosons.}
\label{fig:Chap1:TopPhys:Couplings}
\end{figure}


For the charged weak current only the only the left-handed top couples to the \PWpm with coupling. This is done
via the \Wtb vertex with a strength of $g\gamma^{\mu}(1-\gamma^{5})V_{tb}$ (Figures \ref{fig:Chap1:top:decay:TopQuarkDecay} 
and \ref{fig:Chap1:TopPhys:Couplings:W}). The value of $V_{tb}$ is given in Table \ref{tab:Chap1:CKM}.
%This can be obtained from equation \ref{eq:chap1:EW:CovariantDerivatice1} andf
% $\mathcal{L}_{EW}$ in \ref{eq:chap1:EW:FinalL} .
The top couples to the \PZ bosons (Figure \ref{fig:Chap1:TopPhys:Couplings:Z})with unequal left and right-handed components, 
$\frac{ig}{2\textrm{cos}\,\theta_{W}}\gamma^{\mu}(v_{t}-a_{t}\gamma^{5})$. Being $v_{t} =1/2 -2Q_{t} \textrm{sin}^{2}\theta_{W}$ 
and $a_{t} = 1/2$.
Finally, for the Higgs boson (Figure \ref{fig:Chap1:TopPhys:Couplings:H}), the top quark couples with a Yukawa type interaction $\bar{\Psi}\Psi \phi$ with a strength $\yt = \frac{\sqrt{2}\mtop}{v}$, as Equation \ref{eq:chap1:HiggsMechanism:YukawaCoupling} states. 
All of these couplings are flavour-conserving, with the exception of the charged-current
interaction with the \PW bosons. 

\end{comment}


%%%%%%%%%%%%%%%%%%%%%
%                      Higgs boson                   %
%%%%%%%%%%%%%%%%%%%%%
\section{Higgs boson}
\label{sec:Chap1:HiggsBoson}
Following the top quark, the Englert-Brout-Higgs-Guralnik-Hagen-Kibble-Higgs boson or, for simplicity, Higgs boson (\PH) or just Higgs
is the most massive particle in the SM with a mass of
$\mH = 125.25 \pm 0.17$ GeV \cite{pdgHiggs}. The value provided by \cite{pdgHiggs} is an average
of the ATLAS combined measurement ($\mH = 124.86 \pm 0.27$ GeV
%\footnotesize{assuming statistical uncertainties only, the uncertainty of this result is only $\pm18\%$} 
\cite{ATLAS:2018tdk}) and the CMS results ($\mH = 125.46 \pm 0.16$ \cite{CMS:2020xrn}). %$\mH = 125.46 \pm 0.13$(stat)$\pm 0.10$(syst) GeV

The Higgs boson existence was theorised in 1964 by three independent groups: Englert-Brout \cite{PhysRevLett.13.321},  
Higgs \cite{PhysRevLett.13.508} and Guralnik-Hagen-Kibble \cite{PhysRevLett.13.585}, and its discovery meant one of the greatest successes of the SM. This theory
was not only able to calculate with great precision the 
observed physics phenomena but also predicted the existence of a particle 
that was found later (see \ref{sec:Chap1:HiggsPhys_discovery}).

\begin{comment}
In the SM, fundamental particles acquire mass through their interactions with the Higgs fields. It is important to note that not all mass is related to the Higgs mechanism.
For instance, the mass of the proton does not came from the interaction of its components with the Higgs but from the kinetic energy of the particles that compose the proton.
\end{comment}

%%%%%%%%%%%%%%%%%%%%%
%                      Higgs discovery             % Maybe remove this
%%%%%%%%%%%%%%%%%%%%%
\subsection{Higgs-boson discovery}
\label{sec:Chap1:HiggsPhys_discovery}
Any particle physicist enthusiast remembers July 4th of 2012 pretty well, it was the day when LHC 
experiments ATLAS \cite{20121_ATLAS_HiggsDiscovery} 
and CMS  \cite{201230_CMS_HiggsDiscovery} 
announced the discovery of a massive state \PH with the properties expected for the Higgs boson.
%This discovery of the Higgs boson and, by extension, the Higgs field completed the SM.
% Was the Higgs boson found then the one predicted by Higgs and Englert or one of many?
% There is no theoretical predictions for the Higgs mass

Both the ATLAS and CMS Collaborations reported excesses of events for 2011 (\CM=7 TeV and \lumi=4.8 fb$^{1}$) 
and 2012 (\CM=8 TeV and \lumi=458 fb$^{1}$)  datasets of proton-proton ($\Pproton \Pproton$) collisions.
This surplus of events was compatible in its production and decay with the SM Higgs boson in the
mass region $m_{H}\in [124$, $135]\,$GeV with significances of 2.9$\sigma$ for ATLAS and 3.1 $\sigma$ for CMS.
At Tevatron (circular proton-antiproton collider at Fermilab), the experiments CDF \cite{CDF:2012jmx} and D$\emptyset$
 \cite{D0:2012jgw} also reported
an excess in the mass region $m_{H}\in [120$, $135]$ GeV.
The discovery was the result of the combination of several  individual searches.

%%%%%%%%%%%%%%%%%%%%%%%%%
%                       Higgs production                       %
%%%%%%%%%%%%%%%%%%%%%%%%%
\subsection{Higgs boson production at LHC}
%\paragraph{\PHiggs production}\mbox{}\\
\label{sec:Chap1:Higgs_production}

One of the reasons why the Higgs boson was found the latest  among SM fundamental particles is because it is a fairly heavy particle and, hence, it was necessary a lot of energy to produce it. Even though
that colliders such us SLAC or LEP had enough energy, they were colliding electrons and positrons.,
Since the coupling of the Higgs to fermions is proportional to the fermions mass, 
the $\Pelectron \APelectron \rightarrow \PHiggs$ processes 
is highly suppressed\footnote{The dominant Higgs production in  
$\Pelectron \APelectron$ annihilation is the so called Higgsstrahlung, 
an $s$-channel process in which the $\PHiggs$ is produced in association 
to a $\PZ$ boson similarly to Figure \ref{fig:Chap1:Higgs:LOFeynman_C}. Due to electrons 
small mass, the electron-Higgs coupling does not favour the 
$\Pelectron \APelectron \rightarrow \PHiggs$ process.}  and, for this reason, there 
were not enough  statistics of events with a Higgs boson at SLAC and LEP. The most favoured 
way of producing a Higgs boson is trough the mediation of the heaviest 
fundamental particles in the SM because these have the strongest couplings 
with the Higgs and, consequently, the greater cross section.
%Figure \ref{fig:Chap1:Higgs:LOFeynman} shows the dominant mechanisms for Higgs boson production at the LHC. 
\begin{figure}
\centering
\begin{subfigure}{.23\textwidth}
  \centering
  \includegraphics[width=.99\linewidth]{Chapter1/Higg_Production_LO_FeynmanDiagrams_gluonFusion}
  \caption{Gluon Fusion\\ ($\Pgluon \Pgluon$F)}
  \label{fig:Chap1:Higgs:LOFeynman_A}
\end{subfigure}%
\begin{subfigure}{.23\textwidth}
  \centering
  \includegraphics[width=.99\linewidth]{Chapter1/Higg_Production_LO_FeynmanDiagrams_VectorBosonFusion}
  \caption{\PW or \PZ fusion\\ (VBF)}
  \label{fig:Chap1:Higgs:LOFeynman_B}
\end{subfigure}%
\begin{subfigure}{.23\textwidth}
  \centering
  \includegraphics[width=.99\linewidth]{Chapter1/Higg_Production_LO_FeynmanDiagrams_HiggstrahlungFusion}
  \caption{Higgsstrahlung\\ (VH)}
  \label{fig:Chap1:Higgs:LOFeynman_C}
\end{subfigure}%
\begin{subfigure}{.23\textwidth}
  \centering
  \includegraphics[width=.99\linewidth]{Chapter1/Higg_Production_LO_FeynmanDiagrams_AssociatedttH}
  \caption{\ttH production}
  \label{fig:Chap1:Higgs:LOFeynman_D}
\end{subfigure}%
\caption{Lowest-order Feynman diagrams for the dominant production mechanisms of a Higgs boson at hadron colliders.}
\label{fig:Chap1:Higgs:LOFeynman}
% The strength of Higgs coupling to other particles depends on their mass and for this reason it likes to 
% accompany large mass produced particles like those containing the heaviest top quark
\end{figure}

The four most dominant processes for Higgs boson production at LHC are summarised in Figure \ref{fig:Chap1:Higgs:LOFeynman}:
\begin{itemize}
  \item \textbf{Gluon-gluon Fusion} ($\Pgluon \Pgluon$F): 
				This channel is depicted in Figure \ref{fig:Chap1:Higgs:LOFeynman_A} and, as the diagram shows, the process 
				$\Pgluon \Pgluon \rightarrow \PHiggs$ has to be mediated by a massive fermion loop. This due to the fact that
				there is no direct gluon-Higgs coupling within the SM.  Although in principle all quarks should be included in the
				loop, in practise it is the top quark the one doing so becase its coupling to the Higgs boson is 35 times stronger
				than the next-heaviest fermion, the bottom quark.
				Due to the abundance of gluons in $\Pproton \Pproton$ collisions, the $\Pgluon \Pgluon$F is very favoured at LHC.
				
				Another interesting property is that the $\Pgluon \Pgluon$F production rate is sensible to the \CP-mixing angle in 
				the top Yukawa coupling. This is related to the one of the major aims of this thesis, the search of a presence of
				\CP-odd contributions in \yt.
				
							
  \item \textbf{Vector Boson Fusion} (VBF): 
  				The second most important mode is the radiation by the incoming quarks of a pair of
				\PW or  \PZ vector bosons that fuse to 
  				form a Higgs as Figure \ref{fig:Chap1:Higgs:LOFeynman_B} illustrates. The vector bosons of
				the process $V\bar{V}\rightarrow \PHiggs$ are originated from initial state quarks which scatter
				thorough the final state (changing its flavours in the case of \PW fusion) producing two forward jets.
				
  \item \textbf{Higgsstrahlung} (VH): 
  				There is another significant contribution involving the \PW or \PZ bosons, the Higgsstrahlung or associated 
				$\PW\PH$ or $\PZ\PH$ production. Here, a off-shell \PW or \PZ (formed from the annihilation of two 
				quarks) radiate a Higgs boson via $V^{*} \rightarrow V\PHiggs$. 
				Figure \ref{fig:Chap1:Higgs:LOFeynman_C} depicts the VH associated production.
  \item \textbf{Quark-pair associated production} ($q\bar{q}H$): 
  				In this mode, the Higgs is produced from a $q\bar{q}$ pair via $q\bar{q}\rightarrow \PHiggs$ with a $q\bar{q}H$
				final state. Typically, the involved quark pair is either a \bbbar or \ttbar. In the case of \ttbar 
				(Figure \ref{fig:Chap1:Higgs:LOFeynman_D}), the top quarks decay before hadronising, leading
				to final states with a high number of physics objects.
  \item \textbf{Associated Higgs boson and single-top quark} ($tHX$): 
  				This sub-dominant contribution can be either a \tHq or a \tWH. The former constitutes the central topic 
				developed in this thesis, where this process is searched. Details about this production modes are further
				discussed in Section \ref{sec:Chap1:tHq} .
  
\end{itemize}
  



%Due to the abundance of gluons in $\Pproton \Pproton$ collisions, the most dominant one, i.e. the one with largest cross-section, 
%is the Higgs production via gluon fusion mediated by top quarks (Figure \ref{fig:Chap1:Higgs:LOFeynman_A}). %This mode ($\Pgluon \Pgluon$F) 
%The second most important is the radiation by the incoming quarks of a \PW or \PZ vector bosons that fuse to 
%from a Higgs (Figure \ref{fig:Chap1:Higgs:LOFeynman_B}). 
%There is another significant contribution involving the \PW or \PZ bosons, the Higgsstrahlung or associated 
%$\PW\PH$ or $\PZ\PH$ production, a process in which the \PW or \PZ (formed from the annihilation of two 
%quarks) radiate a Higgs boson(Figure \ref{fig:Chap1:Higgs:LOFeynman_C}. 
%The last major contribution to the Higgs production at the LHC is its productions in association with a pair of 
%top and anti-top quarks (Figure \ref{fig:Chap1:Higgs:LOFeynman_D}). 

The cross section of the different mechanisms for single-Higgs-boson\footnote{So far,
the single Higgs production has been heavily studied at LHC but during Run\,3 the interest in double-Higgs production is increasing.}
production at $\CM=13\,$TeV are shown in Figure \ref{fig:Chap1:Higgs:CrossSection} as a function of \mH.
For Figure \ref{fig:Chap1:Higgs:CrossSection:Out}, the $\sigma_{\tH}$ accounts for the \tchannel and \schannel but not the $\Ptop\PW$-channel.
Assuming a $\mH=125.2\,$GeV, the cross sections for Higgs production are \cite{LHCHiggsCrossSectionWorkingGroup:2016ypw}:
%source: https://twiki.cern.ch/twiki/bin/view/LHCPhysics/CERNYellowReportPageAt13TeV


\begin{minipage}[t]{0.3\textwidth}
  \centering\raisebox{\dimexpr \topskip-\height}{%
  \includegraphics[width=\textwidth]{Chapter1/Pie_HiggsXSec}}
  %\captionof{figure}{}
  %\label{fig:Chap1:Higgs:Prod_and_BR:Prod}
\end{minipage}\hfill
\begin{minipage}[t]{0.7\textwidth}
\begin{flushleft}
\begin{flalign*}
	\sigma_{ggF}	&= 48.5_{-3.3}^{+2.2}\,\textrm{pb} \\
	\sigma_{VBF}	&= 3.78 \pm 0.05\,\textrm{pb} \\
	\sigma_{WH} 	&= 1.37\pm 0.03\,\textrm{pb} \\
	\sigma_{ZH} 	&= 0.89^{+0.04}_{-0.03}\,\textrm{pb} \\
	\sigma_{\ttH}	&= 0.5^{+0.03}_{-0.05}\,\textrm{pb} \\
	\sigma_{\bbbar H}	&=0.49^{+0.10}_{-0.11}\,\textrm{pb} \\
	\sigma_{tHX}	&= 0.09\pm 0.01\,\textrm{pb} 
\end{flalign*}
\end{flushleft}
\end{minipage}


\begin{figure}
\centering
\begin{subfigure}{.55\textwidth}
  \centering
  \includegraphics[width=.94\linewidth]{Chapter1/HiggsXSec_vs_mH_ZoomOut}
  \caption{}
  \label{fig:Chap1:Higgs:CrossSection:Out}
\end{subfigure}%
\begin{subfigure}{.45\textwidth}
  \centering
  \includegraphics[width=.94\linewidth]{Chapter1/HiggsXSec_vs_mH_ZoomIn}
  \caption{}
  \label{fig:Chap1:Higgs:CrossSection:In}
\end{subfigure}
\caption{Higgs boson production cross-sections as function of \mH 
at $\CM=13\,$TeV \cite{LHCHiggsCrossSectionWorkingGroup:2016ypw}.
A wide range of \mH values is showed in (a). In (b) is shown the result zooming 
around the measured  Higgs mass value. Looking at (a) it can be seen that if the Higgs 
weighted just about $60\,$GeV more there would have been only two relevant decay 
modes, \HWW and \HZZ. On the other hand, if had Higgs been just $30\,$GeV lighter, these
two channels would have been very difficult to observe.}
\label{fig:Chap1:Higgs:CrossSection}
\end{figure} % Source: Higgs boson CrossSection WG: https://twiki.cern.ch/twiki/bin/view/LHCPhysics/LHCHWG

%%%%%%%%%%%%%%%%%%%%%%%%%
%                       Higgs decays                       %
%%%%%%%%%%%%%%%%%%%%%%%%%
\subsection{Higgs-boson decay}
%\paragraph{\PHiggs decay}\mbox{}\\
\label{sec:Chap1:Higgs_decay}

The Higgs boson has a very short lifetime ($\tau_{\PH} = 1.6 \times 10^{-22}\,$s \cite{LHCHiggsCrossSectionWorkingGroup:2016ypw}) and, hence,
is always detected through its decay products. 
%The branching ratio (BR) is the fraction of particles which decay by an individual decay mode with respect
%to the total number of particles which decay and for the Higgs is shown in Figure \ref{fig:Chap1:Higgs:BR}.
Figure \ref{fig:Chap1:Higgs:BR} shows the branching ratio\footnote{The fraction of particles which decay by an 
individual decay mode with respect to the total number of particles which decay.} (BR) for the different Higgs-boson-decay-modes.
%\pablo{Looking at Figure \ref{fig:Chap1:Higgs:BR:Out} it can be seen that if the Higgs weighted just about $50\,$GeV more there would have been only two relevant decay modes, H->WW and H->ZZ.}

Despite the expected large Yukawa coupling between the 
Higgs boson and the top quark, the $\PH \rightarrow \ttbar$ is
forbidden becase the $\mH<2\mtop$. Consequently, the most 
prominent decay mode is the $\PHiggs \rightarrow \bbbar$ followed by the
$\PHiggs \rightarrow \PWplus \PWminus$. This is why for the \tHq searches, the 
channel in which the Higgs decay to \bbbar is the one with higher statistics. 
For the rest fermionic decays, the decay rates are ordered by the fermion masses,
being the $\Ptauon\APtauon$ decay mode (Figure \ref{fig:Chap1:Higgs:DecayModes:Htautau}) the most relevant among the leptonic.
Regardless of the expected large coupling between the weak force bosons and the Higgs, the $\PHiggs \rightarrow V V^{*}$ is
suppressed due to the requirement that one vector boson has to be produced off-shell\footnote{Off-shell menas that the particle is produced virtually and it does not satisfy the energy-momentum relation.}. 
In the context of determining possible non-SM \CP contributions in the top-Higgs coupling, 
the $\PHiggs \rightarrow  \Pgamma \Pgamma$ is also a relevant process because this
decay rate is sensible to the top Yukawa coupling.




% The relevance of each production and decay mode is portrayed in the pie charts of Figure \ref{fig:Chap1:Higgs:Prod_and_BR}. 
%Apart from the $\PWplus \PWminus$ decay, the other Higgs
%decay channel taken into consideration for the analysis carried in this thesis are $\PHiggs \rightarrow \PZ \PZ$ and $\PHiggs \rightarrow \Ptau \Ptau$.
For the analysis carried in this thesis, are of particular the decays $\PHiggs \rightarrow \PWplus \PWminus$, 
$\PHiggs \rightarrow \PZ \PZ$ (Figure \ref{fig:Chap1:Higgs:DecayModes:HZZ}) and $\PHiggs \rightarrow \Ptauon \APtauon$.
Sorted by its importance and assuming a $\mH=125.2\,$GeV, the BR for the Higgs are \cite{MelladoGarcia:2150771}: 
% Recommended values for SM Higgs BR: CERN Report4 :: https://cds.cern.ch/record/2150771/files/LHCHXSWG-DRAFT-INT-2016-008.pdf


\begin{minipage}[t]{0.4\textwidth}
  \centering\raisebox{\dimexpr \topskip-\height}{%
  \includegraphics[width=\textwidth]{Chapter1/Pie_HiggsDecay}}
  %\captionof{figure}{}
  %\label{fig:Chap1:Higgs:Prod_and_BR:BR}
\end{minipage}\hfill
\begin{minipage}[t]{0.6\textwidth}
\begin{flushleft}
\begin{align*}
	\PHiggs &\rightarrow  \bbbar 			& (57.92 \pm 0.29)\% \\
	\PHiggs &\rightarrow  \PWplus\PWminus	& (21.70 \pm 0.11)\% \\
	\PHiggs &\rightarrow  \Pgluon \Pgluon 	& (8.17 \pm 0.26)\% \\ 
	\PHiggs &\rightarrow  \Ptauon\APtauon	& (6.24 \pm 0.03)\% \\
	\PHiggs &\rightarrow  \Pcharm \APcharm	& (2.888 \pm 0.014)\% \\ 
	\PHiggs &\rightarrow  \PZ\PZ			& (2.667 \pm 0.013)\% \\  
	\PHiggs &\rightarrow  \Pgamma \Pgamma	& (2.270 \pm 0.023)\% \\ 
	\PHiggs &\rightarrow  \Pmuon\APmuon	& (2.165 \pm 0.011)\% \\  
	\PHiggs &\rightarrow  \PZ \Pgamma		& (0.155 \pm 0.008)\% \\
	\PHiggs &\rightarrow Others			& < 0.2 \%
\end{align*}
\end{flushleft}
\end{minipage}


\begin{figure}
\centering
\begin{subfigure}{.5\textwidth}
  \centering
  \includegraphics[width=.94\linewidth]{Chapter1/HiggsBR_ZoomOut}
  \caption{}
  \label{fig:Chap1:Higgs:BR:Out}
\end{subfigure}%
\begin{subfigure}{.5\textwidth}
  \centering
  \includegraphics[width=.94\linewidth]{Chapter1/HiggsBR_ZoomIn}
  \caption{}
  \label{fig:Chap1:Higgs:BR:In}
\end{subfigure}
\caption{Standard Model Higgs-boson-decay branching ratios as function of \mH at $\CM=13\,$TeV \cite{LHCHiggsCrossSectionWorkingGroup:2016ypw}. In (a) the BR are shown in a Higgs mass 
range $\mH \in (90, 10^3)\,$ GeV.  In (b) only values of \mH around the measured one are shown.
 Looking at (a) it can be seen that if the Higgs 
weighted just about $50\,$GeV more there would have been only two relevant decay 
modes, \HWW and \HZZ. On the other hand, if had Higgs been just $10\,$GeV lighter, these
two channels would have been very difficult to observe.}
\label{fig:Chap1:Higgs:BR}
\end{figure} % Source: Higgs boson CrossSection WG: https://twiki.cern.ch/twiki/bin/view/LHCPhysics/LHCHWG

\begin{figure}
\centering
\begin{subfigure}{.45\textwidth}
  \centering
  \includegraphics[width=.9\linewidth]{Chapter1/H_decayModes_Htautau}
  \caption{\Htautau decay.}
  \label{fig:Chap1:Higgs:DecayModes:Htautau}
\end{subfigure}%
\begin{subfigure}{.4\textwidth}
  \centering
  \includegraphics[width=.85\linewidth]{Chapter1/H_decayModes_HZZ}
  \caption{\HZZ decay.}
  \label{fig:Chap1:Higgs:DecayModes:HZZ}
\end{subfigure}
\caption{Feynman diagrams for the Higgs decay into a pair of (a) tau leptons and (b)  \PZ bosons. 
Both decay modes are taken into account for the associated \tHq production described in this thesis.
In (a), the \APtauon is decaying to quarks which will form hadrons, 
  therefore, it is referred as hadronic tau. In contrast, the \Ptauon in (a) is a leptonic tau. 
  For the diagram in (b), the Higgs decays into a four light-leptons final state.}
\label{fig:Chap1:Higgs:DecayModes}
\end{figure}

%\begin{figure}
%\centering
%\begin{subfigure}{.5\textwidth}
%  \centering
%  \includegraphics[width=.94\linewidth]{Chapter1/Pie_HiggsXSec}
%  \caption{Production}
%  \label{fig:Chap1:Higgs:Prod_and_BR:Prod}
%\end{subfigure}%
%\begin{subfigure}{.5\textwidth}
%  \centering
%  \includegraphics[width=.97\linewidth]{Chapter1/Pie_HiggsDecay}
%  \caption{Decay}
%  \label{fig:Chap1:Higgs:Prod_and_BR:BR}
%\end{subfigure}
%\caption{Percentage fractions for Higgs boson (a) production at $\CM=13\,$TeV for $\Pproton \Pproton$ collisions and (b) BR fraction for different decay channels \cite{LHCHiggsCrossSectionWorkingGroup:2016ypw}.}
%\label{fig:Chap1:Higgs:Prod_and_BR}
%\end{figure} % Source: https://cds.cern.ch/record/2800522?ln=en


\begin{comment}
%%%%%%%%%%%%%%%%%%%%%
%            Higgs boson physics                %
%%%%%%%%%%%%%%%%%%%%%
\subsection{Higgs boson physics}

\pablo{Work in progress}


The gauge symmetry is broken is broken by the vacuum, triggering the EW Spontaneous Symmetry Breaking (SSB). This means that the
symmetry group of the EW sector, $U(2)_{L} \bigotimes U(1)_{Y}$.

$SU(3)_{C} \bigotimes SU(2)_{L} \bigotimes U(1)_{Y} \xrightarrow{\text{SSB}} SU(3)_{C}  \bigotimes U(1)_{QED}$ 

% Higgs boson mass and width. Coupling to bosons. Coupling to fermions https://inspirehep.net/files/6c0b1f30a286a7aca715341b4fd90442

Production and decay rates, constrains on its couplings: \url{https://arxiv.org/abs/1606.02266}

The Higgs mass is given by $m_{H} =\sqrt{\lambda /2} v $, being $v$ the vacuum expectation value of the Higgs field and $\lambda$ the Higgs self-coupling. 

Electro weak symmetry breaking \cite{Pich:2015lkh}: \url{https://arxiv.org/pdf/1512.08749.pdf}
In this paper the Yukawa coupling of the top is introduced, link it with the tHq paper. 

% At peak lumi, the LHC was producing a $\PHiggs \rightarrow \Pgamma \Pgamma$ every 45 minutes \cite hoecker

% Slide 34, BR of Higgs decays https://indico.cern.ch/event/763013/contributions/3358697/attachments/1813182/2962418/Higgs-1.pdf

\pablo{Work in progress}

\pablo{This is not getting into the thesis because we don't want to verbose that much but we could talk about: Lepton Flavour Violating (FLV) Higgs, dark Higgs, etc, etc}\pablo{The precision measurement of the properties of the Higgs is one of the main focusses of LHC.}

\end{comment}



%                          %
%  i n t e r p l a y  %
%                          %


%%%%%%%%%%%%%%%%%%%%%
%                Top-Higgs interplay              %
%%%%%%%%%%%%%%%%%%%%%
\section{Top quark and Higgs boson interplay}
\label{sec:Chap1:top-Higgs}

%General yt
So far, the couplings of the Higgs boson to the SM particles have been found to be 
uniquely determined by the masses of these particles. Being this strength proportional
to the mass in the case of fermions and the squared mass for the bosons. 
Figure \ref{fig:Chap2:HiggsAllCouplings} presents 
the coupling-mass relationship of the Higgs boson with other SM particles.
Since the top quark is the most massive particle, 
the Yukawa coupling between the top quark and the Higgs boson (\yt) is expected 
to be the strongest among all fermions and, hence, its
study is of crucial importance, as it is discussed in references\,\cite{Farina:2012xp}\cite{Biswas:2012bd} 
and developed in the succeeding sections. The Yukawa coupling is expected to be of the order of the unity:
\begin{equation*}
	\yt = \frac{\sqrt{2}\mtop}{v} = 2^{3/4}G_{F}^{1/2}\mtop = 0.995 \simeq 1 \, .
\end{equation*}
This value is quite larger than the couplings of the other quarks. For comparison
$y_b \simeq 0.025$ and $y_c \simeq 0.007 >> y_{s,d,u}$.  

\begin{figure}
    \centering
    \includegraphics[width = 0.6\textwidth]{Chapter1/HiggsAllCouplings.png}
    \caption{The coupling of the Higgs boson to fermions (\Pmu, \Ptau, \Pbottom, \Ptop) 
    		and bosons (\PW, \PZ) as a function of the particle’s mass \cite{Collaboration:2045852}. 
		% Scaled under some theoretical assumptions.
		The diagonal line indicates the Standard Model prediction.}
    \label{fig:Chap2:HiggsAllCouplings}
\end{figure}


%   Yukawa coupling measurement
%   The SM expectation of the Yukawa coupling value  is $\yt = \frac{\sqrt{2}\mtop}{v}$ = 0.9956 \pm 0.0043$ 
%   source :: https://iopscience.iop.org/article/10.1088/1361-6471/44/6/063001/pdf

%The study of the couplings of the Higgs to the other particles are of prime importance 
%since they control the behaviour of the whole theory at high energy
%"""
%Measuring processes involving real top quarks in the final state
%will bring invaluable information. With the largest rate, the Higgs production in association
%with a top pair is a golden channel and has received great attention by the experimental 
%as well as theoretical communities \cite{Farina:2012xp}
%"""

%ttH
The production of a pair of top quarks along with a Higgs boson (\ttH) allows possible 
to measure the absolute value of \yt. This process has the advantage of 
being the leading mechanism to produce the Higgs together with the quark 
top. At $\CM=13\,$TeV it has a cross section of \pablo{poner cálculos del SM para $\sigma_{\ttH}$}.

%tH
Having a very much lower cross section than \ttH \pablo{(poner $\sigma_{\tH}$)}, 
the Higgs boson production alongside a single top quark (\tH) brings valuable information, specially
regarding the sign of the Yukawa coupling. Note that the sign of \yt is not a well-defined physical 
property by itself but the relative sign compared to the coupling of the Higgs to 
weak\footnote{The coupling of the Higgs to the gauge bosons is taken as positive.} boson is indeed physical \cite{Farina:2012xp}. 
This is explored with more detail in \ref{sec:Chap1:tHq}. 

% "The sign of the top Yukawa coupling is not physical by itself, but the relative sign compared to the Higgs
%    coupling to gauge bosons (we take the latter to be positive) is physical."  
%									- Farina

%A change in the Yukawa sign and/or absolute value with respect to its SM value
%would signal an origin of the fermion masses different from the described by the EWSB
%because the relative sign of the Higgs coupling to fermions
%and gauge vector bosons is crucial for recovering the unitarity and renormalisability of the
%theory \cite{Appelquist:1987cf}. \pablo{no estoy seguro de si este párrafo es cierto}





%%%%%%%%%%%%%%%%%%%%%%%
%                     CP in top-Higgs                     %
%%%%%%%%%%%%%%%%%%%%%%%
\subsection{\CP properties in top-Higgs interactions}
The \CP properties of the Yukawa coupling of the Higgs boson to the
top quark can be probed through the associated production of these 
two particles. While SM predicts the Higgs to be a scalar boson ($J^{\CP}=0^{++}$),
the presence of a $J^{\CP}=0^{+-}$ pseudoscalar admixture has not 
been excluded yet. This pseudoscalar would introduce a second coupling
to the top quark. Finding a \CP-odd contribution would be a sign of physics
beyond the SM and could account for the imbalance between matter
antimatter in the universe \cite{ATLAS:2020ior}. 

% A pseudoscalar is a quantity that behaves like a scalar, except that it changes sign under a parity inversion
% J = Spin
% C = charge conjugation eigenvalue
% P = Parity eigenvalue

The production rates of \ttH and \tH depend on the \yt coupling. The former
is specially sensible to \yt deviations from the SM as it is described in Section
\ref{sec:Chap1:tHq}. As already mentioned, the presence of a \CP-mixing in \yt would also affect the
$\Pgluon \Pgluon$F production and $\PHiggs \rightarrow \Pgamma \Pgamma$ decay rates.



%%%%%%%%%%%%%%%
%              ttH                      %
%%%%%%%%%%%%%%%
\subsection{\ttH}
\label{sec:Chap1:ttH}
The production of a top-antitop pair in association with a Higgs boson is one of the most important process
to measure the strength of the Yukawa coupling ($|\yt|$), wich is crucial to understand the origin of the fermion mases.
Detecting a deviation from the SM prediction for $\sigma(\ttH)$ could indicate the presence of new physics that
violate the \CP symmetry. But, as Figure \ref{fig:Chap1:xsec_vs_yt_tH_ttH} ilustrates, this process 
is not sensible to the sign of the Yukawa mixing angle.


From the phenomenology point of view, the calculations for the \ttH production cross-section
at $\CM=13\,$TeV can be calculated an NLO+NNLL
accuracy\,\cite{Broggio:2016lfj}:
\begin{equation*}
	\sigma_{NLO+NNLL} (\ttH) = 486^{+29.9}_{-24.5}\,fb.
\end{equation*}
This calculation depends on the chosen scales for the soft and hard processes but it gives and idea of the order 
of magnitud for this process. 
The LO Feynman diagrams for the \ttH production are presented in Figure \ref{fig:Chap1:ttH:Feynman}
% https://inspirehep.net/literature/1495438

\begin{figure}
\centering
\begin{subfigure}{.45\textwidth}
  \centering
  \includegraphics[width=.9\linewidth]{Chapter1/ttH_Feynman_A}
  \caption{}
  \label{fig:Chap1:ttH:Feynman:A}
\end{subfigure}%
\begin{subfigure}{.45\textwidth}
  \centering
  \includegraphics[width=.9\linewidth]{Chapter1/ttH_Feynman_B}
  \caption{}
  \label{fig:Chap1:ttH:tFeynman:B}
\end{subfigure}%
\caption{LO Feynman diagrams for \ttH production. Although this is the most 
relevant mechanism for the associated production of a Higgs boson with, at least, one top quark, 
the \ttH is still a rare process. It counts for roughly 1\% of all Higgs productions}
\label{fig:Chap1:ttH:Feynman}
\end{figure}


The first associated production of a Higgs boson with a pair of top 
quarks was observed in 2008 by ATLAS \cite{ATLAS:2018mme} 
%with a signal strength $\mu=\sigma/\sigma_{SM} $ of $0.84^{+0.64}_{-0.61}$
and CMS \cite{Skovpen:2018aoe}. 
This process marked a significant milestone for the field of high-energy 
physics because it helped establishing the first direct 
measurement of the tree-level coupling of the Higgs boson 
to the top quark, which was in agreement with the SM expectation.

The associated production of Higgs boson with top quark pair has been studied by ATLAS and CMS previously
not only during Run$\,$1 at $\CM=7\,$TeV and $8\,$TeV \cite{ATLAS:2014ayi} \cite{CMS:2014tll} 
but also at Run$\,$2\,\cite{ATLAS:2018mme}, where the cross section was expected 
to be increased by a factor of four. The ATLAS Run$\,$2 analyses use only an integrated 
luminosity up to 79.8 fb$^{-1}$ and considers the following Higgs-decay channels: 
\bbbar, $\PWplus \PWminus$, \Ptauon\APtauon, \Pgamma\Pgamma and \PZ\PZ.
Assuming the SM BR, the total measured-cross-section by  \cite{ATLAS:2018mme} is
$\sigma(\ttH)= 670^{+200}_{-190}\,$fb. This result, which is a combination 
of all the mentioned decay channels, is in agreement with the SM predictions and
is shown in Figure \ref{fig:Chap1:ttH:SigmaMeasure}. 
Meanwhile, CMS has found a strength of $\mu_{\ttH} = 1.38^{+0.36}_{-0.29}$ for the \Pgamma\Pgamma Higgs
decay channel and $\mu_{\ttH} = 0.92^{+0.36}_{-0.29}$ for the multipleton channels\,\cite{Giraldi:2022wkt}.
 %source CMS: https://arxiv.org/pdf/2208.08209.pdf

\begin{figure}
    \centering
    \includegraphics[width = 0.71\textwidth]{Chapter1/ttH_SigmaMeasure}
    \caption{Comparison of the measured \ttH production cross-section 
    to its SM theoretical expectation\,\cite{ATLAS:2018mme}.
    The red vertical line indicates the SM prediction.}
    \label{fig:Chap1:ttH:SigmaMeasure}
\end{figure}


% Status of ttH searches: http://cds.cern.ch/record/2748825/files/CERN-THESIS-2020-253.pdf

%Higgs boson production in association with top quarks in final states with electrons, muons, and hadronically decaying tau leptons at 13 T3V
%https://cds.cern.ch/record/2725523/files/HIG-19-008-pas.pdf
%\subsubsection{\ttH Standard Model}
%\subsubsection{\ttH Charge-Parity}


%%%%%%%%%%%%%%%%%%%
%              tH and tHq                       %
%%%%%%%%%%%%%%%%%%%
\subsection{\tH}
\label{sec:Chap1:tHq}

The associated \tH production takes place via three different types of processes.
Firstly, the \tchannel, where the Higgs boson couples to a top quark or \PW boson 
(Figures \ref{fig:Chap1:tHq:Feynman_LO_top} and \ref{fig:Chap1:tHq:Feynman_LO_W} respectively).
In this channel the top and Higgs are created along with an additional quark, 
giving rise to the so called \tHq production. 
The other two production modes are the \tW process, in which the Higgs couples to the top quark (Figure \ref{fig:Chap1:tHW:Feynman_LO}), 
and the  \schannel. In Section \ref{sec:Chap1:tH:ProductionModes}, the details of the
associated top-Higgs production modes are given.
All three processes have a much smaller cross section than the main Higgs production channels that where
discussed in Section \ref{sec:Chap1:Higgs_production}. However, the \tH modes yield a unique feature
that make them fascinating: they are simultaneously sensitive to the sign and magnitude of the Higgs coupling to both
the top quark, \yt, and the weak bosons, $g_{HVV}$.

\begin{figure}
\centering
 \begin{subfigure}{.29\textwidth}
  \centering
  \includegraphics[width=.9\linewidth]{Chapter1/tHq_production_LO_Feynman_top}
  \caption{}
  \label{fig:Chap1:tHq:Feynman_LO_top}
 \end{subfigure}%
 \begin{subfigure}{.29\textwidth}%
  \centering
  \includegraphics[width=.9\linewidth]{Chapter1/tHq_production_LO_Feynman_W}
  \caption{}
  \label{fig:Chap1:tHq:Feynman_LO_W}
 \end{subfigure}%
 \begin{subfigure}{.29\textwidth}
  \centering
  \includegraphics[width=.9\linewidth]{Chapter1/tHW_production_LO_Feynman}
  \caption{}
  \label{fig:Chap1:tHW:Feynman_LO}
 \end{subfigure}%
    \caption{Representative LO Feynman diagrams for the \tchannel \tHq associated production, where 
    the Higgs boson couples either to the top quark (a) or the W boson (b). Here $g_{HVV}$ is the coupling 
    of the Higgs boson to the vector bosons. On (c) an example of the  \tWH production is presented.}
    \label{fig:Chap1:tHq:Feynman_LO}
\end{figure}




%\paragraph{\tHq production modes}\mbox{}\\
\subsubsection{\tH production modes}
\label{sec:Chap1:tH:ProductionModes}
At LO, the production of a Higgs boson in association with a single-top quark and additional parton (\tHq) 
in $\Pproton \Pproton$ collisions is classified in three groups according 
to the virtuality of the \PW boson. These groups are: \tchannel and
\schannel productions, and associated production with and on-shell \PW boson. 
This categorisation is the same as for the single-top-quark (Section 
\ref{sec:Chap1:Top:Production:SingleTop}), which makes sense 
since the \tHq production is, basically, a single-top-quark process in which 
a Higgs boson is radiated either from the \PW boson or the top quark.
Note that this separation, while useful, is not physical and it only
holds at LO and 5FS. At higher orders in QCD or in other flavour
scheme, the classification becomes fuzzy.


%  tH ::  t-channel
\paragraph{\tH production in the \tchannel$\,$::$\,$\tHq}\mbox{}\\
The \tchannel production modes resemble the ones described in Figure \ref{fig:Chap1:top:singletop:tchannel}.
These are classified in 4FS and 5FS as it is done for the single-top case. 
The 4FS and the 5FS modes are shown in Figures \ref{fig:Chap1:tH:tchannel:4F} and \ref{fig:Chap1:tH:tchannel:4F} respectively. 
For the 4FS modes, the diagrams in which the gluon decays to a top pair (\ref{fig:Chap1:tH:tchannel:4F:C}, \ref{fig:Chap1:tH:tchannel:4F:D} and \ref{fig:Chap1:tH:tchannel:4F:E}) contribute less than the ones in which it does to a $\Pbottom\APbottom$ (\ref{fig:Chap1:tH:tchannel:4F:A} and \ref{fig:Chap1:tH:tchannel:4F:B}) because it is easier for the gluon to decay into  
$\Pbottom\APbottom$ than into \ttbar. The NLO cross-section for the \tHq process at $\CM=13\,$TeV is 
given by\footnote{The calculation of the $\sigma_{NLO}^{\tchannel}$ depends on the choice of scale. 
The numbers given here correspond to $\mu = (m_{H} + \mtop)/4$.}:
\begin{align}
	\sigma_{NLO}^{\tchannel} (\tH) & = 47.64\pm 9.7^{+2.9\%}_{-3.1\%}\\
	\sigma_{NLO}^{\tchannel} (\APtop\PHiggs) & = 24.88 \pm 10.2^{+3.5\%}_{-2.6\%} \, .
\end{align}
Combining the \tH and $\APtop\PHiggs$ contributions results in:
\begin{equation}
	\sigma_{NLO}^{\tchannel} (\tH + \APtop\PHiggs) = 72.55 \pm 10.1^{+3.1\%}_{-2.4\%}\,.
\end{equation}


\begin{figure}
\centering
\begin{subfigure}{.31\textwidth}
  \centering
  \includegraphics[width=.95\linewidth]{Chapter1/thq_tchannel_A}
  \caption{}
  \label{fig:Chap1:tH:tchannel:4F:A}
\end{subfigure}%
\begin{subfigure}{.31\textwidth}
  \centering
  \includegraphics[width=.95\linewidth]{Chapter1/thq_tchannel_B}
  \caption{}
  \label{fig:Chap1:tH:tchannel:4F:B}
\end{subfigure} 
\begin{subfigure}{.31\textwidth}
  \centering
  \includegraphics[width=.95\linewidth]{Chapter1/thq_tchannel_C}
  \caption{}
  \label{fig:Chap1:tH:tchannel:4F:C}
\end{subfigure} \quad
\begin{subfigure}{.31\textwidth}
  \centering
  \includegraphics[width=.95\linewidth]{Chapter1/thq_tchannel_D}
  \caption{}
  \label{fig:Chap1:tH:tchannel:4F:D}
\end{subfigure} %\quad
\begin{subfigure}{.31\textwidth}
  \centering
  \includegraphics[width=.95\linewidth]{Chapter1/thq_tchannel_E}
  \caption{}
  \label{fig:Chap1:tH:tchannel:4F:E}
\end{subfigure}%
\caption{LO Feynman diagrams for \tchannel \tH production in the 4FS. 
The red line represents the top quark while the blue is the \Pbottom quark.
This model works for energy scales on the order on $m_{\Pbottom}$.}
\label{fig:Chap1:tH:tchannel:4F}
\end{figure}

\begin{figure}
\centering
\begin{subfigure}{.5\textwidth}
  \centering
  \includegraphics[width=.9\linewidth]{Chapter1/thq_tchannel_F}
  \caption{}
  \label{fig:Chap1:tH:tchannel:5F:A}
\end{subfigure}%
\begin{subfigure}{.4\textwidth}
  \centering
  \includegraphics[width=.9\linewidth]{Chapter1/thq_tchannel_G}
  \caption{}
  \label{fig:Chap1:tH:tchannel:5F:B}
\end{subfigure}%
\caption{LO Feynman diagrams for \tchannel \tH production in 
the 5FS. Here, the \Pbottom quarks are considered massless.}
\label{fig:Chap1:tH:tchannel:5F}
\end{figure}


For \tHq and single-top-quark production at colliders, the 5FS calculations are easier 
than the 4FS due to the lesser final state-multiplicity and smaller phase space. 
This is why in the 5FS the single-top production
is known at NNLO while the 4FS this is done only for NLO. Another advantage of 
the 5FS is that the \tchannel, \schannel and associated \tWH production 
do not interfere until NNLO. Contrary, the in 4FS, the
\tchannel at NLO and \schannel at NNLO can interfere. 
Nevertheless, these interferences are very small and can be neglected \cite{Demartin:2015uha}.

Other feature of the 4FS is that it is assumed that the energy scale
of the hard process ($Q$) is not much higher than the bottom quark
mass, which is also significantly larger than the QCD scale ($\Lambda_{QCD})$.
Therefore, the model is limited to $Q \geq m_{\Pbottom} >> \Lambda_{QCD}$.
When $Q >> m_{\Pbottom}$ inaccuracies appear. 
In contrast, the 5FS asumes $Q >> m_{\Pbottom}$. In practise, the bottom mass
is set to zero in 5FS to simplify calculations \cite{Demartin:2015uha}.

The work developed in this thesis is focussing on this production type.
The associated production of a Higgs boson and top quark with an additional
light quark ($q$) and \Pbottom quark. This light quark is usually referred as spectator quark and it is expected 
to produce a jet in the ATLAS calorimeters (in Chapter \ref{chap:ATLAS} the 
detector and its components are presented)  with high $\eta$. 
The reason here to have this large $\eta(q)$ is because the $q$ was contained within the initial parton
and, therefore, it continues in the direction of the beam. %The \Pbottom quark is usually named
%spectator or second \Pbottom and it is typically not detected due to its small $\pT$, as
%Figure \ref{fig:Chap1:top:singletop:tchannel:ptVSeta} depicts. 
The \tWH is considered a background in this analysis because
it does not have the same signature as the \tHq process.



%  tH ::  s-channel
\paragraph{\tH production in the \schannel}\mbox{}\\
The \schannel contribution to the total cross-section of the \tH process
is very small. Additionally, this channel contributes at los \pT and,
since a \pT cut is applied in ATLAS, the \schannel events are suppressed.
For these two reasons, this channel plays a less important rol in the
associated top-Higgs production. The NLO total cross-section for the 
\tH process via the \schannel at $\CM=13\,$TeV is:
\begin{equation}
	\sigma_{NLO}^{\schannel} (\tH + \APtop\PHiggs) = 2.812^{+3.3\%}_{-3.1\%}\,.
\end{equation}

In reference \cite{Demartin:2015uha} is shown that the shapes of the distributions 
of most observables in the \schannel differ significantly 
from those of the \tchannel. So, even though the total cross-section of the \tH production 
with the \schannel is much more smaller than the one for \tchannel, one 
could think that including the \schannel in the analysis would increase the precision. 
This is not the case because the LHC is not sensible for to the \tH production
via the \schannel for the reasons mentioned above. 
In fact, not event the \schannel-single-top production 
(without any associated Higgs boson) has been found at LHC. 

\begin{figure}
\centering
\begin{subfigure}{.4\textwidth}
  \centering
  \includegraphics[width=.9\linewidth]{Chapter1/thq_schannel_A}
  \caption{}
  \label{fig:Chap1:tH:schannel:A}
\end{subfigure}%
\begin{subfigure}{.4\textwidth}
  \centering
  \includegraphics[width=.9\linewidth]{Chapter1/thq_schannel_B}
  \caption{}
  \label{fig:Chap1:tH:schannel:B}
\end{subfigure}%
\caption{LO Feynman diagrams for \schannel \tH production in the 5FS.}
\label{fig:Chap1:tH:schannel}
\end{figure}

% Esto es sólo a nivel teórico: 
%	Even though the total cross section of the \tHq production with the \schannel is much more smaller than the one for \tchannel (see Table \pablo{crear tabla}) and
%	the discrepancy probably falls within  the uncertainty band, the study of the \schannel would increase precision to the analysis.

% poner las secciones eficaces calculadas con Sherpa : Poner la configuración que utiliza Sherpa para hacer estas secciones
% https://cds.cern.ch/record/2798242/files/ATL-COM-PHYS-2021-1081.pdf. 
% Nuestras nuestras son con MC@NLO, que sólo tiene el \tchannel 


%\pablo{
%\begin{itemize}
%	\item El \schannel no se estudia debido a su pequeña sección eficaz en comparación a el \tchannel ¿verdad?. No, el \schannel no se utiliza en nuestras muestras simuladas porque la contribución de este canal es baja \pt y como ponemos cortes en \pt, la contribución del \schannel se supirme. 
%	\item Está bien expresada la eq. \ref{eq:Chap1:sigmathqM}  
%	\item ¿Por qué el SM predice que los diagramas de \tH and de $\PW \PH$ interfieren?
%\end{itemize}
%}

%  tH ::  tW-channel
\paragraph{\tH production in the \tW}\mbox{}\\
The production of the Higgs boson in association with a top quark and \PW boson
(\tWH) is a process that can be easily defined at LO accuracy in QCD and in the 5FS, i.e. 
through the partonic process $\Pgluon \Pbottom \rightarrow \Ptop \PW (\PHiggs)$\cite{Demartin:2016axk}.
When NLO corrections are applied, the \tWH interferes with the LO \ttH production.
This arise from the $\Pgluon \Pgluon  \rightarrow \Ptop \PW \Pbottom (\PHiggs)$
with a resonant \APtop interfering with $\Pgluon \Pgluon  \rightarrow \ttbar (\PHiggs)$.
This makes the \tWH process difficult to distinguish from the \ttH, which has a cross
section of one order of magnitude larger.

\begin{comment}
\begin{figure}
\centering
\begin{subfigure}{.4\textwidth}
  \centering
  \includegraphics[width=.9\linewidth]{Chapter1/tW_A}
  \caption{}
  \label{fig:Chap1:tH:tWH:A}
\end{subfigure}%
\begin{subfigure}{.4\textwidth}
  \centering
  \includegraphics[width=.9\linewidth]{Chapter1/tW_B}
  \caption{}
  \label{fig:Chap1:tH:tWH:B}
\end{subfigure}%
\caption{LO Feynman diagrams for \tW production in the 5FS.}
\label{fig:Chap1:tH:tW}
\end{figure}
\end{comment}

\begin{figure}
\centering
\begin{subfigure}{.37\textwidth}
  \centering
  \includegraphics[width=.9\linewidth]{Chapter1/tWH_A}
  \caption{}
  \label{fig:Chap1:tH:tWH:A}
\end{subfigure}%
\begin{subfigure}{.27\textwidth}
  \centering
  \includegraphics[width=.9\linewidth]{Chapter1/tWH_B}
  \caption{}
  \label{fig:Chap1:tH:tWH:B}
\end{subfigure}%
\begin{subfigure}{.27\textwidth}
  \centering
  \includegraphics[width=.9\linewidth]{Chapter1/tWH_C}
  \caption{}
  \label{fig:Chap1:tH:tWH:C}
\end{subfigure} %\quad
\begin{subfigure}{.36\textwidth}
  \centering
  \includegraphics[width=.87\linewidth]{Chapter1/tWH_D}
  \caption{}
  \label{fig:Chap1:tH:tWH:D}
\end{subfigure}%
\begin{subfigure}{.27\textwidth}
  \centering
  \includegraphics[width=.9\linewidth]{Chapter1/tWH_E}
  \caption{}
  \label{fig:Chap1:tH:tWH:E}
\end{subfigure}%

\caption{LO Feynman diagrams for $\Ptop\PWminus\PHiggs$ production in the 5FS. 
Diagrams extracted from reference \cite{Demartin:2016axk}.}
\label{fig:Chap1:tH:tW}
\end{figure}


Regarding the possibility of finding this process in the LHC it is difficult to know if it possible to
observe it over the \ttH signal, which already is a rare process. 

Anyways, the MC simulated \tWH samples should be taken into account into account
for the \tHq search, where it is a background. Alternatively, in a more extended \tH search,
it would be beneficial considering the \tWH process part of the signal along with \tHq. 
By doing this, we would have a more comprehensive view of the associated
top-Higgs production. Although this analysis is outside the scope of this manuscript, 
some studies have been done in this regard.


%\begin{itemize}
%	\item \tchannel production 
%	\item \schannel production: Vanishing due to its small cross section. 
%	\item associated production with and on-shell \PW boson
%\end{itemize}

\subsubsection{Higgs characterisation in \tH}
%\paragraph{Higgs characterisation in \tH}\mbox{}\\
%\pablo{Sección 4 de Demartin. Esto quizás se pueda mover al \ref{sec:Chap1:tHq}}
The Higgs characterisation model used in the thesis is the one described in reference \cite{Demartin:2015uha}.
Let's consider a spin-0 particle with a \CP-violating Yukawa interaction with the top quark, $X_0$.
This $X_0$ particle couples to both scalar and pseudoscalar fermionic densities, and its
interaction with the \PW boson is the one described by the SM. The reason to call this particle $X_0$
instead of $H$ is because its description does not correspond to the typical realisation of the Higgs
but, in practise, we are referring to the Higgs. Within this model, the term in the effective Lagrangian
that describes the Higgs-top Yukawa coupling below the EWSB scale is:
\begin{equation}
\label{eq:chap1:Lagrangian:tHq:A}
	\mathcal{L} = -\bar{\psi_{t}}[\text{cos}(\alpha)\kappa_{Htt} g_{Htt}  +  i \text{sen}(\alpha)\kappa_{Att} g_{Att} \gamma^{5}]\psi_{t}X_{0}\, ,
\end{equation}
where $\psi_{t}$ and $X_{0}$ represent the top quark and the Higgs boson respectively and $\alpha$ is the \CP mixing phase.
The $\kappa_{Htt}$ and $\kappa_{Att}$ are real-dimensionless-rescaling parameters. 
Finally, $g_{Htt} = g_{Att} = \frac{\mtop}{v} = \frac{\yt}{\sqrt{2}}$. The Lagrangian \ref{eq:chap1:Lagrangian:tHq:A} can
be rewritten as:
\begin{equation}
\label{eq:chap1:Lagrangian:tHq:b}
	\mathcal{L} = -\frac{\yt}{\sqrt{2}}\bar{\psi_{t}}[\text{cos}(\alpha)\kappa_{Htt} +  i \text{sen}(\alpha)\kappa_{Att} \gamma^{5}]\psi_{t}X_{0}\, .
\end{equation}
The advantage of this top-Higgs parametrisation is that is simple to interpolate between
the \CP-even ($\text{cos}(\alpha) = 1$ and $\text{sen}(\alpha) = 0$) 
and the \CP-odd ($\text{cos}(\alpha) = 0$ and $\text{sen}(\alpha) = 1$) escenarios. 
The SM coupling corresponds to the \CP-even: $\mathcal{L} = -\frac{\yt}{\sqrt{2}} \bar{\psi_{t}} \psi_{t} X_{0} \,$.

The proposed Lagrangian for the interaction of the Higgs with a top quark is 
based on considering the SM an effective field theory (EFT) applicable only
up to energies no exceeding certain scale $\Lambda$ \cite{Grzadkowski:2010es}.

Figure \ref{fig:Chap1:xsec_vs_yt_tH_ttH} shows the cross section for the $\Ptop X_{0}$ production 
in the \tchannel as function of the \CP-mixing angle. For comparison, the $\ttbar X_0$ is also included.
In the same way that $\Ptop X_{0}$ models the \tHq process, $\ttbar X_0$ models the \ttH.
The uncertainty band is derived from the choice of scale ($\mu$) and the FS dependence.
The values of $\kappa_{Htt}$ and $\kappa_{Att}$ in  Figure \ref{fig:Chap1:xsec_vs_yt_tH_ttH} 
are set to reproduce the SM expectation for the gluon fusion cross-section. 


\begin{figure}
    \centering
    \includegraphics[width = 0.91\textwidth]{Chapter1/xsec_vs_yt_tH_ttH}
    \caption{NLO cross-section as a function of the \CP-mixing angle for \tchannel $\Ptop X_{0}$ and $\ttbar X_0$
    at $\CM = 13\,$TeV. 
    The $X_0$ is represents a general \CP-violating Higgs boson.
    Note that while the \ttH cross-section degenerate under the 
    transformation $\yt \rightarrow -\yt^{SM}$, this is not the case for $\sigma(\tHq)$, which is sensible to $\alpha$.}
    \label{fig:Chap1:xsec_vs_yt_tH_ttH}
\end{figure}

The first relevant appreciation by looking at Figure \ref{fig:Chap1:xsec_vs_yt_tH_ttH} 
is that the \ttH cross-section is symmetric
around a \CP angle of $\alpha = \pi/2$. This implies that by measuring 
$\sigma(\ttH)$ it would not be possible to discriminate between the
the \CP-odd and \CP-even scenarios. However, for the \tHq production,
this degeneracy is removed by the interference of the LO diagrams as it 
is described in Section \ref{sec:Chap1:tH:SensibilityToYukawa}.
 
% --> \CP Properties of Higgs Boson Interactions with Top Quarks in the \tH and \ttH processes at ATLAS: \url{https://inspirehep.net/literature/1790698} \\ % https://arxiv.org/pdf/2004.04545.pdf

% Since only 1\% of all Higgs bosons are produced in association with top quarks, the observation of 
% such processes is very challenging. Notably, it is a immensely ambitious task to measure the associated 
% production of a Higgs with a single top which has an extremely low cross section as can be seen in 
% Figure \ref{fig:Chap1:Higgs:CrossSection:In}. 

\subsubsection{\tHq sensibility to \yt}
\label{sec:Chap1:tH:SensibilityToYukawa}
%\paragraph{\tHq sensibility to \yt}\mbox{}\\
As already mentioned, the \tH production is among the few LHC processes that are 
sensible to the relative size and phase between
the couplings of the Higgs top and the Higgs to the gauge bosons. 
% As already commented in the introduction of the Section \ref{sec:Chap1:top-Higgs}, 
% the main interest of the \tHq process is that it is among the few processes in the LHC
% that are sensible to the relative size and phase between
% the couplings of the Higgs top and the Higgs to the gauge bosons. 
The other mechanisms capable of determining this relative sign 
are $\PHiggs \rightarrow \Pgamma \Pgamma$ and $\Pg \Pg \rightarrow \PZ \PZ$.

For the \tHq, this ability is due to the fact that in the SM the \tHq production of the where 
the \PHiggs couples to the $\PW$ (Figure \ref{fig:Chap1:tHq:Feynman_LO} (b)) interfere 
destructively with those in which the \PH is radiated from the top (Figure \ref{fig:Chap1:tHq:Feynman_LO} (a)). 
As its explained in Section \ref{sec:Chap2:LHC:lumi}, the cross section is 
proportional to the square of the matrix element, $\mathcal{M}$, and if there are several diagrams 
for a same process, the matrix elements have to be summed before squaring leading to interference terms. For the \tHq production:
\begin{equation}\label{eq:Chap1:sigmathqM}
	\sigma_{\tHq} \propto |\mathcal{M}_{\Pq \Pq \rightarrow \tHq}|^{2} = | \mathcal{M}_{\Pq \Pq \rightarrow \tHq_{\PW \PH}} + \mathcal{M}_{\Pq \Pq \rightarrow \tHq_{tH}} |^{2} \, .
\end{equation}
When squaring the scattering amplitude, the destructive interference\footnote{By destructive interference 
is meant that the relative sign between $\mathcal{M}_{\Pq \Pq \rightarrow \tHq_{\PW \PH}}$ and $\mathcal{M}_{\Pq \Pq \rightarrow \tHq_{tH}}$ 
is negative.} term decreases the $\sigma_{\tHq}$. 
This behaviour makes the \tHq cross-section exceptionally sensible to the departures of \yt from the SM predictions. 
Typically, the destructive interference yields a reduction in the rate as compared to the
contribution from each individual diagram by about an order of magnitude \cite{Tait:2000sh}.
Therefore, in the presence of non-SM new physics, a positive relative sign between the \yt and the $g_{HVV}$ couplings would imply that the
amount of \tHq events recorded should increase a factor of $\sim 13$ over the SM expectations, surpassing the number of events from \ttH production \cite{Biswas:2013xva}.

This can be clearly seen in Figure \ref{fig:Chap1:xsec_vs_yt_tH_ttH}. 
In contrast to the cross-section for \ttH, which degenerates ($\sigma(\ttH, \yt) = \sigma(\ttH, -\yt)$),
the $\sigma(\tHq)$ increases with the \CP-mixing angle.




%\subsubsection{\tH Standard Model}
%\subsubsection{\tH Charge-Parity}
\subsubsection{ATLAS and CMS results}


In order to gather the necessary information, the widest campaign of measurements has to 
be undertaken, including as many possible decay modes. 
In this context, the scope of this thesis is the study of the production \tH with a final state 
characterised by two light leptons (\Plepton), i.e. electrons (\Pepm) or muons (\Pmupm), and one
hadronically decaying tau lepton (\tauhad). This signature is usually referred as dileptau or lep-had 
channel and is denoted by \dileptau.

The \tHq production has already been studied at LHC... \pablo{Wait for 22nd may LHCP talks. There will be updated results.sx}
%\pablo{Usar los papers the \tHq de CMS la sección 5 de Demartin \cite{Demartin:2015uha}}


%POSTAMBLE
\begin{comment}

\bibliographystyle{../../jhep}
\bibliography{../../biblio}

\end{document}

%ENDPOSTAMBLE
\end{comment}
