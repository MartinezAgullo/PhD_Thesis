%PREAMBLE
\begin{comment}
%\documentclass[11pt]{article}
%\newcommand{\pablo}[1]{\textcolor{blue}{{\bf  #1}}}
\newcommand{\carlos}[1]{\textcolor{red}{{\bf  #1}}}
\newcommand{\fabio}[1]{\textcolor{purple}{{\bf  #1}}}
\newcommand{\susana}[1]{\textcolor{violet}{{\bf  #1}}}

% HEP names :: https://ctan.javinator9889.com/macros/latex/contrib/hepnames/hepnames.pdf

\DeclarePairedDelimiter\bra{\langle}{\rvert}
\DeclarePairedDelimiter\ket{\lvert}{\rangle}
\DeclarePairedDelimiterX\braket[2]{\langle}{\rangle}{#1 \delimsize\vert #2}

\newcommand*{\yt}{\ensuremath{y_{t}}\xspace}
\newcommand*{\tchannel}{\ensuremath{t\text{-channel}}\xspace}
\newcommand*{\schannel}{\ensuremath{s\text{-channel}}\xspace}
%\newcommand*{\lepT}{\ensuremath{\Pl_{\Pt}}\xspace}
%\newcommand*{\lepH}{\ensuremath{\Pl_{\PHiggs}}\xspace}

%\newcommand*{\muR}{\ensuremath{\mu_{\text{R}}}\xspace}
%\newcommand*{\muF}{\ensuremath{\mu_{\text{F}}}\xspace}

% Add external packages
\usepackage[italic]{hepnicenames}

%%%%%%%%%%%%%%%%%%%%%%%
%  From NA-HIGG-2020-02-INT1-defs.sty    %
%%%%%%%%%%%%%%%%%%%%%%%
% Basic tHq-related macros
%\newcommand*{\tHq}{\ensuremath{\Pqt{}\PH{}\Pq}\xspace}
%\newcommand*{\tHq}{\ensuremath{\Ptop \PHiggs \Pq}\xspace}
%\newcommand*{\tHq}{\Pqt{}\PH{}\Pq}
\newcommand*{\tHq}{\ensuremath{tHq}\xspace}
\newcommand*{\tH}{\ensuremath{\Pqt{}\PH{}}\xspace}
\newcommand*{\tHqsec}{\texorpdfstring{\Pqt{}\PH{}\Pq}{tHq}}
\newcommand*{\tHqML}{\ensuremath{\Pqt{}\PH{}\Pq\,(\text{ML})}\xspace}
\newcommand*{\tHqbb}{\ensuremath{\Pqt{}\PH{}\Pq\,(\bbbar)}\xspace}
\newcommand*{\tbarHq}{\Paqt{}\PH{}Pq}
\newcommand*{\tHbb}{\ensuremath{\tHq (\PH \to \bbbar)}\xspace}
\newcommand*{\tHtautau}{\ensuremath{\tHq (\PH \to \Pgt{}\Pgt)}\xspace}
\newcommand*{\dR}{\ensuremath{\Delta R}\xspace}
\newcommand*{\trexfitter}{TRExFitter\xspace}
\newcommand*{\thqloop}{\texttt{tHqLoop}\xspace}


\newcommand*{\MpT}{\ensuremath{\vec{p}_{\text{T}}^{\text{miss}}}\xspace}
\newcommand*{\mtw}{\ensuremath{m_{\text{T}}(\Pl,\MET)}}
\newcommand*{\mlb}{\ensuremath{m_{\Pl\Pqb}}}
\newcommand*{\mOSSF}{\ensuremath{m_{\text{OSSF}}}\xspace}

% Background processes
\newcommand*{\ttX}{\ensuremath{\Pqt{}\Paqt{}X}\xspace}
\newcommand*{\tX}{\ensuremath{\Pqt{}X}\xspace}

%\newcommand*{\ttX}{\Pqt{}\Paqt{}+X}
\newcommand*{\ttH}{\Pqt{}\Paqt{}\PH}
%\newcommand*{\ttH}{\ensuremath{\Pqt{}\Paqt{}\PH}\xspace}
\newcommand*{\ttZ}{\Pqt{}\Paqt{}\PZ}
\newcommand*{\ttV}{\ensuremath{\Pqt{}\Paqt{}V}\xspace}
\newcommand*{\ttW}{\Pqt{}\Paqt{}\PW}
\newcommand*{\ttWj}{\ensuremath{t\bar{t}W+j}\xspace}
\newcommand*{\tZq}{\Pqt{}\PZ{}\Pq}
\newcommand*{\tWZ}{\Pqt{}\PW{}\PZ}
\newcommand*{\tWH}{\Pqt{}\PW{}\PH}
\newcommand*{\tHW}{\Pqt{}\PW{}\PH}
\newcommand*{\tW}{\Pqt{}\PW}
\newcommand*{\Wt}{\Pqt{}\PW}
\newcommand*{\diboson}{diboson\xspace}
\newcommand*{\Diboson}{Diboson\xspace}
\newcommand*{\triboson}{triboson\xspace}
\newcommand*{\Triboson}{Triboson\xspace}
\newcommand*{\Vjets}{\ensuremath{V\text{+\,jets}}\xspace}

\newcommand*{\ttt}{\ensuremath{ttt}\xspace}
\newcommand*{\tttt}{\ensuremath{t\bar{t}t\bar{t}}\xspace}
\newcommand*{\ggH}{\ensuremath{ggH}\xspace}
\newcommand*{\qqH}{\ensuremath{qqH}\xspace}
\newcommand*{\WH}{\ensuremath{WH}\xspace}
\newcommand*{\ZH}{\ensuremath{ZH}\xspace}

% Fake leptons
\newcommand*{\elHF}{\ensuremath{e_{\text{HF}}}\xspace}
\newcommand*{\muHF}{\ensuremath{\mu_{\text{HF}}}\xspace}
\newcommand*{\elCo}{\ensuremath{e_{\text{conv}}}\xspace} 
\newcommand*{\kelHF}{\ensuremath{\mu(e_{\text{HF}})}\xspace}
\newcommand*{\kmuHF}{\ensuremath{\mu(\mu_{\text{HF}})}\xspace}
\newcommand*{\kelCo}{\ensuremath{\mu(e_{\text{conv}})}\xspace} 

% Signal regions
\newcommand*{\dileptau}{\ensuremath{2\Pl+1\tauhad}\xspace}
\newcommand*{\dilepOStau}{\ensuremath{2\Pl\,\text{OS}+1\tauhad}\xspace}
\newcommand*{\dilepSStau}{\ensuremath{2\Pl\,\text{SS}+1\tauhad}\xspace}

\newcommand*{\onelep}{\ensuremath{1\Pl}\xspace}
\newcommand*{\dilep}{\ensuremath{2\Pl}\xspace}
\newcommand*{\dilepOS}{\ensuremath{2\Pl\,\text{OS}}\xspace}
\newcommand*{\dilepSS}{\ensuremath{2\Pl\,\text{SS}}\xspace}
%\newcommand*{\SS}{\ensuremath{\text{SS}}\xspace}
%\newcommand*{\OS}{\ensuremath{\text{OS}}\xspace}

\newcommand*{\trilep}{\ensuremath{3\Pl}\xspace}
\newcommand*{\lepditau}{\ensuremath{1\Pl+2\tauhad}\xspace}




\newcommand*{\lumi}{\ensuremath{\mathcal{L}}\xspace}
\newcommand*{\lumiunits}{$\,$ cm$^{2} \,$s$^{-1}$\xspace}


%Other
\newcommand*{\CM}{\ensuremath{\sqrt{s}}\xspace}
 
\newcommand*{\Wtb}{\ensuremath{tWb}\xspace}
\newcommand*{\tWb}{\Wtb}
%\newcommand*{\tWb}{\ensuremath{tWb}\xspace}
\newcommand*{\pfour}{\ensuremath{\boldsymbol{\textrm{p}}}\xspace} 

\newcommand*{\tchan}{\ensuremath{t}-channel}
\newcommand*{\mtop}{\ensuremath{m_{\Ptop}}\xspace}
%\newcommand*{\mH}{\ensuremath{m_H}\xspace}
%\newcommand*{\HT}{\ensuremath{H_{\text{T}}}\xspace}

%\newcommand*{\PWplus}{\ensuremath{\PW^{+}}\xspace}
%\newcommand*{\PWminus}{\ensuremath{\PW^{-}}\xspace}
%\newcommand*{\Pgamma}{\ensuremath{\gamma}\xspace}

 \newcommand*{\greekphys}{\ensuremath{\varphi\upsilon\sigma\iota\kappa \eta}\xspace}
 \newcommand*{\greekatom}{\ensuremath{\alpha \tau o \mu o \nu}\xspace}
 \newcommand{\emu}{\ensuremath{\Pe/\Pmu}\xspace}

\newcommand*{\momentum}{\ensuremath{\overrightarrow{p}}\xspace} 
%\newcommand*{\CP}{\ensuremath{\mathcal{CP}}\xspace}
\newcommand*{\CP}{CP\xspace}

% Decays (Please, check latex/atlasprocess.sty and latex/atlasparticle.sty for more definitions!)
\newcommand{\bb}{\ensuremath{\Pqb\Paqp}\xspace}
\newcommand{\WW}{\ensuremath{\PW\PW^{*}}\xspace}
\newcommand{\ZZ}{\ensuremath{\PZ\PZ^{*}}\xspace}
\newcommand{\Higgsdecays}{\ensuremath{\PH \rightarrow b\bar{b}, \WW, \ZZ, \tau\tau}\xspace}
\newcommand{\Higgsdecayslep}{\ensuremath{\PH \rightarrow \WW, \ZZ, \tau\tau}\xspace}
\newcommand{\HWW}{\ensuremath{\PH \rightarrow \WW}\xspace}
\newcommand{\HZZ}{\ensuremath{\PH \rightarrow \ZZ}\xspace}

% reconstruction definitions
\newcommand{\pnutop}{\ensuremath{\vec{p^{\Pnu, \text{top}}}}\xspace}
\newcommand{\pnutopx}{\ensuremath{p^{\Pnu, \text{top}}_x}\xspace}
\newcommand{\pnutopy}{\ensuremath{p^{\Pnu, \text{top}}_y}\xspace}
\newcommand{\pnutopz}{\ensuremath{p_{z}(\Pnu_{\text{top}})}\xspace}
\newcommand{\pnutopT}{\ensuremath{\pT(\Pnu_{\text{top}})}\xspace}
\newcommand{\phinutop}{\ensuremath{\phi(\Pnu_{\text{top}})}\xspace}
\newcommand{\pltop}{\ensuremath{\vec{p^{\Plepton, \text{top}}}}\xspace}
\newcommand{\pltopx}{\ensuremath{p^{\Plepton, \text{top}}_x}\xspace}
\newcommand{\pltopy}{\ensuremath{p^{\Plepton, \text{top}}_y}\xspace}
\newcommand{\pltopz}{\ensuremath{p^{\Plepton, \text{top}}_z}\xspace}
\newcommand{\pltopT}{\ensuremath{\pT(\Plepton_{\text{top}})}\xspace}
\newcommand{\philtop}{\ensuremath{\phi(\ell_{\text{top}})}\xspace}
\newcommand{\phibtop}{\ensuremath{\phi(b_{\text{top}})}\xspace}
\newcommand{\tauvis}{\ensuremath{\Ptau_{\text{vis}}}\xspace}

\newcommand{\leptop}{\ensuremath{\Plepton^{\text{top}}}\xspace}
\newcommand{\lepH}{\ensuremath{{\Plepton^{\PH}}}\xspace}
\newcommand{\Hvismass}{\ensuremath{m_{\PH}^{\text{vis}}}\xspace}
\newcommand{\toprecomass}{\ensuremath{\mtop^{\text{reco}}}\xspace}
% \newcommand{\toprecomass}{\ensuremath{\Pqt_{\text{reco}}^{\text{m}}\xspace}}
% \newcommand{\Hvismass}{\ensuremath{\PH_{\text{vis}}^{\text{m}}\xspace}}
\newcommand{\MMC}{\texttt{MissingMassCalculator}\xspace}

% luminosity (2015)
\newcommand{\lumiFifteenRelUnc}{1.13} % in [%]
\newcommand{\lumitagFifteen}{{\small\texttt{OfLumi-13TeV-008}}}
%\newcommand{\lumiFifteenInPbNoUnits}{3219.56}
%\newcommand{\lumiFifteenInFbNoUnits}{3.2}
\newcommand{\lumiFifteenInPbNoUnits}{3244.54} % final luminosity recommendation for Run 2 analyses (https://twiki.cern.ch/twiki/bin/viewauth/Atlas/LuminosityForPhysics#2015_2018_13_TeV_proton_proton_f)
\newcommand{\lumiFifteenInFbNoUnits}{3.2}
%\newcommand{\dataperiodsFifteen}{D--J}
\newcommand{\dataperiodsFifteen}{D--H,J}
\newcommand{\firstdatarunFifteen}{276262}
\newcommand{\lastdatarunFifteen}{284484}
\newcommand{\datarunsFifteen}{\firstdatarunFifteen--\lastdatarunFifteen}
\newcommand{\dataeventsFifteen}{220.58M}

% luminosity (2016)
\newcommand{\lumiSixteenRelUnc}{0.89} % in [%]
\newcommand{\lumitagSixteen}{{\small\texttt{OfLumi-13TeV-009}}}
%\newcommand{\lumiSixteenInPbNoUnits}{32988.1}
%\newcommand{\lumiSixteenInFbNoUnits}{33.0}
% final luminosity recommendation for Run 2 analyses (https://twiki.cern.ch/twiki/bin/viewauth/Atlas/LuminosityForPhysics#2015_2018_13_TeV_proton_proton_f)
\newcommand{\lumiSixteenInPbNoUnits}{33402.2}
\newcommand{\lumiSixteenInFbNoUnits}{33.4}
%\newcommand{\dataperiodsSixteen}{A--L}
\newcommand{\dataperiodsSixteen}{A--G,I,K,L}
\newcommand{\firstdatarunSixteen}{297730}
\newcommand{\lastdatarunSixteen}{311481}
\newcommand{\datarunsSixteen}{\firstdatarunSixteen--\lastdatarunSixteen}
\newcommand{\dataeventsSixteen}{1057.84M}

% luminosity (2017)
\newcommand{\lumiSeventeenRelUnc}{1.13} % in [%]
\newcommand{\lumitagSeventeen}{{\small\texttt{OfLumi-13TeV-010}}}
%\newcommand{\lumiSeventeenInPbNoUnits}{44307.4}
%\newcommand{\lumiSeventeenInFbNoUnits}{44.3}
% final luminosity recommendation for Run 2 analyses (https://twiki.cern.ch/twiki/bin/viewauth/Atlas/LuminosityForPhysics#2015_2018_13_TeV_proton_proton_f)
\newcommand{\lumiSeventeenInPbNoUnits}{44630.6}
\newcommand{\lumiSeventeenInFbNoUnits}{44.6} 
%\newcommand{\dataperiodsSeventeen}{B--K}
\newcommand{\dataperiodsSeventeen}{B--F,H,I,K}
\newcommand{\firstdatarunSeventeen}{325713}
\newcommand{\lastdatarunSeventeen}{340453}
\newcommand{\datarunsSeventeen}{\firstdatarunSeventeen--\lastdatarunSeventeen}
%%%\newcommand{\datarunsSeventeen}{324320--341649}
\newcommand{\dataeventsSeventeen}{1340.80M}

% luminosity (2018)
\newcommand{\lumiEightteenRelUnc}{1.10} % in [%]
\newcommand{\lumitagEightteen}{{\small\texttt{OfLumi-13TeV-010}}}
%\newcommand{\lumiEightteenInPbNoUnits}{58450.1}
%\newcommand{\lumiEightteenInFbNoUnits}{58.5}
% final luminosity recommendation for Run 2 analyses (https://twiki.cern.ch/twiki/bin/viewauth/Atlas/LuminosityForPhysics#2015_2018_13_TeV_proton_proton_f)
\newcommand{\lumiEightteenInPbNoUnits}{58791.6}
\newcommand{\lumiEightteenInFbNoUnits}{58.8}
\newcommand{\lumiEightteenInPb}{\SI{\lumiEightteenInPbNoUnits}{\per\pb}}
\newcommand{\lumiEightteenInFb}{\SI{\lumiEightteenInFbNoUnits}{\per\fb}}
%\newcommand{\dataperiodsEightteen}{B--Q}
\newcommand{\dataperiodsEightteen}{B--D,F,I,K,L,M,O,Q}
\newcommand{\firstdataruEightteen}{348885}
\newcommand{\lastdatarunEightteen}{364292}
\newcommand{\datarunsEightteen}{\firstdataruEightteen--\lastdatarunEightteen}
%%%\newcommand{\datarunsEightteen}{348197--364292}
\newcommand{\dataeventsEightteen}{1716.77M}

% luminosity (2015+2016+2017)
%\newcommand{\lumiInPb}{80515.06~\invpb}
% \newcommand{\partlumi}{\SI{80.52}{\per\fb}}
%\newcommand{\datafirstyear}{2015}
%\newcommand{\datalastyear}{2017}

% luminosity (2015+2016+2017+2018)
%
% https://twiki.cern.ch/twiki/bin/viewauth/Atlas/LuminosityForPhysics#2015_2018_13_TeV_proton_proton_f
% final luminosity recommendation for Run 2 analyses (central value + uncertainty)
\newcommand{\lumiRelUnc}{0.83} % in [%]
\newcommand{\lumiInPbNoUnits}{140068.94} % in pb-1
\newcommand{\lumiInFbNoUnits}{140} % in fb-1
\newcommand{\lumiWithUnc}{\ensuremath{140.1 \pm 1.2}\,\si{\per\fb}} % in fb-1
%
% old recommendation
%\newcommand{\lumiRelUnc}{1.7} % in [%]
%\newcommand{\lumiInPbNoUnits}{138965.16} % in pb-1
%\newcommand{\lumiInFbNoUnits}{139} % in fb-1
%\newcommand{\lumiWithUnc}{\ensuremath{\lumiInFbNoUnits \pm 2.4}\,\si{\per\fb}} % in fb-1
\newcommand{\dataeventsAll}{4335.99M}
%
\newcommand{\lumiInPb}{\SI{\lumiInPbNoUnits}{\per\pb}}
%\newcommand{\lumi}{\SI{\lumiInFbNoUnits}{\per\fb}}
\newcommand{\datafirstyear}{2015}
\newcommand{\datalastyear}{2018}

% % tunes and PDF sets
\def\cteq{CTEQ6L1\xspace}
\def\ctten{CT10\xspace}
\def\cttennlo{CT10\,NLO\xspace}
\def\cttennnlo{CT10\,NNLO\xspace}
\def\ctfourteennlo{CT14\,NLO\xspace}
\def\ctfourteennnlo{CT14\,NNLO\xspace}
\def\nnpdfnnlo{NNPDF3.0\,NNLO\xspace}
\def\nnpdfnlofourflav{NNPDF3.0\,NLO\,nf4\xspace}
\def\nnpdfnlo{NNPDF3.0\,NLO\xspace}
\def\nnpdftwonlo{NNPDF2.3\,NLO\xspace}
\def\nnpdftwo{NNPDF2.3\,LO\xspace}
\def\nnpdftwofiveflav{NNPDF2.3\,5f\,FFN\xspace}
\def\mstw{MSTW2008\,NLO\xspace}
\def\a14{A14\xspace}
\def\auet{AUET2\xspace}
\def\aznlo{AZNLO\xspace}
\def\mmhtnnlo{MMHT2014\,NNLO\xspace}
\def\mmhtnlo{MMHT2014\,NLO\xspace}
\def\mmhtlo{MMHT2014\,LO\xspace}
\def\mstwnlo{MSTW2008\,68\%\,CL\,NLO \xspace}
\def\mstwnnloninety{MSTW2008\,90\%\,CL\,NNLO \xspace}
\def\ueee{UE-EE-5\xspace}




 
\endinput


%\begin{document}
asdf
%ENDPREAMBLE
\end{comment}

\chapter{Conclusion}
%{\LARGE \textbf{Theoretical framework}\\}

%\tableofcontents
\label{chap:Conclusion}
\vspace*{0.1 cm} 
\hspace*{200pt} \\
%\hspace*{175pt} \textit{Terminamos.} \\
%\hspace*{200 pt}     ---\textsc{Mi ex} \\% \textit{} \\
\hspace*{0.25\textwidth} \textit{Cuando el río es lento y se cuenta con una buena } \\
\hspace*{0.25\textwidth} \textit{bicicleta o caballo sí es posible bañarse dos (y hasta tres, } \\
\hspace*{0.25\textwidth} \textit{de acuerdo con las necesidades higiénicas de cada } \\
\hspace*{0.25\textwidth} \textit{quién) veces en el mismo río.} \\
\hspace*{205pt} ---\textsc{Augusto Monterroso,} \\% \textit{} \\
\hspace*{240 pt}     \textsc{Heraclitana (1978)} \\% \textit{} \\
\vspace*{2cm} 


%%%%%%%%%%%%%%%
%         Conclusion              %
%%%%%%%%%%%%%%%
This thesis presents the study to measure of  the direct production of a Higgs boson in association
with a single-top quark, focusing on final states with two light-flavour leptons and one hadronically-decaying-\Ptau 
lepton, employing the ATLAS detector. Attending to the relative charge between the light lepton, 
this investigation is divided in two channels: \dilepOStau and \dilepSStau.

The search for such a rare process is motived by the intricate interplay between two fundamental 
particles: the Higgs boson and the top quark. On on hand te Higgs boson plays a critical role in our understanding 
of mass acquisition by particles through the Spontaneous Symmetry Breaking mechanism.
On the other hand, top quark, notable for being the most massive particle in the SM and the only one that 
decays before its hadronisation. Therefore, the Yukawa coupling between these two particles is expected 
to be the largest in the SM and it can be measured trough this interaction.  This measurement is central
to the experimental program of the LHC and it could hint at possible 
CP violation, influencing the \tHq production cross-section.
 
In this thesis the theoretical foundation of physics of the top quark and Higgs boson are
discussed, a review of the ATLAS detector and its performance is given, and the simulation chain 
and object reconstruction are described. Then, the search of the \tHq production is carefully detailed. 

This analysis uses proton-proton collisions at $\CM=13$~TeV from the ATLAS detector during Run 2 of the LHC
with a total integrated luminosity of 140 fb$^{-1}$. The parton-level information is implemented and used to
reconstruct the \tHq process. The origin of the light-lepton is assed via the use of BDT. 
Then the rate of misidentified particles is addressed using the template fit method to correct
the MC yields. 

Afterwards, by using several BDTs, the SRs, CRs and VRs of the analysis are defined.
Utilising these regions, a profile-likelihood-binned fit with Asimov data and real data  is 
conducted to determine the normalisation of the \ttW process and the signal strength of 
the \tHq production. 
The culmination of this analysis is the determination of limits on the \tHq production cross-section.
The \tHq signal-strength values are shown in Table~\ref{tab:Conclusion:SignalStrength}.
The expectation is obtained using Asimov data and the observed is obtained with the fit
to the real data. 

\begin{table}[h]
\centering
\begin{tabular}{l|c|c}
\cline{2-3}
            		& \multicolumn{2}{c}{$\mu_{\tHq}$} 		\\ \cline{2-3}
            		& Expected       & Observed			\\ \midrule
\dilepOStau 	& $\pm 21.58$  &  $-19.73 \pm 20.16$  	\\
\dilepSStau 	& $\pm 7.20$ 	&  $-2.59 \pm 5.44$          	\\ \bottomrule
\end{tabular}
\caption{Expected and observed signal strength values for the two \dileptau channels.
The uncertainty is the combination of the statistical and systematic effects.}
\label{tab:Conclusion:SignalStrength}
\end{table}

This result is fully compatible with the SM. 
The limits at 95\% CL on the \tHq cross-section are in Table~\ref{tab:Conclusion:UpperLimit}.

\begin{table}[h]
\centering
\begin{tabular}{l|c|c}
\cline{2-3}
            		& \multicolumn{2}{c}{$\mu_{\tHq}^{\text{CL95}}$} \\ \cline{2-3}
            		& Expected       				& Observed       \\ \midrule
\dilepOStau 	& $57.98^{+39.38}_{-21.07}$		&    34.49            \\
\dilepSStau 	& $18.88^{+33.83}_{-11.6}$		&    13.8            \\ \bottomrule
\end{tabular}
\caption{Values of the upper limits of the signal strength at 95\% CL for the two \dileptau channels.}
\label{tab:Conclusion:UpperLimit}
\end{table}


Looking ahead, the prospects for enhancing the outcomes of this analysis are promising, driven by several anticipated developments:
\begin{itemize}
	% Identificar mejor la identificación de taus es muy difícil pero lo que se podría hacer es mejorar la 
	% determinación de los fake factors (se puede hacer con más estadística o con un método mejor).
	% Nuestro sistemático dominante se debe al método de determinación de fake factors, mejorar el método
	% llevaría a reducir este sistemático.
	%\item The normalisation of fake factors\pablo{poner algo de cómo reducir el impacto de los fake taus}
	\item The upcoming high-luminosity Run 3 promises increased statistical significance of the analysis.
		This will be very beneficial since in the \dilepSStau channel (the one with the best sensitivity)
		the statistical uncertainty is the dominant. In the \dilepOStau the uncertainty due to the statistical
		sample is similar to the systematic uncertainty and more data events will undoubtedly enrich this search.

	\item The current techniques to estimate the rates of backgrounds due to misidentified objects will also
		benefit from an increase on the statistical sample since these are data-driven methods.
		
	\item With more statistics we could explore splitting the regions used in the fit calculations
		according to the track multiplicity of the \tauhad (1-prong or 3-prong) . %1-prong // 3-rpong
		Since jets with different origin mimic differently the \tauhad depending on its prongness,
		it can be beneficial to explore this classification when defining the regions used in
		the profiled-likelihood fit.
		%the different relative background conntributions in all regions
		%considered in the fit.
		
	\item Currently there is a mild tension between the simulation and the experimental results for \ttW. 
		Therefore, progress in the simulation techniques or better understanding of these process will
		further refine the analysis. This will be useful in the \dilepSStau channel, where this is the second
		most important background.
\end{itemize}


Additionally, this analysis can be extended in several ways.
First, the region definition can be improved by combining the BDTs described in this
thesis with the NNs that are also being used by the ongoing ATLAS analysis. 
Secondly, incorporating the \tWH processes in the signal would allow to perform
a more complete \tH search and, hence, the inverted-top-Yukawa-coupling hypothesis 
can be properly tested.

From the two channels explored in this thesis, the \dilepSStau one is the 
one that can add more sensitivity to future combinations with other
\tHq channels. 

%More studies to be done:
%\begin{itemize}
	%\item Comment the results using an alternative region definition based on combination of BDT and NN
	%\item Discuss using \tH instead of \tWH
	%\item Discuss further studies using the inverted \yt
	%\item Añadir más datos mejorará la precisión estadísitica de los dos canals pero, sobre todo del canal SS 
	%	donde la statdística es la principal unncertainty conn diferencia. En OS la incertidumbre estadística 
	%	es un poco mayor que la sistemàtica.
	%\item Al tener más datos también podríamos separar los \tauhad en un 1prong y 3 prong. Esto nos permite
	%	explorar nuevas técnicas de ajuste. Los 1prong y 3 prong tienen distintos tipos de fondo
	%\item Las técnicas de estimación de tau-fake-SFs también pueden mejorar al mejorar la estadística
	%\item Los resultados obtenidos para esta tesis son comparables a los de otros
	%	canales (poner resultados Jesús). A mí me sale -3.78 $\pm$ 5 y a él algo compatible en 2LSS (5,1 $\pm$ 4.8) y en el 3L (6.3 $\pm$ 7.1)
		
%\end{itemize}

\begin{comment}
\pablo{En estos párrafos hay snipets de cosas de estadística}


\paragraph{Comments about hypothesis rejection}\mbox{}\\
%https://arxiv.org/pdf/1007.1727.pdf
In the search of a new signal process one defines:
\begin{itemize}
	\item Null hypothesis ($H_0$):  Taking into account only the backgrounds
	\item Alternative hypothesis ($H_1$): Considering both the backgrounds and the signal
\end{itemize}
When setting limits, the model with signal-plus-background hypothesis
plays the role of $H_0$, which is tested against the background-only hypothesis, $H_1$.
The level of agreement between the data and a hypothesis $H$ is given by the $p$-value.
The $p$-value is defined as the probability, under the assumption of $H$, of finding data of 
equal or greater incompatibility with the predictions of $H$. One can regard the hypothesis 
as excluded if its $p$-value is observed below a specified threshold.

Rejecting the background-only hypothesis in a statistical sense is only part of discovering 
a new phenomenon. A widely used procedure to establish discovery (or exclusion) in particle physics is based
on a frequentist significance test using a likelihood ratio as a test statistic. In addition to
parameters of interest such as the rate (cross section) of the signal process, the signal and
background models will contain in general nuisance parameters whose values are not taken
as known a priori but rather must be fitted from the data.

Let's consider the distribution of a kinematic variable measured at the experiment. 
This distribution can be presented in a binned histogram. The expected number
of events the $i^{th}$ bin is given by:
\begin{equation*}
	n_{i}^{exp} = \mu s_{i}^{exp} + b_{i}^{exp}
\end{equation*}
Where $\mu$ is the signal strength and, if $\mu=0$ corresponds tothe background-only hypothesis .
The $s_{i}$ + $b_{i}$ are defined as:
\begin{equation*}
  s_{i}^{exp} = s_{\text{tot}}^{exp} \int_{\text{bin } i} f_s(x; \theta_s) \,dx
\end{equation*}
\begin{equation*}
  b_{i}^{exp} = b_{\text{tot}}^{exp} \int_{\text{bin } i} f_b(x; \theta_b) \,dx
\end{equation*}
The integral over $x$ runs over the kinematic distribution. The functions 
$f_s(x; \theta_s$ nad $f_b(x; \theta_s)$ are the probability density functions 
(pdfs) of the variable $x$ for signal and background events.


% Bayesian vs frequentist
\paragraph{Bayesian vs frequentist}\mbox{}\\
From the frequentist point of view the probability is defined as the fraction of times an 
event occurs, in the limit of very large number ($N \to \infty$) of repeated trials
\begin{equation*}
	 \mathcal{P} = \lim_{N \to \infty} \frac{\text{Number of favorable cases}}{N}
\end{equation*}
where $N$ is the number of trials. Even though this infinity can be conceptually unpleasant, for 
LHC experiments, the amount of events is so large that this $\mathcal{P}$ definition becomes
acceptable \cite{Lista:2016chp}. 

In contrast, for the Bayesian (or subjective) probability expresses the degree of belief that a 
claim is true. Starting from a prior probability, following some observation, the probability can 
be modified into a posterior probability. The more information an individual receives, the more 
Bayesian probability is insensitive on prior probability \cite{Lista:2016chp}.

The Bayes theorem \cite{Bayes:1764vd} states that considering two events $A$ and $B$, the probability of $A$
to happen given that $B$ takes places is
\begin{equation*}
	 \mathcal{P}(A|B)= \frac{\mathcal{P}(B|A)\mathcal{P}(A)}{\mathcal{P}(B)}
\end{equation*}
where $\mathcal{P}(B|A)$ is the conditional probability of $B$ given $A$ and $\mathcal{P}(B)$ 
the probability of the event $B$ to happen. 
Here, $\mathcal{P}(A)$ has the role of prior probability while $\mathcal{P}(A|B)$ is known
as posterior probability.



\paragraph{Signal strenght}\mbox{}\\
The signal strength is defined as the ratio between the measured process
and its SM prediction. For the $\tHq$, the production signal-strength is: 
\begin{equation*}
	\mu_{\tHq} = \frac{\sigma_{\tHq}}{(\sigma_{\tHq})_{SM}} \, .
\end{equation*}
For particular decay mode $f$ its signal strength is:
\begin{equation*}
	\mu^{f} = \frac{BR^f}{(BR^f)_{SM}} \, , 
\end{equation*}
being $BR^f$ the branching ratio for the $f$ decay mode. %The subscript SM refers to the SM prediction.
Since cross-section and the BR cannot be separated without further assumptions, only the product
can measured experimentally, leading to the combined signal strength:
\begin{equation*}
	\mu_{\tHq}^{f} = \frac{\sigma_{\tHq}\cdot BR^f}{(\sigma_{\tHq})_{SM} \cdot (BR^f)_{SM}} = \mu_{\tHq} \cdot \mu^{f} \, .
\end{equation*}
In out particular case $f$ is \dilepOStau or \dilepSStau. 
\end{comment}
 


%POSTAMBLE
\begin{comment}
asdf
%\end{document}
%ENDPOSTAMBLE
\end{comment}
