
\lhead[{\bfseries \thepage}]{ \rightmark}
\rhead[Resumen \leftmark]{\bfseries \thepage}

%\markboth{Resumen de la tesis}{Resumen de la tesis}
%\addcontentsline{toc}{part}{Resumen de la tesis} 
\label{partIII}

%\begin{comment}
 \newcounter{betasect}

 \renewcommand\thesection{%
 \ifnum\value{betasect}=1%
A%%
 \else
\ifnum\value{betasect}=2%
B%%
\else
\ifnum\value{betasect}=3%
C%%
\else
\ifnum\value{betasect}=4%
D%%
\else
 \arabic{section}%%
 \fi\fi\fi\fi}%

 \newenvironment{csection}{%
 \setcounter{betasect}{1}%%
 }{%
 \setcounter{betasect}{0}%%
 }%

 \newenvironment{bsection}{%
 \setcounter{betasect}{2}%%
 }{%
 \setcounter{betasect}{0}%%
 }%
%\end{comment}


\setcounter{section}{0}

\chapter*{Búsqueda de la producción asociada de un bosón de Higgs y un quark top en el estado final con un tau hadrónico}
\addcontentsline{toc}{chapter}{Resumen de la tesis} 
\label{chap:resumen_cast}

\section{Marco teórico}
\subsection{El Model Estándar}
El Modelo Estándar de Física de Partículas (SM, por sus siglas en inglés) es un marco teórico que describe los
constituyentes básicos de la materia y sus interacciones. Es el modelo más ampliamente aceptado y confirmado experimentalmente en la física de partículas. 

El SM incluye dos tipos de partículas elementales, los fermiones y bosones. 
Los fermiones son partículas subatómicas que obedecen las reglas de la mecánica cuántica 
estadística de Fermi-Dirac. Este tipo de partícula se caracteriza por tener un spin semientero 
y obedecer el principio de exclusión de Pauli, el cual establece que dos fermiones no pueden 
ocupar el mismo estado cuántico al mismo tiempo. Los fermiones se dividen en quarks y leptones. 
Ambos tipos de fermiones son los constituyentes básicos de la materia pero son distintos entre sí.

Por un lado, los quarks son partículas que tienen carga eléctrica fraccionaria y
son la unidad fundamental de los protones y los
neutrones. Estas partículas se combinan en grupos para formar hadrones (mesones y bariones). Los bariones
incluyen los protones y los neutrones, que son las partículas subatómicas más abundantes en la materia. Los 
mesones tienen un número par de quarks, lo que hace que tengan spin entero y sean bosones.
Los quarks se dividen en seis \textit{sabores}  (no confundir con la sensación que producen
las sustancias con gusto) diferentes: arriba, abajo, encanto, extraño, arriba y abajo. 
La forma más habitual de referirse a estos es por sus nombres en ingles: up, down, charm, strange,
top y bottom. 

Por otro lado, los leptones son partículas elementales que no interactúan fuertemente.
Los leptones están divididos en dos clases: neutrinos y leptones cargados.
Los neutrinos son partículas muy ligeras y se les conoce por su interacción débil con la materia.
Los leptones cargados (electron, muón y tauón) tienen carga eléctrica entera. A diferencia de los quarks,
este tipo de partículas no se combinan entre sí para formar otras aunque, sí que que lo hacen con 
hadrones para formar, por ejemplo, los átomos.


Los otros elementos que componen el SM son los bosones, partículas de spin entero
que median las interacciones fundamentales de la física. Los bosones de gauge (spin 1)
son responsables de describir tres de las cuatro fuerzas fundamentales de la 
naturaleza\footnote{La gravedad queda fuera del SM.}: 
\begin{itemize}
	\item Interacción electromagnética: Mediada por el fotón (\Pgamma), es la teoría que estudia los fenómenos 
		eléctricos y magnéticos. Todas las partículas cargadas interactúan entre sí a través de esta fuerza.
		Las principales características de la interacción electromagnética son su infinito alcance 
		y la ausencia de masa para sus portadores. Es responsable de la estabilidad de los átomos, 
		ya que mantiene unidos a los electrones en su órbita alrededor del núcleo, y de la transmisión 
		de la luz y otras formas de radiación electromagnética.
		La teoría que describe esta interacción se denomina electrodinámica cuántica. 
	\item Interacción nuclear débil: Medida por dos bosones \PW (\PWplus y \PWminus) y el boson \PZ.
		Esta es responsable de la radioactividad beta, en la que un neutrón se descompone 
		en un protón, un electrón y un antineutrino. También es la fuerza que media la desintegración
		del quark top a un quark \Pbottom y un boson \PW. Además, la interacción nuclear débil es 
		crucial en el proceso de fusión en las estrellas, donde se combinan protones para formar 
		elementos más pesados.
		Las fuerzas nuclear débil y electromagnéticas son descritas simultáneamente por
		la teoría  electrodébil.
	\item Interacción nuclear fuerte: Mediada por el gluón, es responsable de mantener unidos los protones y 
		neutrones en el núcleo atómico. Es la interacción más fuerte de la naturaleza pero su rango de acción
		está limitado a distancias subatómicas.	
		Debido al confinamiento por color de la teoría nuclear fuerte ni los gluones ni los quarks aparecen 
		aislados (salvo a altas energías). 
		La teoría que describe esta interacción se llama cromodinámica cuántica. Esta teoría, al igual que la
		electrodinámica cuántica y la teoría electrodébil, está basada en el formalismo de la 
		teoría cuántica de campos.	
\end{itemize}


\subsection{La física del quark top}
\subsection{La física del bosón de Higgs}
\section{Dispositivo experimental}
\subsection{El gran colisionador de hadrones}
El Gran Colisionador de Hadrones (LHC, por sus siglas en inglés) es un acelerador de partículas que se encuentra
en el CERN (Centro Europeo para la Investigación Nuclear o Laboratorio Europeo de Física de Partículas 
Elementales), en Ginebra, Suiza. Fue diseñado para colisionar protones y iones pesados con alta energía, lo que
permite a los científicos estudiar la estructura subatómica de la materia y buscar nuevas partículas.

El LHC es una máquina de vanguardia que utiliza tecnología avanzada para acelerar partículas hasta velocidades 
cercanas a las de la luz antes de chocarlas entre sí. Estos choques generan partículas secundarias que se 
detectan mediante una red de detectores situados en su interior. Los datos recopilados de estas colisiones se 
utilizan para investigar la física de partículas.  Llegando a energías de $\CM = 13\,$TeV, el LHC es el mayor 
acelerador de partículas construido y constituye una herramienta clave para el avance de la ciencia.

Los cuatro principales detectores que rodean el LHC son: ATLAS, CMS, LHCb y ALICE. El primero
de estos es el experimento en el cual se desarrolla esta tesis.

\subsection{El detector ATLAS}
ATLAS es uno de los principales detectores del LHC y se utiliza para medir las propiedades de las partículas resultantes de las colisiones de hadrones. ATLAS tiene una estructura cilíndrica y es uno de los más grandes
detectores de partículas del mundo, midiendo aproximadamente 46 metros de largo y 25 metros de diámetro. 
Está compuesto por varios componentes y subcomponentes. Cada uno de estos sistemas se encarga de 
registrar un tipo de información diferente. En orden dentro hacia fuera, ATLAS está compuesto por:
\begin{itemize}
	\item Detector interno (Inner Detector, ID)
	\begin{itemize}
		\item Pixel
		\item SCT
		\item TRT	
	\end{itemize}
	\item Imán solenoidal:
	\item Calorímetro electromagnético (ECAL):
	\item Calorímetro hadrónico (HCAL):
	\item Muon Spectrometer (MS):
\end{itemize}

\section{Búsqueda de procesos \tH  con un con final \dileptau}
\subsection{Selección de eventos}
\subsection{Estimación del fondo}
\subsection{Fontes de incertidumbre}
\subsection{Resultados}
\section{Conclusiones}
